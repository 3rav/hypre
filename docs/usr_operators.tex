
\chapter{Operators: Matrices, Solvers and Preconditioners}

{\bf Operators:} Mathematically, the line between "solver" and "preconditioner" is
somewhat blurry (discussed 
below), and \hypre{} takes advantage of this property
to unify them into a single type: "Operator".
We can unify much of 
the discussion of solvers and preconditioners under the heading of Operators,
and then specialize those 
discussions.

The defining characteristics of Operators are that they take a vector as input
and return another vector as 
output. Thus, this generalization also includes matrices, as well as nonlinear
operators, transpose matrices, 
etc. 
The fundamental "use" function of an Operator is the \code{Apply} function. For
components that 
encapsulate "matrices", the \code{Apply} function corresponds to the traditional
matrix-vector multiplication. For 
solvers, \code{Apply} is the traditional "solve" function. 
Preconditioners are traditionally
referred to as being "applied", 
and the \code{Apply} function quite naturally corresponds to this application. 
Thus,
use of matrices, solvers, and 
preconditioners is done through the \code{Apply} function.

\section{Preconditioners are Solvers}

Users of \hypre{} will quickly notice that there is
no class or interface called 
"preconditioner". Instead, we have chosen to interpret preconditioners as
"solvers", based on the fact that 
both traditional solvers and preconditioners are in some senses "approximations
to the inverse" of the linear 
operator defining the system to be solved. The difference between them is one
of degrees more than a fundamental distinction, in that 
preconditioners generally are very crude and very easy to compute
approximations to the inverse, while 
solvers are more accurate, i.e. within the desired convergence tolerance. By
treating solvers and 
preconditioners uniformly within \hypre{}, we make their use uniform for users and
reduce the complexity 
of \hypre{}'s object model. We also enable novel algorithmic possibilities such as
using what are 
traditionally "solvers" as preconditioners. In fact, since Operators include
solvers and matrices, it is 
straightforward to use matrices as preconditioners, should a direct
approximation to the inverse exist. 

\section{Setup}

Most components that correspond to the traditional "matrix", as well as
some solvers in which the 
construction of the underlying matrix is handled by the solver for the user,
are set up through one or more 
of the conceptual interfaces discussed in a later chapter. Here we will
concentrate on other ways to 
construct Operators.

\subsection{Building a Krylov Solver with an Operator}

Various Krylov solvers are included in \hypre{}.
Such methods are very flexible in that they can be used to attempt
the solution of any system defined by any component that supports the
Operator interface.
