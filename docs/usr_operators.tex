
\chapter{Operators: Matrices, Solvers and Preconditioners}

{\bf Operators:} Mathematically, the line between "solver" and "preconditioner" is
somewhat blurry (discussed 
below), and \hypre{} takes advantage of this property
to unify them into a single type: "Operator".
We will discuss the common aspects of
solvers and preconditioners under the heading of Operators,
and then consider special characteristics of each separately.

The defining characteristics of Operators are that they take a vector as input
and return another vector as 
output. Thus, this generalization also includes matrices, solvers, and preconditioners,
as well as nonlinear
operators, transpose matrices, 
and other less common operators. 
The fundamental "use" function of an Operator is the \code{Apply} function, which
is the function that transforms an input vector to an output vector. 
For traditional "matrices", the \code{Apply} function corresponds to
matrix-vector multiplication. For 
solvers, \code{Apply} is the traditional "solve" function. 
Preconditioners are traditionally
referred to as being "applied", 
and the \code{Apply} function quite naturally corresponds to this functionality. 

Operators also have auxiliary uses that are available for more
expert users.
Operators can be "multiplied" or "composed" to form a new operator.
For example, given Operator A and Operator B, the result of composing
them is an operator C whose action on a vector x is equivalent to
applying B to x, and then applying A to the result, i.e. C(x)=A(B(x)).
All Operators support the Compose function. 
Most support it simply by chaining \code{Apply} functions together according
to the definition, though it is possible that the equivalent of matrix 
multiplication is implemented in certain cases.

The converse of the composition function for Operators is the Split function.
Only the Operators that also support the OperatorSplittable interface support
this function.
The Split function returns two Operators whose composition is equal to 
the original Operator.
Typically, Operators that are formed through a Compose function are Splittable.
The other major use of Splittable Operators is for preconditioners that can be 
used as ``split preconditioners'', such as incomplete factorization methods and
factorized sparse approximate inverse functions.

\section{Preconditioners are Solvers}

Users of \hypre{} will quickly notice that there is
no class or interface called 
"preconditioner". Instead, we have chosen to interpret preconditioners as
"solvers", based on the fact that 
both traditional solvers and preconditioners are in some senses "approximations
to the inverse" of the linear 
operator defining the system to be solved. The difference between them is one
of degrees more than a fundamental distinction, in that 
preconditioners generally are very crude and very easy to compute
approximations to the inverse, while 
solvers are more accurate, i.e. within the desired convergence tolerance. By
treating solvers and 
preconditioners uniformly within \hypre{}, we make their use uniform for users and
reduce the complexity 
of \hypre{}'s object model. We also enable novel algorithmic possibilities such as
using what are 
traditionally "solvers" as preconditioners. In fact, since Operators include
solvers and matrices, it is 
straightforward to use matrices as preconditioners, should a direct
approximation to the inverse exist. 

\section{Setup of Solvers}

Most components that correspond to the traditional "matrix", as well as
some solvers in which the 
construction of the underlying matrix is handled by the solver for the user,
are set up through one or more 
of the conceptual interfaces discussed in a later chapter. Here we will
concentrate on other interfaces for
constructing Operators.

\subsection{The SolverBuilder interface}

The main function in the SolverBuilder interface


Various Krylov solvers are included in \hypre{}.
Such methods are very flexible in that they can be used to attempt
the solution of any system defined by any component that supports the
Operator interface.
