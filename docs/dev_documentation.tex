%==========================================================================
\chapter{Documentation Standards}
\label{Documentation Standards}

All documentation is available on the Scalable Linear Solvers Members Only
webpage in HTML format.  You can also get postscript version from the
repository (in the docs directory).  Both of these are updated daily since
autotest builds the docs directory every night. 
\newline
To add/update documentation do the following:
\newline
To add to the Users Manual or Developers Manual create a Latex file and prepend
\begin{verbatim} usr_ \end{verbatim}
to the filename.  In the file use the chapter and label conventions following
in other
\begin{verbatim} usr_*.tex \end{verbatim} 
files.  Then add this file to 
\begin{verbatim} usr_manual.tex \end{verbatim} 
in the ``Include Chapters'' section.  Both this change and the new latex file
must be committed.
\newline\newline To add to the Code Reference Manual DOC++ must be used.  The
first step is to add DOC++ style comments to your source code in source.c.  The
CASC software development web page has information on how to do this.  When the
docs directory is built, these comments will be put into a file called
source.dxx.  Next, add a file to the docs directory (or use an existing one if
applicable-- currently the sections are divided into ``Interface Reference''
and ``Implementation Reference'') called
\begin{verbatim} filename_ref.dxx. \end{verbatim} 
Examine additional targets in the Makefile if you want to build only certain
parts of the documentation.  If creating a new one, follow the conventions used
in
\begin{verbatim} interface_ref.dxx \end{verbatim} and be sure to include it
in
\begin{verbatim} code_ref.dxx. \end{verbatim}  
In the new (or existing) .dxx file include source.c for all newly documented
source files, including full path name.
\newline\newline To build documentation go into the docs directory and type
``make''.  Examine additional targets in the Makefile if you want to build only
certain parts of the documentation.  If you are adding or changing any
documentation please check to make sure it builds correctly prior to committing
it to the repository-- if the build breaks then the online documentation will
not be available to anyone on the following day. You will find the html files
in automatically created directories called
\begin{verbatim} docs/HYPRE_dev_manual and docs/HYPRE_usr_manual and docs/HYPRE_code_ref.\end{verbatim}

Example documentation for C routines can be found in the directory
\file{struct_matrix_vector} in the files:
\begin{verbatim}
   communication.c
   communication_info.c
   computation.c
\end{verbatim}
