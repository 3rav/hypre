\documentstyle[12pt]{article}

\textwidth 16cm
\textheight 21cm
\topmargin -0.5cm
\oddsidemargin 0.5cm
\evensidemargin 0.5cm
\setlength{\headsep}{0.5cm} 
\catcode`\_=\active \global\let_=\_
\catcode`\$=\active \global\let$=\$


\begin{document}
\title{hypre Developer's Manual}
\newpage
\tableofcontents

\date{October 09, 1998}
\newpage

\section{Purpose of this document}

\pagenumbering{arabic}
This Developer's Manual is intended to serve as a repository for any and all
information regarding the development of hypre.  It is a useful resource for
the developers of hypre to document their own work as well as become 
acquainted with other aspects of the project, including project guidelines and
standards.

\section{Requirements Analysis for hypre}

\subsection{The Functional Requirements of hypre}
R1.	Hypre will consist of a set of linear solvers, preconditioners,
	libraries of linear solvers, and an interface that allows various 
	applications to use these libraries and linear solvers in various ways.

	R1.1: Current linear solver include ILUT, AMG, EBE, PILUT, SMG, and
	

	R1.2: Current libraries of linear solvers include PetsC, ISIS, and
	


R2.	Hypre must allow access to other libraries of linear solvers
	directly through the hypre interface and allow hypre's other linear
	solvers and preconditioners to be accesses via the standard interface 
	for the other libraries of linear solvers.


R3.	New linear solvers, preconditioners, and libraries of linear solvers 
	should be able to be added to hypre and accessed through the hypre
	interface any time without much modification.

	R3.1: A degree of modularity and loose coupling should exist to 
	ensure modification and enhancements are simple.

	R3.2 (related to R2 also): every linear solver and preconditioner
	should exist independently of the hypre interface.

	
R4.	The applications that will use hypre may provide data in a variety
	of forms.  Hyper should allow new forms of data to be input
	without much modification.
	
	R4.1: Currently application data may consist of linear algebra, 
	finite elements, stencils, and

	R4.2: The hypre interface should be modular to the extent that adding
	new valid application data forms will be simple.


R5.	Hypre should provide mechanisms to maximize the number of linear
	solvers and preconditioners avialable to an application regardless of 
	the format of that data provided by those applications.  

	R5.1: The user will choose which linear solver or preconditioner
	to apply to the data. (A set of valid choices will be provided).
	
	R5.1: This should be accomplished primarily by "translating" the 
	application data into the storage schema required by the solver or
	preconditioner or by applting the coefficient access method 
	required by the solver or preconditioner.

\subsection{Non-Functional Requirements of hypre}

R6.	Hypre should be portable across these platforms: Sun, DEC, BLue, and
	
R7.	Hypre should work with applications written in C, C++, FORTRAN77, and
	
R8.	Hypre should should be easy to compile, configure, and install.

R9.	Hypre should be scalable.

R10.	Hypre should be work with applications that utilize the MPI standard
	as well as those incorporating both MPI and OpenMP or MPI and Pthreads.

\section{The Design of hypre}

\subsection{Software Architecture}
block diagram will be shown here

\subsection{Object Model}
plenty more diagram and class descriptions and discussion of ESI standards.

\section{The Implementation of hypre}

\subsection{Error Handling}

\subsection{Makefile Standards}

   		 Using Autoconfig in hypre
   		 -------------------------

NOTE: These procedures currently only apply to the following directories:

  utilities
  struct_matrix_vector
  struct_linear_solvers

To automatically generate machine specific makefiles, type `configure'
in the `linear_solvers' directory.  The configure script is a portable
script generated by GNU Autoconf.  It runs a series of tests to
determine characteristics of the machine on which it is running, and
it uses the results of the these tests to produce the machine specific
makefiles, called `Makefile', from template files called `Makefile.in'
in each directory.  Once the makefiles are produced you can run make
as you would with any other makefile.

The configure script produces a file called `config.cache' which
stores some of its results.  If you wish to run configure again in a
way that will get different results, you should remove `config.cache'.

This configure script primarily does the following things:

  - selects a C compiler
  - provides either optimization or debugging options for the C compiler
  - finds the headers and libraries for MPI

The configure script has some command-line options that can give you
some control over the choices it will make.  You can type

  configure --help

to see the list of all of the command-line options to configure, but
the most significant options are at the bottom of the list, after the
line that reads "--enable and --with options recognized:"

--with-CC=ARG		
			This option allows you to choose the C compiler
			you wish to use.  The default compiler that
			configure chooses is gcc, if it is available.

--enable-opt-debug=ARG  
			Choose whether you want the C compiler to have
			optimization or debugging flags.  For debugging,
			replace `ARG' with `debug'.  For optimization,
			replace `ARG' with `opt'.  If you want both sets
			of flags, replace `ARG' with `both'.  The default
			is optimization

--without-MPI		
			This flag suppresses the use of MPI.  It produces
			makefiles that serve the same purpose as the
			files `Makefile.seq'

--with-mpi-include=DIR
--with-mpi-libs=LIBS
--with-mpi-lib-dirs=DIRS
			These three flags are to be used if you want to
			override the automatic search for MPI.  If you use
			one of these flags, you must use all three.
			Replace `DIR' with the path of the directory that
			contains mpi.h, replace `LIBS' with a list of the
			stub names of all the libraries needed for MPI, and
			replace `DIRS' with a list of the directory paths
			containing the libraries specified by `LIBS'
			NOTE:  The lists `LIBS' and `DIRS' should be
			space-separated and contained in quotes, e.g.
			--with-mpi-libs="nsl socket mpi" and
			--with-mpi-lib-dirs="/usr/lib /usr/local/mpi/lib"

--with-mpi-flags=FLAGS	
			Sometimes other compiler flags are needed for
			certain MPI implementations to work.  Replace
			`FLAGS' with a space-separated list of whatever
			flags are necessary.  This option does not
			override the automatic search for MPI.  It can be
			used to add to the results of the automatic search,
			or it can be used along with the three previous
			flags.

--with-MPICC=ARG	
			The automatic search for the MPI stuff is based on
			a tool such as `mpicc' that configure finds in
			your $PATH.  If there is an implementation of MPI
			that you wish to use, you can replace `ARG' with
			the name of that MPI version's C compiler wrapper,
			if it has one.  (MPICH has mpicc, IBM MPI has
			mpcc, other MPIs use other names.)  configure will
			then automatically find the necessary libraries
			and headers.



\subsection{Documentation Standards}

DOCUMENTATION

  DOC++ is to be used for documentation.

  Example documentation for C routines can be found in the directory
  `struct_matrix_vector' in the files:

     DevManual.dxx
     communication.c

  Initial user's and developer's manuals need to be constructed in
  DOC++ to get things started.

  The above `.dxx' file could serve as a beginning developer's manual.

--------------------------------------------------------------------------



\subsection{Coding Standards}
These guidelines are based on those provided by the ISE project in CASC.
See `file:/home/casc/software_development/html/index.html' for more info.

--------------------------------------------------------------------------

GENERAL GUIDELINES

The guidelines given in the ISE project document `coding_style.ps'
should be followed.  There are a few additions/changes that should
be made here:

  1. The "order of parameters" section should be modified so that
  opaque object names always appear first in the argument list.
  For example, a routine for setting the coefficients of a matrix
  object should look like

     HYPRE_SetMatrixCoeffs(matrix_object, ...)

  even though the matrix_object is an I/O parameter.  This is
  just to provide some consistency.

  2. Guidelines for indentation are given below.

  3. Guidelines for documentation are given below.

--------------------------------------------------------------------------

NAMING CONVENTIONS FOR HYPRE


 1. Functions and macros with parameters should be named as follows:

      HYPRE_MixedCase(...)          /* User interface routines */
      hypre_MixedCase(...)          /* Internal non-user routines */

 2. Macros without parameters should be named as follows:

      HYPRE_ALL_CAPS          /* User interface macros */
      hypre_ALL_CAPS          /* Internal non-user macros */

 3. Types should be named as follows:

      HYPRE_MixedCase          /* User interface types */
      hypre_MixedCase          /* Internal non-user types */

 4. It is preferred that variables be named as follows:

      lower_case_with_underscores

    This is not a strict rule, but in general one should try
    to adhere to this so that it is easy to distinguish variables
    from functions and macros.
 
--------------------------------------------------------------------------

User interface vs. internal code

The term "user" here refers to someone who wants to solve linear systems,
but is not developing linear solvers.  The term "internal code" refers to
code that is not part of the user interface.

NOTE: It is important to clearly distinguish between supported
routines and everything else in the library.  The mechanism described
below is one way of making this explicit.

The following convention has been used to some degree already in hypre
for separating the two above notions.  Files containing internal
routines are given names *without* the `hypre_' prefix and include
only *one* internal header file with some name that is unique in the
hypre library (Note: The SMG code currently uses the convention of
giving this header file the same name as the user header file, but
with a lower-case `hypre_' prefix.  See the below example).  All user
interface routines are defined in files beginning with the prefix
`HYPRE_'.  One user interface header file is defined that contains
prototypes, etc., for the interface routines.

By doing this, it is easy to create prototypes for internal routines and
user interface routines.  For example,

  mkproto *.c >> hypre_lib.h
  mkproto HYPRE_*.c >> HYPRE_lib.h

We should continue this convention for now.  It is possible that this
may be changed to some other convention in the future, but if we have
a common way of doing it now, it will be easier to change later.

Note that NT does not distinguish between lower and upper case file
names, so we will need to change the naming convention slightly.
Suggestions?

--------------------------------------------------------------------------

INDENTATION

  The ISE project file `sample.emacs' should be used to provide a
  common indentation style for the library.  If you don't use emacs
  as your editor, you can use the scripts written by Rob Falgout
  to indent files from the shell command line.  See the README file
  in `~falgout/tools'.

  The files `sample_c_code.c' and `sample_f77_code.c' have been
  provided as examples of indentation style for Hypre.  They are
  modifications of those provided by the ISE project.

  NOTE: the fortran source file and the fortran mode defined in
  the `sample.emacs' file do not agree.  The fortran part of this
  document needs work.  Volunteers?

--------------------------------------------------------------------------

\subsection{Language Interoperability}

       Zero-based vs. One-based Indexing in HYPRE Matrices


Some definitions:

0-based indexing.  Row and column indices are numbered starting at 0.
1-based indexing.  Row and column indices are numbered starting at 1.
0-based storage.   Natural for C, possible in Fortran.
1-based storage.   Natural for Fortran, sometimes used in C.


I. The issues and a suggested decision

The issues are as follows:

 1. We want the code to be natural to write.
 2. We want to interoperate with other codes as much as possible.
 3. We want the code to be as efficient as possible.

Section III below contains 4 different ways to write 1 sample code in C.
The sample code does a simple matvec, allocates an int array that will
mark the coarse grid variables in a multigrid scheme, directly calls
a fortran subroutine to fill in the coarse grid array (we might
typically have a C wrapper around the fortran subroutine, but the
intention here is to explicitely demonstrate the parameter translation
required to interface with the routine), then de-allocates the coarse
grid array.

The A and B examples use 0-based and 1-based indexing with no pointer
tricks.  Example C uses 1-based indexing with a pointer trick that is
left fully exposed for illustrative purposes.  Example D supports both 0
and 1-based indexing and uses a hypre_BaseArray "class" (see section II
below for details on this class) to hide the tricks illustrated in
example C.  In these examples, it is assumed that the fortran subroutine
provides the same indexing support as the calling C-routines.

Issue 1:  Example A is the most straightforward and least confusing
of all the examples.  That is, 0-based indexing is the most natural for
C programming.  However, examples C and D are not all that bad either.
Example B is horribly confusing.  The most natural style for fortran is
1-based, but a fortran variant of examples C and D for 0-based indexing
would also be fairly natural.  Arrays would be declared like this:

     integer  Ai(base:An+base)

and the iiA loop would look like this:

     do iiA = base, An-1+base

Issue 2:  We cannot dictate what other code groups support regarding
indexing.  So, if we want to interoperate with other codes, we need to
support both 0-based and 1-based indexing as in example D below.

Issue 3:  To keep efficiency high, we do *not* want to have to modify
the data in the A_i and A_j arrays.  That is, we want to use the pointer
tricks below to interoperate with other codes.  However, this means that
the whole code does *either* 0-based *or* 1-based indexing during any single
run, and *not* both.

Suggested decision: Support both 0-based and 1-based indexing.


II. Hiding the details

Although code for supporting both 0-based and 1-based indexing is not
too ugly even when the tricks are exposed, it is easier to make mistakes
and/or get confused during code development and debugging.  In an attempt
to minimize this, we introduce a hypre_BaseArray "class" in subsection II.1,
and a set of coding rules and guidelines in subsection II.2.

II.1. The BaseArray class

Types:
  hypre_IntBaseArray, hypre_DoubleBaseArray, ...

  Note: These types are just typedefs for `int' and `double', so
  compilers will not treat them at distinct types (strong typing).
  It would be nice to have compilers do strong typing on them
  to help reduce mistakes, but it is apparantly not possible to
  do this in C *and* retain the ability to use the `[]' operators
  and do casting.

Constructor/destructor macros:
  base_array = hypre_AllocBaseArray(type, size, base)
  hypre_FreeBaseArray(base_array, base)

Macro that returns a pointer to the real data:
  base_array_data = hypre_BaseArrayData(base_array, base)

Macro that converts from 1 base index to another
  new_index = hypre_ConvertBaseIndex(index, base, new_base)

Function that converts from 1 base to another (This function actually
modifies the data in the array, so it should be used only when necessary):
  hypre_ConvertBaseArray(&base_array, base, new_base)

II.2. Coding Rules and Guidelines

- All arrays with array indices that refer to either rows/columns of
  matrices or elements of vectors must be hypre_BaseArray types.
  This reduces the possibility of making coding errors.

- All hypre_BaseArray types in the same routine should use the same base.
  It is recommended that only one `base' variable be defined for each
  routine in an attempt to enforce this.

- Care must be taken when calling other codes that depend on the matrix
  and vector base to convert (if necessary) to a base that the called
  code supports.


III. Sample code

/*--------------------------------------------------------------------------
 * Indexing example code:
 *--------------------------------------------------------------------------*/

/*-----------------------------------------------
 * A) 0-based indexing
 *-----------------------------------------------*/

int
hypre_SampleA( hypre_Matrix *A,
               hypre_Vector *x,
               hypre_Vector *y )
{
   double  *A_data = hypre_MatrixData(A);
   int     *A_i    = hypre_MatrixI(A);
   int     *A_j    = hypre_MatrixJ(A);
   int     *A_n    = hypre_MatrixN(A);

   double  *x_data = hypre_VectorData(x);
   double  *y_data = hypre_VectorData(y);

   int      iiA, jjA;

   int     *coarse_grid;

   /* ... */

   /* matvec */
   for (iiA = 0; iiA < A_n; iiA++ )
   {
      for (jjA = A_i[iiA]; jjA < A_i[iiA+1]; jjA++)
      {
         y_data[iiA] += A_data[jjA] * x_data[A_j[jjA]];
      }
   }

   /* coarse-grid allocation */
   coarse_grid = hypre_TAlloc(int, A_n);

   /* coarse-grid selection function call */
   hypre_coarsen_( ..., coarse_grid);

   /* coarse-grid de-allocation */
   hypre_TFree(coarse_grid);

   return 0;
}

/*-----------------------------------------------
 * B) 1-based indexing (no tricks)
 *-----------------------------------------------*/

int
hypre_SampleB( hypre_Matrix *A,
               hypre_Vector *x,
               hypre_Vector *y )
{
   double  *A_data = hypre_MatrixData(A);
   int     *A_i    = hypre_MatrixI(A);
   int     *A_j    = hypre_MatrixJ(A);
   int     *A_n    = hypre_MatrixN(A);

   double  *x_data = hypre_VectorData(x);
   double  *y_data = hypre_VectorData(y);

   int      iiA, jjA;

   int     *coarse_grid;

   /* ... */

   /* matvec */
   for (iiA = 1; iiA <= A_n; iiA++ )
   {
      for (jjA = A_i[iiA-1]; jjA < A_i[iiA]; jjA++)
      {
         y_data[iiA-1] += A_data[jjA-1] * x_data[A_j[jjA-1]-1];
      }
   }

   /* coarse-grid allocation */
   coarse_grid = hypre_TAlloc(int, A_n);

   /* coarse-grid selection function call */
   hypre_coarsen_( ..., coarse_grid);

   /* coarse-grid de-allocation */
   hypre_TFree(coarse_grid);

   return 0;
}

/*-----------------------------------------------
 * C) 1-based indexing (with tricks):
 *-----------------------------------------------*/

int
hypre_SampleC( hypre_Matrix *A,
               hypre_Vector *x,
               hypre_Vector *y )
{
   double  *A_data = hypre_MatrixData(A) - 1;
   int     *A_i    = hypre_MatrixI(A) - 1;
   int     *A_j    = hypre_MatrixJ(A) - 1;
   int     *A_n    = hypre_MatrixN(A);

   double  *x_data = hypre_VectorData(x) - 1;
   double  *y_data = hypre_VectorData(y) - 1;

   int      iiA, jjA;

   int     *coarse_grid;

   /* ... */

   /* matvec */
   for (iiA = 1; iiA < A_n+1; iiA++ )
   {
      for (jjA = A_i[iiA]; jjA < A_i[iiA+1]; jjA++)
      {
         y_data[iiA] += A_data[jjA] * x_data[A_j[jjA]];
      }
   }

   /* coarse-grid allocation */
   coarse_grid = hypre_TAlloc(int, A_n) - 1;

   /* coarse-grid selection function call */
   hypre_coarsen_( ..., coarse_grid + 1);

   /* coarse-grid de-allocation */
   hypre_TFree(coarse_grid + 1);

   return 0;
}

/*-----------------------------------------------
 * D) 0 & 1-based indexing:
 *
 * We use the same tricks as in example C, except
 * that we "hide" most of them in the following
 * macros and functions:
 *
 * hypre_MatrixData, hypre_MatrixI, hypre_MatrixJ
 * hypre_VectorData
 * hypre_AllocBaseArray
 * hypre_FreeBaseArray
 * hypre_BaseArrayData
 *
 * The integer `base' is either a 0 or a 1 depending
 * on whether the storage is 0 or 1-based.
 *
 * We also use the following types so that the
 * code is more explicit about which variables
 * we are applying "tricks" to.
 *-----------------------------------------------*/

int
hypre_SampleD( hypre_Matrix *A,
               hypre_Vector *x,
               hypre_Vector *y )
{
   hypre_DoubleBaseArray  *A_data = hypre_MatrixData(A);
   hypre_IntBaseArray     *A_i    = hypre_MatrixI(A);
   hypre_IntBaseArray     *A_j    = hypre_MatrixJ(A);
   hypre_IntBaseArray     *A_n    = hypre_MatrixN(A);

   hypre_DoubleBaseArray  *x_data = hypre_VectorData(x);
   hypre_DoubleBaseArray  *y_data = hypre_VectorData(y);

   hypre_IntBaseArray     *coarse_grid;

   int                     iiA, jjA;

   /* ... */

   /* matvec */
   for (iiA = base; iiA < A_n+base; iiA++ )
   {
      for (jjA = A_i[iiA]; jjA < A_i[iiA+1]; jjA++)
      {
         y_data[iiA] += A_data[jjA] * x_data[A_j[jjA]];
      }
   }

   /* coarse-grid allocation ( = hypre_TAlloc(int, A_n) - base ) */
   coarse_grid = hypre_AllocBaseArray(int, A_n, base);

   /* coarse-grid selection function call */
   hypre_coarsen_( ..., &base,
                   hypre_BaseArrayData(coarse_grid, base);

   /* coarse-grid de-allocation ( = hypre_TFree(coarse_grid + base) ) */
   hypre_FreeBaseArray(coarse_grid, base);
   return 0;
}

\section{Testing and Releasing hypre}

\subsection{Autotest Procedures}
What is autotest?
\newline
----------------------------------------------------------------------------
\newline
Autotest is a set of driver programs and scripts set up as a cron job to test daily code changes.

How is autotest structured?
\newline
----------------------------------------------------------------------------

\newline
The /linear_solvers/test/ subdirectory (in the repository) contain
1.) various driver programs (.c files)
2.) bourne shell scripts to run the driver programs
3.) test_drivers.sh script calls all other scripts and writes their results to a file autotest.log.
4.) the autotest script-- this copy is not used by the cron job however.

The /home/casc/software/hypre/autotest directory contains
1.) the autotest script which is used by the cron job.
2.) the linear_solvers directory which autotest checks out of the repository
3.) Autotest script checks out the repository, configures, makes the library,
calls the test_drivers.sh script in the newly checked out directory structure, 
and sets permissions.  This file should seldom change.

Features of autotest
\newline
----------------------------------------------------------------------------
\newline
1.) works with the autoconfig infrastructure.
2.) test suites can be added/removed easily.
3.) identify failure/success of a test (future feature)
4.) email code developers of test failure (future feature)
5.) test on all target platforms (future feature)

How to modify autotest
\newline
----------------------------------------------------------------------------
\newline
1.) create and test a driver program -- somedriver.c
2.) create and test a Bourne shell script to run the driver -- somedriver.sh
3.) append the following to the test_drivers.sh file
HYPRE_SOMEDRIVER_LOG_FILE=''somedriver.log''
./somedriver.sh >> $HYPRE_SOMEDRIVER_LOG_FILE
4.) modify the Makefile.in file in the test subdirectory as follows:
add the necessary information to LFLAGS = \ section
add the necessary target information to the TARGETS section
add the necessary rules to the RULES section
5.) modify the /linear_solvers/ Makefile.in file as follows:
add the necessary information to the HYPRE_DIRS =\ section
6.) modify the /linear_solvers/ configure.in file as follows
add information to the AC_OUTPUT section (first one)
add information to the CASC_CONFIG_OUTPUT_LIST section
add information to the final AC_OUTPUT section
7.) To test the changes 
in linear_solvers directory type 
>autoconf
>configure
>make
>cd test
>test_driver.sh
>cd /home/casc/software/hypre/autotest/linear_solvers/test/
look at somedriver.log for correct output.
8.)if everything went well commit all added and modified files.  There is no need to modify autotest or the crontab file.



\subsection{Quality Assurance Issues}

\subsection{Installation Procedures}

    Instructions for updating and maintaining installations of hypre

----------------------------------------------------------------------------

GENERAL COMMENTS:

There are two types of hypre installations that are updated and
maintained by members of the Scalable Linear Solvers project:

  1. The "internal" installation:

    - intended to give users access to relatively recent changes.
    - updated somewhat frequently (about once a week).
    - may be relatively unstable.
    - one old installation is retained.
    - installed on:
        CASC Sun workstation cluster

  2. The "external" installation:

    - intended to give users consistent access across several platforms.
    - updated less frequently than internal installation.
    - should be relatively stable.
    - a few old installations are retained.
    - installed on:
        CASC Sun workstation cluster
        Compass DEC cluster
        Blue-Pacific
        Forest DEC cluster  (classified)
        Blue(Sky)-Pacific   (classified)

Rob Falgout is maintaining two lists of users who want to receive
notification of installation updates: one for "internal" updates and
one for "external" updates.  Users should send email to Rob at
rfalgout@llnl.gov to get on either list.

----------------------------------------------------------------------------

INSTALLATION PROCEDURES:

All installations are done with the `update' Bourne-shell script
located in the top-level directory of the hypre repository.  To
get general usage info, do `update -help'.

Internal installation:

  (on CASC cluster only)

  1. Do `update internal' - this checks out the current version of the
     repository, tags it with a date (file `VERSION_DATE'), compiles it,
     temporarily installs it, and sets the permissions of the temporary
     installation to appropriate values.  This also creates the tar file
     `UPDATE.tar', currently only used in the external installation.

  2. Do `update internal install' - this saves the previously
     installed version, moves the temporary installation into its
     correct location, and possibly deletes some older installations.

     Note that step 1 is not actually needed, but is recommended in case
     something doesn't work as planned (e.g., failed checkout, failed
     compile, ...).  Also, step 1 provides the opportunity to test the
     library in non-CASC codes such as ARES or ALE3D via the temporary
     installation in the `UPDATE' subdirectory.

  3. Send email to appropriate users indicating the update.

External installation:

  (on CASC cluster)

  1. Do `update external'
  2. Do `update external install'

  (on all other supported machines, e.g. the Compass cluster)

  1. FTP the file `UPDATE.tar' to west (for example), and put it
     in the install directory (do `update -dirs' for a list).
  2. Do `tar xf UPDATE.tar'
  3. Do `update external -dec '
  4. Do `update external -dec install'

----------------------------------------------------------------------------


\section{Code Reference}
All the DOC++ stuff goes here
   
\end{document}
