%==========================================================================
\chapter{Creating and Installing a HYPRE Distribution}
\label{Creating and Installing a HYPRE Distribution}

Making a HYPRE release available to the user community involves creating a
distribution tar file, copying it to supported CASC and Livermore Computing
(LC) platforms and installing the libraries on those platforms.  As LC adds 
new machines, the latest HYPRE releases will be installed on those machines.

General and beta releases are installed on all supported LC platforms. While
alpha releases are installed {\bf only} on the CASC cluster.  Each release is 
configured and built with different options which results in the creation of 
subdirectories as defined below:
\begin{itemize}
\item optimized -- stored in lib and include
\item optimized and debug -- stored in debug/lib and debug/include

\item serial -- stored in serial/lib and serial/include
\item serial and debug -- stored in serial/debug/lib and serial/debug/include

\item threads -- stored in threads/lib and threads/include
\item threads and debug -- stored in threads/debug/lib and threads/debug/include
\end{itemize}

%==========================================================================
\section{Scripts}
\label{Scripts}

There are three, currently Korn shell, scripts \file{mkdist}, \file{mkdistlinks}
and \file{mklibs} that ease the task of building and installing \hypre{} on most 
platforms.  These scripts reside in the \file{tools} subdirectory of the \hypre{} 
repository or in the root directory on specific platforms.

%==========================================================================
\subsection{mkdist}
\label{mkdist}

The distribution tar file is created with the \file{mkdist} script.

\begin{verbatim}
Usage: ./mkdist [-help] | [-display] | VersionID
       where:
             -help       displays help message only
             -display    reports the CVS rtag history
             VersionID   release number
                         syntax: M.mm.rr  -- general release
                                 M.mm.rrb -- beta release
                                 M.mm.rra -- alpha release
\end{verbatim}

\file{mkdist} extracts the requested version from the repository, reorganizes 
the file structure, builds the documentation and creates the distribution tar 
file named \file{hypre-M.nn.rr.tar.gz}.

%==========================================================================
\subsection{mkdistlinks}
\label{mkdistlinks}

\file{mkdistlinks} creates symbolic links from the \hypre{} distribution tar file
to subdirectories that are defined within this script.  The directory list only
needs to be updated when new classes of platforms are added.

\begin{verbatim}
Usage: ./mkdistlinks hypre-M.mm.rr.tar.gz
       where:
             hypre-M.mm.rr.tar.gz  distribution tar file created by mkdist
\end{verbatim}

%==========================================================================
\subsection{mklibs}
\label{mklibs}

\file{mklibs} generates the various versions of the library (optimized, debug, serial,
threaded) and stores them in the appropriate subdirectories. Optionally, it will 
symbolically link to the \file{/usr/apps/hypre} directory.

\begin{verbatim}
Usage: ./mklibs [-help] | [-n] <-g | -b | -a> hypre-M.mm.rr.tar.gz [install]
       where:
             -help       displays help message only
             -n          do not extract files from hypre-M.mm.rr.tar.gz
             <-g|-b|-a>  release type -- MUST choose one
                         -g for general release
                         -b for beta release
                         -a for alpha release
             hypre-M.mm.rr.tar.gz  distribution tar file created by mkdist
             install     create symbolic links
\end{verbatim}

\file{mklibs} untars the distribution, resulting in a new directory named 
\file{hypre-M.mm.rr} which contains the \file{docs} and \file{src} subdirectories 
as well as the CHANGELOG and other relevant files.

There are two specialized versions of \file{mklibs}:
\begin{itemize}
\item \file{mklibs.aix}   creates 32 and 64 bit libraries on AIX pltforms
\item \file{mklibs.babel} adds --with-babel to each of the installations
\end{itemize}
   
%==========================================================================
\section{Creating a Distribution Tar File}
\label{Creating a Distribution Tar File}

The distribution tar file is created on a local CASC platform.  The following 
steps detail the actions to be taken as well as provide an example command as 
appropriate.  Prior to executing these steps the release number will be assigned 
by the project team. In the examples below, M-mm-rr is a place-holder for the 
actual release number, which may or may not have an 'a' or 'b' appended; depending 
on the type of release being created.

\begin{enumerate}
\item Go to a local directory
\begin{verbatim}
      Example: cd /home/hill66
\end{verbatim}
\item Retrieve the latest version of \hypre{} from the CVS repository
      results in a new directory named linear\_solvers which contains the source code
\begin{verbatim}
      Example: cvs checkout linear_solvers
\end{verbatim}
\item Move to the linear\_solvers directory
\begin{verbatim}
      Example: cd linear_solvers
\end{verbatim}
\item Move to the config directory
\begin{verbatim}
      Example: cd config
\end{verbatim}
\item Change the HYPRE\_version number, date and time in the \file{configure.in} file
\begin{verbatim}
      Example: vi configure.in
               change numeric values following HYPRE_* in:
                  m4_define(HYPRE_VERSION, 1.9.0b)
                  m4_define(HYPRE_DATE, 2005/02/24)
                  m4_define(HYPRE_TIME, 14:26:53)
\end{verbatim}
\item Move to the linear\_solvers directory
\begin{verbatim}
      Example: cd linear_solvers
\end{verbatim}
\item Run the \file{bootstrap} script to create a new \file{configure} script
\begin{verbatim}
      Example:./config/bootstrap
\end{verbatim}
\item Commit new \file{configure} and \file{configure.in} files to repository
\begin{verbatim}
      Example: cvs commit 
\end{verbatim}
\item Tag the repository with this release number
\begin{verbatim}
      Example: cvs rtag VM-mm-rr linear_solvers
\end{verbatim}
\item Create the distribution tar file by running the \file{mkdist} script,
      located in the tools directory. Output is a new directory named \file{hypre-M.mm.rr}
      and a compressed tar file named \file{hypre-M.mm.rr.tar.gz}
\begin{verbatim}
      Example: ./tools/mkdist VM-mm-rr
\end{verbatim}
\item Copy the tar file to /usr/casc/hypre on the local system
\begin{verbatim}
      Example: cp hypre-M.mm.rr.tar.gz /usr/casc/hypre
\end{verbatim}
\end{enumerate}

%==========================================================================
\section{Installing a Distribution on the CASC Cluster}
\label{Installing a Distribution on the CASC Cluster}

On the CASC cluster the root directory for \hypre{} is \file{/usr/casc/hypre}.  The 
subdirectory is \file{i686-pc-linux-gnu}. The following procedure will build the 
library with its options and stored in the appropriate directories.

\begin{enumerate}
\item Move to the /usr/casc/hypre directory
\begin{verbatim}
      Example: cd /usr/casc/hypre
\end{verbatim}
\item Create symbolic links to the tar file by running the \file{mkdistlinks} script
\begin{verbatim}
      Example: ./mkdistlinks hypre-M.mm.rr.tar.gz
\end{verbatim}
\item Move to the CASC cluster subdirectory; i686-pc-linux-gnu
\begin{verbatim}
      Example: cd i686-pc-linux-gnu
\end{verbatim}
\item Build and install the library by runnning the \file{mklibs} script; which is
         located in the \file{/usr/casc/hypre} directory.
      The arguments -g (general), -b (beta) or -a (alpha) indicate the type of 
          release being generated.  If the install argument is included the 
          symbolic links will be created automatically.
\begin{verbatim}
      Example: ../mklibs <-g | -b | -a> hypre-M.mm.rr.tar.gz <install>
\end{verbatim}
\end{enumerate}

%==========================================================================
\section{Installing a Distribution on LC Platforms}
\label{Installing a Distribution on LC Platforms}

On all of the supported LC platforms, the root directory for \hypre{} is 
\file{/usr/gapps/hypre}.  Under the root are subdirectories for each class 
of machines identified by the LC defined environment variable, SYS\_TYPE.
The following table shows the SYS\_TYPE values for supported LC open (OCF) 
and secure (SCF) platforms as of April 2005.

\begin{verbatim}
  SYS_TYPE                       OCF                    SCF
aix_5_64                         ---                  tempest
aix_5_ll                        frost               white, ice
aix_5_64_fed                   newberg                 uv, um
tru64_5                          gps                    sc
sles_9_ppc64                     bgl                    ---
chaos_2_ia32                     ilx                    ace
chaos_2_ia32_elan3        pengra, alc, mcr,       adelie, emperor,
                             pvc, bigdev            lilac, gviz
chaos_2_ia32_elan4             thunder                  cub
\end{verbatim}

To relieve the user of needing to know the SYS\_TYPE on each platform, the path
\file{/usr/apps/hypre} will be symbolically linked to the appropriate subdirectory
on each platform by the development team.  The command to create the link is:
\begin{verbatim}
ln -s /usr/gapps/hypre/<SYS_TYPE> /usr/apps/hypre
  where the actual value of SYS_TYPE is substituted for <SYS_TYPE>.
\end{verbatim}

%==========================================================================
\subsection{Open Computing Facility (OCF) Platforms}
\label{Open Computing Facility (OCF) Platforms}

Before the \hypre{} library can be built on an OCF platform it needs to be 
copied from the CASC cluster.  This can be done from any CASC cluster directory
in which \file{hypre-M.mm.rr.tar.gz} exists by using the command (host is any OCF
platform):
\begin{verbatim}
  scp hypre-M.mm.rr.tar.gz host:/usr/gapps/hypre
\end{verbatim}

When the distribution tar file has been successfully copied, create the symbolic
links to the tar file. This only needs to be done once for all platforms.

\begin{enumerate}
\item Move to the \file{/usr/gapps/hypre} directory
\begin{verbatim}
      Example: cd /usr/gapps/hypre
\end{verbatim}
\item Create symbolic links to the tar file by running the \file{mkdistlinks} script
\begin{verbatim}
      Example: ./mkdistlinks hypre-M.mm.rr.tar.gz
\end{verbatim}
\end{enumerate}

The procedure below MUST be repeated for each class of platforms on which \hypre{} is 
to be installed. 

\begin{enumerate}
\item If not already in the \file{/usr/gapps/hypre} directory; move to it
\begin{verbatim}
      Example: cd /usr/gapps/hypre
\end{verbatim}
\item Move to the subdirectory for this platform, identified by SYS\_TYPE
\begin{verbatim}
      Example: cd chaos_2_ia32_elan3
\end{verbatim}
\item Build and install the library by runnning the \file{mklibs} script; which is
         located in the \file{/usr/gapps/hypre} directory.
      The arguments -g (general) or -b (beta) indicate the type of release being 
          generated.  If the install argument is included, the symbolic links will 
          be created automatically.
\begin{verbatim}
      Example: ../mklibs <-g | -b> hypre-M.mm.rr.tar.gz <install>
\end{verbatim}
\end{enumerate}

%==========================================================================
\subsection{Secure Computing Facility (SCF) Platforms}
\label{Secure Computing Facility (SCF) Platforms}

Before the \hypre{} library can be built on an SCF platform it needs to be 
copied from an OCF platform.  This is done through the FIS utility by logging
onto an OCF platform, running FTP to connect to FIS, sending the file to the SCF, 
logging out of OCF and logging into an SCF platform after a few hours have elapsed.
From an SCF platform again run FTP to extract the file from FIS and continue as
described below.
\begin{verbatim}
      Example: On OCF: ftp fis
                       cd TO
                       put hypre-M.mm.rr.tar.gz
                       quit
\end{verbatim}

\begin{verbatim}
      Example: On SCF: ftp fis
                       cd FROM
                       get hypre-M.mm.rr.tar.gz
                       quit
\end{verbatim}

When the distribution tar file has been successfully copied, create the symbolic
links to the tar file. This only needs to be done once for all platforms.

\begin{enumerate}
\item Move to the \file{/usr/gapps/hypre} directory
\begin{verbatim}
      Example: cd /usr/gapps/hypre
\end{verbatim}
\item Create symbolic links to the tar file by running the \file{mkdistlinks} script
\begin{verbatim}
      Example: ./mkdistlinks hypre-M.mm.rr.tar.gz
\end{verbatim}
\end{enumerate}

The procedure below MUST be repeated for each class of platforms on which \hypre{} is 
to be installed. 

\begin{enumerate}
\item If not already in the \file{/usr/gapps/hypre} directory; move to it
\begin{verbatim}
      Example: cd /usr/gapps/hypre
\end{verbatim}
\item Move to the subdirectory for this platform, identified by SYS\_TYPE
\begin{verbatim}
      Example: cd chaos_2_ia32_elan3
\end{verbatim}
\item Build and install the library by runnning the \file{mklibs} script; which is
         located in the \file{/usr/gapps/hypre} directory.
      The arguments -g (general) or -b (beta) indicate the type of release being 
          generated.  If the install argument is included, the symbolic links will 
          be created automatically.
\begin{verbatim}
      Example: ../mklibs <-g | -b> hypre-M.mm.rr.tar.gz <install>
\end{verbatim}
\end{enumerate}

%==========================================================================
\section{Example of Creating and Installing a Beta Release}
\label{Example of Creating and Installing a Beta Release}

In the example, the version number, machine names and user names are for illustration 
purposes only; the specific values may differ. Each step is separated by a blank line.
Output from some steps is shown in abbreviated form immediately below the command.

On a local CASC platform:
\begin{verbatim}
 1.  cd /home/hill66

 2.  cvs checkout linear_solvers
     U linear_solvers/CHANGELOG
     U linear_solvers/COPYRIGHT_and_DISCLAIMER
             . . .
     U linear_solvers/utilities/utilities.h
     U linear_solvers/utilities/version

 3.  cd linear_solvers/config

 4.  vi configure.in
         <change values in m4_define(HYPRE_*) statements>
     quit

 5.  ./config/bootstrap

 6.  cvs commit
     Update configure files for release 1.90b

 7.  cvs rtag V1-9-0b linear_solvers

 8.  ./tools/mkdist V1-9-0b
     U linear_solvers/CHANGELOG
     U linear_solvers/COPYRIGHT_and_DISCLAIMER
              . . .
     U linear_solvers/utilities/utilities.h
     U linear_solvers/utilities/version
     checking the hostname... tux149
     checking the architecture... linux
              . . .
     creating hypre-1.9.0b.tar file ...

 9.  cp hypre-1.9.0b.tar.gz /usr/casc/hypre

10.  cd /usr/casc/hypre

11.  ./mkdistlinks hypre-1.9.0b.tar.gz
     lrwxrwxrwx   1 hil66 hypre  22 Apr 13 07:32 alphaev68-dec-osf5.1/hypre-1.9.0b.tar.gz -> ../hypre-1.9.0b.tar.gz
     lrwxrwxrwx   1 hil66 hypre  22 Apr 13 07:32 i686-pc-linux-gnu/hypre-1.9.0b.tar.gz -> ../hypre-1.9.0b.tar.gz
     lrwxrwxrwx   1 hil66 hypre  22 Apr 13 07:32 powerpc-ibm-aix5.2.0.0/hypre-1.9.0b.tar.gz -> ../hypre-1.9.0b.tar.gz
     lrwxrwxrwx   1 hil66 hypre  22 Apr 13 07:32 chaos_2_ia32_elan3/hypre-1.9.0b.tar.gz -> ../hypre-1.9.0b.tar.gz

12.  cd i686-pc-linux-gnu

13.  ../mklibs -b hypre-1.9.0b.tar.gz install
     checking the hostname... blue
     checking the architecture... aix
             . . .
     Cleaning FEI_mv ...
     Cleaning test ...
\end{verbatim}
The build and installation is now complete on the CASC cluster.

Next copy the tar file to an OCF platform (mcr in this example).
\begin{verbatim}

14.  scp hypre-1.9.0b.tar.gz mcr:/usr/gapps/hypre

\end{verbatim}

Now log off the CASC cluster and into an OCF platform (mcr in this example) to build and 
install \hypre{}.
\begin{verbatim}
15.  ssh mcr
16.  cd /usr/gapps/hypre

17.  ./mkdistlinks hypre-1.9.0b.tar.gz
     lrwxrwxrwx   1 hil66 hypre  22 Apr 13 07:32 alphaev68-dec-osf5.1/hypre-1.9.0b.tar.gz -> ../hypre-1.9.0b.tar.gz
     lrwxrwxrwx   1 hil66 hypre  22 Apr 13 07:32 powerpc-ibm-aix5.2.0.0/hypre-1.9.0b.tar.gz -> ../hypre-1.9.0b.tar.gz
     lrwxrwxrwx   1 hil66 hypre  22 Apr 13 07:32 chaos_2_ia32_elan3/hypre-1.9.0b.tar.gz -> ../hypre-1.9.0b.tar.gz

18.  cd chaos_2_ia32_elan3

19.  ../mklibs -b hypre-1.9.0b.tar.gz install
     checking the hostname... blue
     checking the architecture... aix
             . . .
     Cleaning FEI_mv ...
     Cleaning test ...
\end{verbatim}
The build and installation is now complete on one OCF class of machines. For each of the
remaining classes of OCF platforms, log-in, cd to /usr/apps/hypre/<SYS\_TYPE> and repeat
step 19.

Next copy the tar file to an SCF platform (adelie in this example).
\begin{verbatim}
20.  ftp fis
       . . . 
     cd TO
     put hypre-1.9.0b.tar.gz
     quit
\end{verbatim}

Now log off the OCF platform and into an SCF platform (adelie in this example) to retrieve
the tar file.
\begin{verbatim}
21.  ssh adelie

22.  ftp fis
       . . . 
     cd FROM
     get hypre-1.9.0b.tar.gz
     quit
\end{verbatim}

Now build and install \hypre{} on the SCF platform (adelie in this example).
\begin{verbatim}

23.  cd /usr/gapps/hypre

24.  ./mkdistlinks hypre-1.9.0b.tar.gz
     lrwxrwxrwx   1 hil66 hypre  22 Apr 13 07:32 alphaev68-dec-osf5.1/hypre-1.9.0b.tar.gz -> ../hypre-1.9.0b.tar.gz
     lrwxrwxrwx   1 hil66 hypre  22 Apr 13 07:32 powerpc-ibm-aix5.2.0.0/hypre-1.9.0b.tar.gz -> ../hypre-1.9.0b.tar.gz
     lrwxrwxrwx   1 hil66 hypre  22 Apr 13 07:32 chaos_2_ia32_elan3/hypre-1.9.0b.tar.gz -> ../hypre-1.9.0b.tar.gz

25.  cd chaos_2_ia32_elan3

26.  ../mklibs -b hypre-1.9.0b.tar.gz install
     checking the hostname... blue
     checking the architecture... aix
             . . .
     Cleaning FEI_mv ...
     Cleaning test ...
\end{verbatim}
The build and installation is now complete on one SCF class of machines. For each of the 
remaining classes of SCF platforms, log-in, cd to /usr/gapps/hypre/<SYS\_TYPE> and repeat
step 26.
