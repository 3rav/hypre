%==========================================================================
\chapter{Installation}
\label{Installation}

The supported installation platforms (read pre-built \hypre{} releases)
is a dynamic, evolving list of LC/ASCI machines, reflecting a cross
section of the \hypre{} user community.
There are three types of \hypre{} installations that are updated and
maintained by members of the Scalable Linear Solvers project:
\begin{enumerate}

\item The "alpha" installation (designated by an "a" in the version number):
   \begin{itemize}
   \item intended to give users access to relatively recent changes.
   \item updated somewhat frequently.
   \item may be relatively unstable.
   \item three old installations are retained.
   \item installed on:
      \begin{itemize}
      \item CASC Sun workstation cluster
      \item CASC Linux workstation cluster
      \end{itemize}
   \end{itemize}

\item The "beta" installation (designated by an "b" in the version number):
   \begin{itemize}
   \item intended to give users access to relatively recent changes.
   \item intended to give users consistent access across several platforms.
   \item may be relatively unstable.
   \item three old installations are retained.
   \item installed on:
      \begin{itemize}
       \item CASC Sun workstation cluster
       \item CASC Linux workstation cluster
       \item MCR Linux cluster
       \item GPS Compaq cluster
       \item Linux TeraCluster Compaq cluster
       \item Blue-Pacific
       \item Frost
       \item SC cluster          (classified)
       \item Blue(Sky)-Pacific   (classified)
       \item White               (classified)
       \item Adelie              (classified)
       \item Q                   (classified)
      \end{itemize}
   \end{itemize}

\item The general release installation:
   \begin{itemize}
   \item intended to give users consistent access across several platforms.
   \item updated less frequently than internal installation.
   \item should be relatively stable.
   \item three old installations are retained.
   \item installed on:
      \begin{itemize}
       \item CASC Sun workstation cluster
       \item CASC Linux workstation cluster
       \item MCR Linux cluster
       \item GPS Compaq cluster
       \item Linux TeraCluster Compaq cluster
       \item Blue-Pacific
       \item Frost
       \item SC cluster          (classified)
       \item Blue(Sky)-Pacific   (classified)
       \item White               (classified)
       \item Adelie              (classified)
       \item Q                   (classified)
      \end{itemize}
   \end{itemize}

\end{enumerate}

Two mailing list are available to inform users of the hypre
linear solver library when new releases become available.
Announcements about general releases made on the `hypre-announce'
mailing list, and announcements about beta releases are made on the
`hypre-beta-announce' list. 

Subscriptions to either mailing list is handled throught the LLNL
Majordomo list server, Majordomo@lists.llnl.gov.i To add yourself
to a mailing list, send mail to <Majordomo@list.llnl.gov> with
the following command in the body of the email message:

 subscribe hypre-announce

or

 subscribe hypre-beta-announce

%==========================================================================
\section{Installation Procedures}
\label{Installation Procedures}

The installation is broken down into 2 parts: 
  \begin{itemize}
  \item building the distribution (tar file)
  \item installation
  \end{itemize}

Build the \hypre{} distribution:

\begin{enumerate}

   \item Set the version number in \file{configure.ac} by editing
   the \kbd{m4\_define(HYPRE\_VERSION, 1.8.3b)} line, additionally the
   \kbd{m4\_define(HYPRE\_DATE, 2004/02/14)} and,
   \kbd{m4\_define(HYPRE\_TIME, 10:35:00)} lines can be updated as well.
   The version number, by convention, use the following syntax `M.mm.rr'.
   Where the `M' is the major release number, `mm' is the minor
   relases number, and the `rr' an update number. Alpha, and Beta 
   releases are denoted by the following sytanx `M.nn.rra' for 
   alpha releases, and `M.nn.rrb' for appending beta releases. The version 
   number is used for the name of the release creation and is used
   in the tar file name and root directory for the distribution. The
   \file{configure} file mst then be created by running the 
   \kbd{config/boostrap} script, followed by a \kbd{cvs commit}.

   \item Create a symbolic tag for the current version
   of the cvs repository. To create a general release enter:
   \kbd{cvs rtag -b VM-nn-rr linear\_solvers}. Note that a
   general release requires that a cvs rtag with branch option,
   the alpha, and beta releases just use an rtag.
   Version numbers are mapped into a valid rtag branch name by the
   following conversion, `VM-nn-rr', note: the leading (capital)
   `V' and the version number separated with `-' replacing
   the `.'s. This convention will over come the strict cvs
   tag name restrictions (e.g., begin with a letter, and no `.'s).
   Alpha, and Beta releases are denoted by the following sytanx 
   `VM-nn-rra' for alpha releases, and `VM-nn-rrb' for beta 
   releases. An alpha or beta release do not make branches in 
   the repository. For a beta release the command would be similar to: 
   \kbd{cvs rtag VM-nn-rrb linear\_solvers}. Optionally, an rtag
   can be made relative to a date or time. For example set the
   rtag relative to last midnight enter:
   \kbd{cvs rtag -D 0:0 VM-nn-rrb linear\_solvers}.

   \item The distribution is created with the \kbd{mkdist} 
   Bourne-shell script located in the tools directory of the 
   \hypre{} repository.  Typing \kbd{mkdist -help} will give 
   general usage info. Enter: \kbd{mkdist VM-nn-rr}. This checks 
   out the rtag version (step 2 above) from the repository,
   reorganizes the file structure, build the documentation,
   and creates the distribution tar file \file{hypre-M.nn.rr.tar.gz}.

\end{enumerate}

Installing the \hypre{} distribution:

\begin{enumerate}

   \item Copy the tar file \file{hypre-M.nn.rr.tar.gz} to 
   \file{/usr/gapps/hypre/} and, \file{/usr/casc/hypre/}.
   Run the script \kbd{mkdistlinks} to create symbolic-links
   between the tar file and the 
   appropriate systems, and locations for the build. For non-CASC 
   machines there are two install locations:
      \begin{itemize}
       \item \file{/usr/gapps/hypre/}\textit{canonical system name}
       \item \file{/usr/casc/hypre/}\textit{canonical system name}
      \end{itemize}
   The \textit{canonical system name} is determined by 
   \kbd{config.guess}. The appropriate directory is found using
   the command:
      \begin{itemize}
       \item \file{/usr/gapps/hypre/`/usr/gapps/hypre/config.guess`}
       \item \file{/usr/casc/hypre/`/usr/gapps/hypre/config.guess`}
      \end{itemize}
   However if the above name is insufficient to distinguish between
   similar systems (e.g., AIX 5.1 runs on both Blue, and Frost. Both
   are IBM SP, but Blue is 604e processor and Frost is Power3, capable
   of running in 64-bit mode) then all bets are off. 
   
   The user's of \hypre{} needn't deal with these details for the 
   Non-CASC systems, because the \file{/usr/apps/hypre} directory
   will automatically be linked to the appropate hypre system 
   installation. Through a \kbd{SYS\_TYPE} environment variable
   Some mappings between the \file{/usr/gapps/hypre/} 
   \textit{canonical system name} directory names, the
   \kbd{SYS\_TYPE}, and the system  follow:
\begin{enumerate}
   \item \textit{canonical system name}  \kbd{SYS\_TYPE}   \kbd{System}
   \begin{itemize}
   \item  \file{alphaev67-dec-osf5.1}        tru64\_5sc           tc2k
   \item  \file{alphaev68-dec-osf5.1}        tru64\_5             gps
   \item  \file{i686-pc-linux-gnu}           redhat\_7a\_ia32      vivid,ilx
   \item  \file{i686-pc-linux-gnu-pengra}    redhat\_7a\_ia32\_qsw  pengra
   \item  \file{mips-sgi-irix6.5}            irix64              riptide
   \item  \file{powerpc-ibm-aix5.1.0.0}      aix\_5               blue
   \item  \file{powerpcll-ibm-aix5.1.0.0}    aix\_5\_ll            frost
   \end{itemize}
\end{enumerate}
   see: 
\htmladdnormallink{https://lc.llnl.gov/computing/techbulletins/bulletin258l.html}
{https://lc.llnl.gov/computing/techbulletins/bulletin258l.html}

   Copying the tar file on to an SCF machine requires going through
   File Interchange System (FIS) see: 
\htmladdnormallink{http://www-lc.llnl.gov:6336/dynaweb/LCdocs/fis/}
{http://www-lc.llnl.gov:6336/dynaweb/LCdocs/fis/}

   \item Run the Bourne-shell install script \kbd{mklibs} located in 
   the tools directory of the \hypre{} repository, a copy also
   exist at the root level of the hypre install directory. There are
   a couple specialized versions of \kbd{mklibs}, including:
   \kbd{mklibs.aix}, which creates combined 32, and 64 bit
   libraries, and \kbd{mklibs.babel} which adds --with-babel to
   each of the installations. Typing
   \kbd{mklibs -help} will provide usage information. Executing
   \kbd{mklibs} will untar the distribution, and build the hypre 
   libraries. The untaring of the distribution, creates a directory
   \file{hypre-M.nn.rr} which contains the following subdirectories:
      \begin{itemize}
       \item \file{bin}     contains hypre utilities
       \item \file{docs}    PostScript, PDF, and HTML documentation
       \item \file{src}     source code
      \end{itemize}
   The build process will create these directories:
      \begin{itemize}
       \item \file{debug}   debug compiled lib, and include subdirectories
       \item \file{include} include files
       \item \file{lib}     optimized compiled libraries
       \item \file{serial}  no parallel/serial compiled libraries
       \item \file{share}   the Babel SIDL file
       \item \file{threads} OpenMP compiled libraries
      \end{itemize}
   At a minimum an optimized and a debugged version of the library will
   be generated. Optionally, OpenMP versions of the libraries will
   be built and installed in a \file{threads}, and \file{threads/debug}
   directory, assuming the target system supports OpenMP. The
   user's will see this located at: \file{/usr/apps/hypre/hypre-M.nn.rr}.
   \kbd{mklibs} script has an optional third argument \kbd{install} which 
   will symbolically link the distribution to the `common' user
   accessable directories, depending on the installion type
   specified. Typical command to build a general distribution:\linebreak
   \kbd{mklibs -g hypre-M.nn.rr.tar.gz install}\linebreak
   The results of this build would create the following links:\linebreak
   \file{/usr/apps/hypre/bin -> /usr/gapps/hypre/.../hypre-M.nn.rr/bin}\linebreak
   \file{/usr/apps/hypre/debug -> /usr/gapps/hypre/.../hypre-M.nn.rr/debug}\linebreak
   \file{/usr/apps/hypre/docs -> /usr/gapps/hypre/.../hypre-M.nn.rr/docs}\linebreak
   \file{/usr/apps/hypre/include -> /usr/gapps/hypre/.../hypre-M.nn.rr/include}\linebreak
   \file{/usr/apps/hypre/lib -> /usr/gapps/hypre/.../hypre-M.nn.rr/lib}\linebreak
   \file{/usr/apps/hypre/src -> /usr/gapps/hypre/.../hypre-M.nn.rr/src}\linebreak
   \file{/usr/apps/hypre/threads -> /usr/gapps/hypre/.../hypre-M.nn.rr/threads}\linebreak
   \file{/usr/apps/hypre/alpha -> /usr/gapps/hypre/.../hypre-M.nn.rr}\linebreak
   \file{/usr/apps/hypre/beta -> /usr/gapps/hypre/.../hypre-M.nn.rr}\linebreak
   A beta install command of:\linebreak
   \kbd{mklibs -b hypre-M.nn.rrb.tar.gz install}\linebreak
   Would create the following links:\linebreak
   \file{/usr/apps/hypre/alpha -> /usr/gapps/hypre/.../hypre-M.nn.rrb}\linebreak
   \file{/usr/apps/hypre/beta -> /usr/gapps/hypre/.../hypre-M.nn.rrb}\linebreak
   An alpha type install command of:\linebreak
   \kbd{mklibs -a hypre-M.nn.rra.tar.gz install}\linebreak
   Would create the following links:\linebreak
   \file{/usr/apps/hypre/alpha -> /usr/gapps/hypre/.../hypre-M.nn.rra}\linebreak
   Note: in general, alpha releases are only released on CASC systems.

\end{enumerate}

%==========================================================================
\section{Installation Example}
\label{Installation Example}

The following is an example session creating a beta distribution:
\begin{ttfamily}
\begin{mdseries}
\linebreak
\$ \textbf{pwd}\linebreak
/home/hypre\linebreak
\$ \textbf{cvs checkout linear\_solvers}\linebreak
\begin{verbatim}
U linear_solvers/CHANGELOG
U linear_solvers/COPYRIGHT_and_DISCLAIMER
 . . .
U linear_solvers/utilities/utilities.h
U linear_solvers/utilities/version
\end{verbatim}
\$ \textbf{cvs history -T}\linebreak
\begin{verbatim}
T 2000-10-12 20:27 +0000 treadway linear_solvers [V1-3-1b:A]
T 2001-01-11 20:58 +0000 treadway linear_solvers [V1-4-0b:A]
 . . .
T 2001-07-27 16:57 +0000 treadway linear_solvers [V1-6-0:A]
T 2001-08-24 20:26 +0000 treadway linear_solvers [V1-7-0a:A]
\end{verbatim}
\$ \textbf{cd linear\_solvers}\linebreak
\$ \textbf{vi configure.ac}\linebreak
\begin{verbatim}
 . . .
m4_define(HYPRE_VERSION, 1.7.0a)
m4_define(HYPRE_DATE, 2001/08/24)
m4_define(HYPRE_TIME, 20:26:53)
 . . .
\end{verbatim}
\begin{bfseries}
\begin{verbatim}
m4_define(HYPRE_VERSION, 1.7.0b)
m4_define(HYPRE_DATE, 2001/11/13)
m4_define(HYPRE_TIME, 07:25:23)
\end{verbatim}
\end{bfseries}
"configure.ac" 423 lines, 13216 characters\linebreak
\$ \textbf{config/bootstrap}
\begin{verbatim}
Putting files in AC_CONFIG_AUX_DIR, `config'.
\end{verbatim}
\$ \textbf{cvs commit configure.ac configure}\linebreak
\textbf{beta release 1.7.0b}\linebreak
\begin{verbatim}
Release 1.7.0b
CVS: ----------------------------------------------------------------------
CVS: Enter Log.  Lines beginning with `CVS:' are removed automatically
CVS:
CVS: Modified Files:
CVS:   configure.ac configure
CVS: ----------------------------------------------------------------------
~
""/tmp/cvsAAABga42v" 9 lines, 345 characters
Checking in linear_solvers/configure.ac;
/home/casc/repository/linear_solvers/configure.ac,v  <--  configure.ac
new revision: 2.34; previous revision: 2.33
done
Checking in configure;
/home/casc/repository/linear_solvers/configure,v  <--  configure
new revision: 2.107; previous revision: 2.106
done
\end{verbatim}
:q
\$ \textbf{cvs rtag V1-7-0b linear\_solvers}\linebreak
\$ \textbf{./mkdist V1-7-0b}\linebreak
\begin{verbatim}
U linear_solvers/CHANGELOG
U linear_solvers/COPYRIGHT_and_DISCLAIMER
 . . .
U linear_solvers/utilities/utilities.h
U linear_solvers/utilities/version
checking the hostname... perrin
checking the architecture... solaris
 . . .
creating hypre-1.7.0b.tar file ...
\end{verbatim}
\$ \textbf{ls -l hypre-1.7*}\linebreak
\begin{verbatim}
-rw-rw-r--   1 treadway treadway 1532225 Nov 13 07:32 hypre-1.7.0b.tar.gz

hypre-1.7.0b:
total 20
-rw-rw----   1 treadway treadway    4384 Nov 13 07:32 CHANGELOG
-rw-rw----   1 treadway treadway    1645 Nov 13 07:32 COPYRIGHT_and_DISCLAIMER
drwxrwxr-x   2 treadway treadway     512 Nov 13 07:32 bin
drwxrwxr-x   4 treadway treadway     512 Nov 13 07:32 docs
drwxrwxr-x  27 treadway treadway    1024 Nov 13 07:32 src
\end{verbatim}
\$ \textbf{hypre-1.7.0b/src/utilities/version -number}\linebreak
1.7.0b\linebreak
\$ \textbf{scp hypre-1.7.0b.tar.gz blue:/usr/gapps/hypre}\linebreak
\$ \textbf{scp hypre-1.7.0b.tar.gz blue:/usr/casc/hypre}\linebreak
\$ \textbf{ftp fis}\linebreak
\begin{verbatim}
Connected to fis.llnl.gov.
220-                       NOTICE TO USERS
220-This is a Federal computer system and is the property of the
 . . .
220 reebok.llnl.gov FTP server (Version LLNL-22 built 08/13/01 07:32:54) ready.
Name (fis:treadway):
331 Password required for treadway.
Password:
230 User treadway logged in.
\end{verbatim}
ftp> \textbf{cd TO}\linebreak
250 CWD command successful.\linebreak
ftp> \textbf{binary}\linebreak
200 Type set to I.\linebreak
ftp> \textbf{put hypre-1.7.0b.tar.gz}\linebreak
200 PORT command successful.\linebreak
150 Opening BINARY mode data connection for hypre-1.7.0b.tar.gz.\linebreak
226 Transfer complete.\linebreak
local: hypre-1.7.0b.tar.gz remote: hypre-1.7.0b.tar.gz\linebreak
1663281 bytes sent in 0.49 seconds (3342.89 Kbytes/s)\linebreak
ftp> \textbf{quit}\linebreak
221 Goodbye.\linebreak
\$ \textbf{ssh blue}\linebreak
 . . .\linebreak
\$ \textbf{cd /usr/gapps/hypre}\linebreak
\$ \textbf{ls}\linebreak
\begin{verbatim}
aix_4                        hypre-1.7.0a.tar.gz
aix_4ll                      i686-pc-linux-gnu
aix_5                        i686-pc-linux-gnu-chaos
aix_5_ll                     i686-pc-linux-gnu-pengra
alphaev56-dec-osf4.0f        i686-pc-linux-gnu-vivid
alphaev56-dec-osf5.1         irix64
alphaev67-dec-osf5.0         irix_6.5_64
alphaev67-dec-osf5.1         mips-sgi-irix6.5
alphaev67-unknown-linux-gnu  mkdirlinks
alphaev68-dec-osf5.1         mkdistlinks
chaos_2_ia32                 mklibs
config.guess                 mklibs.aix
config.sub                   mklibs.babel
env.blue                     mklibs.tc2k
env.linux                    powerpc-ibm-aix4.3.3.0
hypre-1.2.0.tar.gz           powerpc-ibm-aix5.1.0.0
hypre-1.3.1b.tar.gz          powerpcll-ibm-aix4.3.3.0
hypre-1.4.0b.tar.gz          powerpcll-ibm-aix5.1.0.0
hypre-1.5.0b.tar.gz          redhat_7_ia32
hypre-1.6.0.tar.gz           tru64_5sc
\end{verbatim}
\$ \textbf{./mkdistlinks hypre-1.7.0b.tar.gz}\linebreak
\begin{verbatim}
lrwxrwxrwx   1 treadway hypre  22 Nov 13 07:32 alphaev67-dec-osf5.1/hypre-1.7.0b.tar.gz -> ../hypre-1.7.0b.tar.gz
lrwxrwxrwx   1 treadway hypre  22 Nov 13 07:32 alphaev67-unknown-linux-gnu/hypre-1.7.0b.tar.gz -> ../hypre-1.7.0b.tar.gz
lrwxrwxrwx   1 treadway hypre  22 Nov 13 07:32 alphaev68-dec-osf5.1/hypre-1.7.0b.tar.gz -> ../hypre-1.7.0b.tar.gz
lrwxrwxrwx   1 treadway hypre  22 Nov 13 07:32 i686-pc-linux-gnu/hypre-1.7.0b.tar.gz -> ../hypre-1.7.0b.tar.gz
lrwxrwxrwx   1 treadway hypre  22 Nov 13 07:32 i686-pc-linux-gnu-pengra/hypre-1.7.0b.tar.gz -> ../hypre-1.7.0b.tar.gz
lrwxrwxrwx   1 treadway hypre  22 Nov 13 07:32 mips-sgi-irix6.5/hypre-1.7.0b.tar.gz -> ../hypre-1.7.0b.tar.gz
lrwxrwxrwx   1 treadway hypre  22 Nov 13 07:32 powerpc-ibm-aix5.1.0.0/hypre-1.7.0b.tar.gz -> ../hypre-1.7.0b.tar.gz
lrwxrwxrwx   1 treadway hypre  22 Nov 13 07:32 powerpcll-ibm-aix5.1.0.0/hypre-1.7.0b.tar.gz -> ../hypre-1.7.0b.tar.gz
\end{verbatim}
\$ \textbf{cd `/usr/gapps/hypre/config.guess`}\linebreak
\$ \textbf{ls}\linebreak
\begin{verbatim}
AUTOTEST             hypre-1.2.0.tar.gz   hypre-1.7.0a
STLport-4.0          hypre-1.3.1b         hypre-1.7.0a.tar.gz
alpha                hypre-1.4.0b         hypre-1.7.0b.tar.gz
beta                 hypre-1.4.0b.tar.gz  include
bin                  hypre-1.5.0b         lib
debug                hypre-1.5.0b.tar.gz  src
docs                 hypre-1.6.0          threads
hypre-1.2.0          hypre-1.6.0.tar.gz
\end{verbatim}
\$ \textbf{../mklibs -b hypre-1.7.0b.tar.gz}\linebreak
\begin{verbatim}
checking the hostname... blue
checking the architecture... aix
 . . .
Very-cleaning FEI_mv ...
Very-cleaning test ...
\end{verbatim}
\$ \textbf{ls}\linebreak
\begin{verbatim}
AUTOTEST             hypre-1.2.0.tar.gz   hypre-1.7.0a
STLport-4.0          hypre-1.3.1b         hypre-1.7.0a.tar.gz
alpha                hypre-1.4.0b         hypre-1.7.0b
beta                 hypre-1.4.0b.tar.gz  hypre-1.7.0b.tar.gz
bin                  hypre-1.5.0b         include
debug                hypre-1.5.0b.tar.gz  lib
docs                 hypre-1.6.0          src
hypre-1.2.0          hypre-1.6.0.tar.gz   threads
\end{verbatim}
\$ \textbf{ls hypre-1.7.0b}\linebreak
\begin{verbatim}
CHANGELOG                 debug                     src
COPYRIGHT_and_DISCLAIMER  docs                      threads
README                    include
bin                       lib
\end{verbatim}
\$ \textbf{ls beta/lib}\linebreak
libHYPRE\_DistributedMatrix.a\linebreak
libHYPRE\_DistributedMatrixPilutSolver.a\linebreak
libHYPRE\_Euclid.a\linebreak
libHYPRE\_FEI.a\linebreak
libHYPRE\_IJ\_mv.a\linebreak
libHYPRE\_LSI.a\linebreak
libHYPRE\_MatrixMatrix.a\linebreak
libHYPRE\_ParaSails.a\linebreak
libHYPRE\_blas.a\linebreak
libHYPRE\_parcsr\_ls.a\linebreak
libHYPRE\_parcsr\_mv.a\linebreak
libHYPRE\_seq\_mv.a\linebreak
libHYPRE\_sstruct\_ls.a\linebreak
libHYPRE\_sstruct\_mv.a\linebreak
libHYPRE\_struct\_ls.a\linebreak
libHYPRE\_struct\_mv.a\linebreak
libHYPRE\_superlu.a\linebreak
libHYPRE\_utilities.a\linebreak
libkrylov.a\linebreak
\$ \linebreak
\end{mdseries}
\end{ttfamily}

Variations of mklib and use include:
\begin{bfseries}
\begin{verbatim}
Name          Purpose
mklibs        does default, debug, serial, and opt. threaded
mklibs.aix    build 32 and 64 bit libraries on white & frost
mklibs.babel  does the default plus builds babel libraries
mklibs.tc2k   build for tc2k

Name          Supported host
mklibs        riptide, gps, sc, Q
mklibs.aix    frost and white
mklibs.babel  Solaris, Linux, blue, SKY (AIX<Power 3)
mklibs.tc2k   tc2k
\end{verbatim}
\end{bfseries}
