%==========================================================================
\chapter{Creating and Installing a HYPRE Distribution}
\label{Creating and Installing a HYPRE Distribution}

Making a HYPRE release available to the user community involves creating a
distribution tar file, copying it to all the supported Livermore Computing
(LC) platforms and installing the libraries on those platforms.  As LC adds 
new machines, the latest HYPRE releases will be installed on those machines.

General and beta releases are installed on all supported LC platforms. While
alpha releases are installed {\bf only} on the CASC cluster.  Each release is 
configured and built with several options and stored in a subdirectory of the 
root directory as noted in this chart:

\begin{verbatim}
     Option                   Subdirectory Lib and Include Paths
MPI, no debug (default)         lib,                 include
MPI, debug                      debug/lib,           debug/include
No MPI, no debug                serial/lib,          serial/include
No MPI, debug                   serial/debug/lib,    serial/debug/include
OpenMP, no debug                threaded/lib,        threaded/include
OpenMP, debug	                threaded/debug/lib,  threaded/debug/include
\end{verbatim}


(comment on mk* scripts and define in later section)

%==========================================================================
\section{Creating a Distribution Tar File}
\label{Creating a Distribution Tar File}

The distribution tar file is created on a local CASC platform.  The following 
steps detail the actions to be taken as well as provide a sample command as 
appropriate.  Prior to executing these steps the release number will be assigned 
by the project team. In the examples below, M-mm-rr is a place-holder for the 
actual release number, which may or may not have an 'a' or 'b' appended; depending 
on the type of release being created.

\begin{enumerate}
\item Go to a local directory
\item Retrieve the latest version of \hypre{} from the CVS repository
      results in a new directory named linear\_solvers which contains the source code
\begin{verbatim}
      Example: cvs checkout linear_solvers
\end{verbatim}
\item Move to the linear\_solvers directory
\begin{verbatim}
      Example: cd linear_solvers
\end{verbatim}
\item Move to the config directory
\begin{verbatim}
      Example: cd config
\end{verbatim}
\item Change the HYPRE\_version number, date and time in the \file{configure.in} file
\begin{verbatim}
      Example: vi configure.in
               change numeric values following HYPRE_* in:
                  m4_define(HYPRE_VERSION, 1.9.0b)
                  m4_define(HYPRE_DATE, 2005/02/24)
                  m4_define(HYPRE_TIME, 14:26:53)
\end{verbatim}
\item Move to the linear\_solvers directory
\begin{verbatim}
      Example: cd linear_solvers
\end{verbatim}
\item Run the \file{bootstrap} script to create a new \file{configure} script
\begin{verbatim}
      Example:./config/bootstrap
\end{verbatim}
\item Commit new \file{configure} and \file{configure.in} files to repository
\begin{verbatim}
      Example: cvs commit 
\end{verbatim}
\item Tag the repository with this release number
\begin{verbatim}
      Example: cvs rtag VM-mm-rr linear_solvers
\end{verbatim}
\item Create the distribution tar file by running the \file{mkdist} script,
      located in the tools directory. Output is a new directory named \file{hypre-M.mm.rr}
      and a compressed tar file named \file{hypre-M.mm.rr.tar.gz}
\begin{verbatim}
      Example: ./tools/mkdist VM-mm-rr
\end{verbatim}
\item Copy the tar file to /usr/casc/hypre on the local system
\begin{verbatim}
      Example: cp hypre-M.mm.rr.tar.gz /usr/casc/hypre
\end{verbatim}
\end{enumerate}

%==========================================================================
\section{Installing a Distribution on the CASC Cluster}
\label{Installing a Distribution on the CASC Cluster}

On the CASC cluster the root directory for \hypre{} is \file{/usr/casc/hypre}.  The 
subdirectory (which corresponds to SYS\_TYPE) is \file{i686-pc-linux-gnu}. The following
procedure will build the library with its options and stored in the appropriate 
directories.

\begin{enumerate}
\item Move to the /usr/casc/hypre directory
\begin{verbatim}
      Example: cd /usr/casc/hypre
\end{verbatim}
\item Create symbolic links to the tar file by running the \file{mkdistlinks} script
\begin{verbatim}
      Example: ./mkdistlinks hypre-M.mm.rr.tar.gz
\end{verbatim}
\item Move to the CASC cluster subdirectory; i686-pc-linux-gnu
\begin{verbatim}
      Example: cd i686-pc-linux-gnu
\end{verbatim}
\item Build and install the library by runnning the \file{mklibs} script; which is
         located in the \file{/usr/casc/hypre} directory.
      The arguments -g (general), -b (beta) or -a (alpha) indicate the type of 
          release being generated.  If the install argument is included the 
          symbolic links will be created automatically.
\begin{verbatim}
      Example: ../mklibs <-g | -b | -a> hypre-M.mm.rr.tar.gz <install>
\end{verbatim}
\end{enumerate}

%==========================================================================
\section{Installing a Distribution on LC Platforms}
\label{Installing a Distribution on LC Platforms}

On all of the supported LC platforms, the root directory for \hypre{} is 
\file{/usr/gapps/hypre}.  Under the root are subdirectories for each class 
of machines identified by the LC defined environment variable, SYS\_TYPE.
The following table shows the SYS\_TYPE values for supported LC open (OCF) 
and secure (SCF) platforms as of April 2005.

\begin{verbatim}
  SYS_TYPE                       OCF                          SCF
aix_5_64                         ---                        tempest
aix_5_ll                        frost                     white, ice
aix_5_64_fed                   newberg                      uv, um
tru64_5                          gps                          sc
sles_9_ppc64                     bgl                          ---
chaos_2_ia32                     ilx                          ace
chaos_2_ia32_elan3        pengra, alc, mcr,             adelie, emperor,
                             pvc, bigdev                  lilac, gviz
chaos_2_ia32_elan4             thunder                        cub
\end{verbatim}

To relieve the user of needing to know the SYS\_TYPE on each platform, the path
\file{/usr/apps/hypre} will be symbolically linked to the appropriate subdirectory
on each platform by the development team.  

%==========================================================================
\subsection{Open Computing Facility (OCF) Platforms}
\label{Open Computing Facility (OCF) Platforms}

%==========================================================================
\subsection{Secure Computing Facility (SCF) Platforms}
\label{Secure Computing Facility (SCF) Platforms}


%==========================================================================
\section{Installation Scripts}
\label{Installation Scripts}

%==========================================================================
\subsection{mkdist}
\label{mkdist}

   The distribution is created with the \kbd{mkdist} 
   Bourne-shell script located in the tools directory of the 
   \hypre{} repository.  Typing \kbd{mkdist -help} will give 
   general usage info. Enter: \kbd{mkdist VM-nn-rr}. This checks 
   out the rtag version (step 2 above) from the repository,
   reorganizes the file structure, build the documentation,
   and creates the distribution tar file \file{hypre-M.nn.rr.tar.gz}.


%==========================================================================
\subsection{mkdistlinks}
\label{mkdistlinks}

%==========================================================================
\subsection{mklibs}
\label{mklibs}

   Run the Bourne-shell install script \kbd{mklibs} located in 
   the tools directory of the \hypre{} repository, a copy also
   exist at the root level of the hypre install directory. There are
   a couple specialized versions of \kbd{mklibs}, including:
   \kbd{mklibs.aix}, which creates combined 32, and 64 bit
   libraries, and \kbd{mklibs.babel} which adds --with-babel to
   each of the installations. Typing
   \kbd{mklibs -help} will provide usage information. Executing
   \kbd{mklibs} will untar the distribution, and build the hypre 
   libraries. The untaring of the distribution, creates a directory
   \file{hypre-M.nn.rr} which contains the following subdirectories:
      \begin{itemize}
       \item \file{bin}     contains hypre utilities
       \item \file{docs}    PostScript, PDF, and HTML documentation
       \item \file{src}     source code
      \end{itemize}
   The build process will create these directories:
      \begin{itemize}
       \item \file{debug}   debug compiled lib, and include subdirectories
       \item \file{include} include files
       \item \file{lib}     optimized compiled libraries
       \item \file{serial}  no parallel/serial compiled libraries
       \item \file{share}   the Babel SIDL file
       \item \file{threads} OpenMP compiled libraries
      \end{itemize}
   At a minimum an optimized and a debugged version of the library will
   be generated. Optionally, OpenMP versions of the libraries will
   be built and installed in a \file{threads}, and \file{threads/debug}
   directory, assuming the target system supports OpenMP. The
   user's will see this located at: \file{/usr/apps/hypre/hypre-M.nn.rr}.
   \kbd{mklibs} script has an optional third argument \kbd{install} which 
   will symbolically link the distribution to the `common' user
   accessable directories, depending on the installion type
   specified. Typical command to build a general distribution:\linebreak
   \kbd{mklibs -g hypre-M.nn.rr.tar.gz install}\linebreak
   The results of this build would create the following links:\linebreak
   \file{/usr/apps/hypre/bin -> /usr/gapps/hypre/.../hypre-M.nn.rr/bin}\linebreak
   \file{/usr/apps/hypre/debug -> /usr/gapps/hypre/.../hypre-M.nn.rr/debug}\linebreak
   \file{/usr/apps/hypre/docs -> /usr/gapps/hypre/.../hypre-M.nn.rr/docs}\linebreak
   \file{/usr/apps/hypre/include -> /usr/gapps/hypre/.../hypre-M.nn.rr/include}\linebreak
   \file{/usr/apps/hypre/lib -> /usr/gapps/hypre/.../hypre-M.nn.rr/lib}\linebreak
   \file{/usr/apps/hypre/src -> /usr/gapps/hypre/.../hypre-M.nn.rr/src}\linebreak
   \file{/usr/apps/hypre/threads -> /usr/gapps/hypre/.../hypre-M.nn.rr/threads}\linebreak
   \file{/usr/apps/hypre/alpha -> /usr/gapps/hypre/.../hypre-M.nn.rr}\linebreak
   \file{/usr/apps/hypre/beta -> /usr/gapps/hypre/.../hypre-M.nn.rr}\linebreak
   A beta install command of:\linebreak
   \kbd{mklibs -b hypre-M.nn.rrb.tar.gz install}\linebreak
   Would create the following links:\linebreak
   \file{/usr/apps/hypre/alpha -> /usr/gapps/hypre/.../hypre-M.nn.rrb}\linebreak
   \file{/usr/apps/hypre/beta -> /usr/gapps/hypre/.../hypre-M.nn.rrb}\linebreak
   An alpha type install command of:\linebreak
   \kbd{mklibs -a hypre-M.nn.rra.tar.gz install}\linebreak
   Would create the following links:\linebreak
   \file{/usr/apps/hypre/alpha -> /usr/gapps/hypre/.../hypre-M.nn.rra}\linebreak
   Note: in general, alpha releases are only released on CASC systems.


   

   Copying the tar file on to an SCF machine requires going through
   File Interchange System (FIS) see: 
\htmladdnormallink{http://www-lc.llnl.gov:6336/dynaweb/LCdocs/fis/}
{http://www-lc.llnl.gov:6336/dynaweb/LCdocs/fis/}


%==========================================================================
\section{Installation Example}
\label{Installation Example}

The following is an example session creating a beta distribution:
\begin{ttfamily}
\begin{mdseries}
\linebreak
\$ \textbf{pwd}\linebreak
/home/hypre\linebreak
\$ \textbf{cvs checkout linear\_solvers}\linebreak
\begin{verbatim}
U linear_solvers/CHANGELOG
U linear_solvers/COPYRIGHT_and_DISCLAIMER
 . . .
U linear_solvers/utilities/utilities.h
U linear_solvers/utilities/version
\end{verbatim}
\$ \textbf{cvs history -T}\linebreak
\begin{verbatim}
T 2000-10-12 20:27 +0000 treadway linear_solvers [V1-3-1b:A]
T 2001-01-11 20:58 +0000 treadway linear_solvers [V1-4-0b:A]
 . . .
T 2001-07-27 16:57 +0000 treadway linear_solvers [V1-6-0:A]
T 2001-08-24 20:26 +0000 treadway linear_solvers [V1-7-0a:A]
\end{verbatim}
\$ \textbf{cd linear\_solvers}\linebreak
\$ \textbf{vi configure.ac}\linebreak
\begin{verbatim}
 . . .
\end{verbatim}
\begin{bfseries}
\begin{verbatim}
m4_define(HYPRE_VERSION, 1.7.0b)
m4_define(HYPRE_DATE, 2001/11/13)
m4_define(HYPRE_TIME, 07:25:23)
\end{verbatim}
\end{bfseries}
"configure.ac" 423 lines, 13216 characters\linebreak
\$ \textbf{config/bootstrap}
\begin{verbatim}
Putting files in AC_CONFIG_AUX_DIR, `config'.
\end{verbatim}
\$ \textbf{cvs commit configure.ac configure}\linebreak
\textbf{beta release 1.7.0b}\linebreak
\begin{verbatim}
Release 1.7.0b
CVS: ----------------------------------------------------------------------
CVS: Enter Log.  Lines beginning with `CVS:' are removed automatically
CVS:
CVS: Modified Files:
CVS:   configure.ac configure
CVS: ----------------------------------------------------------------------
~
""/tmp/cvsAAABga42v" 9 lines, 345 characters
Checking in linear_solvers/configure.ac;
/home/casc/repository/linear_solvers/configure.ac,v  <--  configure.ac
new revision: 2.34; previous revision: 2.33
done
Checking in configure;
/home/casc/repository/linear_solvers/configure,v  <--  configure
new revision: 2.107; previous revision: 2.106
done
\end{verbatim}
:q
\$ \textbf{cvs rtag V1-7-0b linear\_solvers}\linebreak
\$ \textbf{./mkdist V1-7-0b}\linebreak
\begin{verbatim}
U linear_solvers/CHANGELOG
U linear_solvers/COPYRIGHT_and_DISCLAIMER
 . . .
U linear_solvers/utilities/utilities.h
U linear_solvers/utilities/version
checking the hostname... perrin
checking the architecture... solaris
 . . .
creating hypre-1.7.0b.tar file ...
\end{verbatim}
\$ \textbf{ls -l hypre-1.7*}\linebreak
\begin{verbatim}
-rw-rw-r--   1 treadway treadway 1532225 Nov 13 07:32 hypre-1.7.0b.tar.gz

hypre-1.7.0b:
total 20
-rw-rw----   1 treadway treadway    4384 Nov 13 07:32 CHANGELOG
-rw-rw----   1 treadway treadway    1645 Nov 13 07:32 COPYRIGHT_and_DISCLAIMER
drwxrwxr-x   2 treadway treadway     512 Nov 13 07:32 bin
drwxrwxr-x   4 treadway treadway     512 Nov 13 07:32 docs
drwxrwxr-x  27 treadway treadway    1024 Nov 13 07:32 src
\end{verbatim}
\$ \textbf{hypre-1.7.0b/src/utilities/version -number}\linebreak
1.7.0b\linebreak
\$ \textbf{scp hypre-1.7.0b.tar.gz blue:/usr/gapps/hypre}\linebreak
\$ \textbf{scp hypre-1.7.0b.tar.gz blue:/usr/casc/hypre}\linebreak
\$ \textbf{ftp fis}\linebreak
\begin{verbatim}
Connected to fis.llnl.gov.
220-                       NOTICE TO USERS
220-This is a Federal computer system and is the property of the
 . . .
220 reebok.llnl.gov FTP server (Version LLNL-22 built 08/13/01 07:32:54) ready.
Name (fis:treadway):
331 Password required for treadway.
Password:
230 User treadway logged in.
\end{verbatim}
ftp> \textbf{cd TO}\linebreak
250 CWD command successful.\linebreak
ftp> \textbf{binary}\linebreak
200 Type set to I.\linebreak
ftp> \textbf{put hypre-1.7.0b.tar.gz}\linebreak
200 PORT command successful.\linebreak
150 Opening BINARY mode data connection for hypre-1.7.0b.tar.gz.\linebreak
226 Transfer complete.\linebreak
local: hypre-1.7.0b.tar.gz remote: hypre-1.7.0b.tar.gz\linebreak
1663281 bytes sent in 0.49 seconds (3342.89 Kbytes/s)\linebreak
ftp> \textbf{quit}\linebreak
221 Goodbye.\linebreak
\$ \textbf{ssh blue}\linebreak
 . . .\linebreak
\$ \textbf{cd /usr/gapps/hypre}\linebreak
\$ \textbf{ls}\linebreak
\begin{verbatim}
aix_4                        hypre-1.7.0a.tar.gz
aix_4ll                      i686-pc-linux-gnu
aix_5                        i686-pc-linux-gnu-chaos
aix_5_ll                     i686-pc-linux-gnu-pengra
alphaev56-dec-osf4.0f        i686-pc-linux-gnu-vivid
alphaev56-dec-osf5.1         irix64
alphaev67-dec-osf5.0         irix_6.5_64
alphaev67-dec-osf5.1         mips-sgi-irix6.5
alphaev67-unknown-linux-gnu  mkdirlinks
alphaev68-dec-osf5.1         mkdistlinks
chaos_2_ia32                 mklibs
config.guess                 mklibs.aix
config.sub                   mklibs.babel
env.blue                     mklibs.tc2k
env.linux                    powerpc-ibm-aix4.3.3.0
hypre-1.2.0.tar.gz           powerpc-ibm-aix5.1.0.0
hypre-1.3.1b.tar.gz          powerpcll-ibm-aix4.3.3.0
hypre-1.4.0b.tar.gz          powerpcll-ibm-aix5.1.0.0
hypre-1.5.0b.tar.gz          redhat_7_ia32
hypre-1.6.0.tar.gz           tru64_5sc
\end{verbatim}
\$ \textbf{./mkdistlinks hypre-1.7.0b.tar.gz}\linebreak
\begin{verbatim}
lrwxrwxrwx   1 treadway hypre  22 Nov 13 07:32 alphaev67-dec-osf5.1/hypre-1.7.0b.tar.gz -> ../hypre-1.7.0b.tar.gz
lrwxrwxrwx   1 treadway hypre  22 Nov 13 07:32 alphaev67-unknown-linux-gnu/hypre-1.7.0b.tar.gz -> ../hypre-1.7.0b.tar.gz
lrwxrwxrwx   1 treadway hypre  22 Nov 13 07:32 alphaev68-dec-osf5.1/hypre-1.7.0b.tar.gz -> ../hypre-1.7.0b.tar.gz
lrwxrwxrwx   1 treadway hypre  22 Nov 13 07:32 i686-pc-linux-gnu/hypre-1.7.0b.tar.gz -> ../hypre-1.7.0b.tar.gz
lrwxrwxrwx   1 treadway hypre  22 Nov 13 07:32 i686-pc-linux-gnu-pengra/hypre-1.7.0b.tar.gz -> ../hypre-1.7.0b.tar.gz
lrwxrwxrwx   1 treadway hypre  22 Nov 13 07:32 mips-sgi-irix6.5/hypre-1.7.0b.tar.gz -> ../hypre-1.7.0b.tar.gz
lrwxrwxrwx   1 treadway hypre  22 Nov 13 07:32 powerpc-ibm-aix5.1.0.0/hypre-1.7.0b.tar.gz -> ../hypre-1.7.0b.tar.gz
lrwxrwxrwx   1 treadway hypre  22 Nov 13 07:32 powerpcll-ibm-aix5.1.0.0/hypre-1.7.0b.tar.gz -> ../hypre-1.7.0b.tar.gz
\end{verbatim}
\$ \textbf{cd `/usr/gapps/hypre/config.guess`}\linebreak
\$ \textbf{ls}\linebreak
\begin{verbatim}
AUTOTEST             hypre-1.2.0.tar.gz   hypre-1.7.0a
STLport-4.0          hypre-1.3.1b         hypre-1.7.0a.tar.gz
alpha                hypre-1.4.0b         hypre-1.7.0b.tar.gz
beta                 hypre-1.4.0b.tar.gz  include
bin                  hypre-1.5.0b         lib
debug                hypre-1.5.0b.tar.gz  src
docs                 hypre-1.6.0          threads
hypre-1.2.0          hypre-1.6.0.tar.gz
\end{verbatim}
\$ \textbf{../mklibs -b hypre-1.7.0b.tar.gz}\linebreak
\begin{verbatim}
checking the hostname... blue
checking the architecture... aix
 . . .
Very-cleaning FEI_mv ...
Very-cleaning test ...
\end{verbatim}
\$ \textbf{ls}\linebreak
\begin{verbatim}
AUTOTEST             hypre-1.2.0.tar.gz   hypre-1.7.0a
STLport-4.0          hypre-1.3.1b         hypre-1.7.0a.tar.gz
alpha                hypre-1.4.0b         hypre-1.7.0b
beta                 hypre-1.4.0b.tar.gz  hypre-1.7.0b.tar.gz
bin                  hypre-1.5.0b         include
debug                hypre-1.5.0b.tar.gz  lib
docs                 hypre-1.6.0          src
hypre-1.2.0          hypre-1.6.0.tar.gz   threads
\end{verbatim}
\$ \textbf{ls hypre-1.7.0b}\linebreak
\begin{verbatim}
CHANGELOG                 debug                     src
COPYRIGHT_and_DISCLAIMER  docs                      threads
README                    include
bin                       lib
\end{verbatim}
\$ \textbf{ls beta/lib}\linebreak
libHYPRE\_DistributedMatrix.a\linebreak
libHYPRE\_DistributedMatrixPilutSolver.a\linebreak
libHYPRE\_Euclid.a\linebreak
libHYPRE\_FEI.a\linebreak
libHYPRE\_IJ\_mv.a\linebreak
libHYPRE\_LSI.a\linebreak
libHYPRE\_MatrixMatrix.a\linebreak
libHYPRE\_ParaSails.a\linebreak
libHYPRE\_blas.a\linebreak
libHYPRE\_parcsr\_ls.a\linebreak
libHYPRE\_parcsr\_mv.a\linebreak
libHYPRE\_seq\_mv.a\linebreak
libHYPRE\_sstruct\_ls.a\linebreak
libHYPRE\_sstruct\_mv.a\linebreak
libHYPRE\_struct\_ls.a\linebreak
libHYPRE\_struct\_mv.a\linebreak
libHYPRE\_superlu.a\linebreak
libHYPRE\_utilities.a\linebreak
libkrylov.a\linebreak
\$ \linebreak
\end{mdseries}
\end{ttfamily}

Variations of mklib and use include:
\begin{bfseries}
\begin{verbatim}
Name          Purpose
mklibs        does default, debug, serial, and opt. threaded
mklibs.aix    build 32 and 64 bit libraries on white & frost
mklibs.babel  does the default plus builds babel libraries
mklibs.tc2k   build for tc2k

Name          Supported host
mklibs        riptide, gps, sc, Q
mklibs.aix    frost and white
mklibs.babel  Solaris, Linux, blue, SKY (AIX<Power 3)
mklibs.tc2k   tc2k
\end{verbatim}
\end{bfseries}
