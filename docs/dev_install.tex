%==========================================================================
\chapter{Installation}
\label{Installation}

There are two types of \hypre{} installations that are updated and
maintained by members of the Scalable Linear Solvers project:
\begin{enumerate}

\item The "internal" installation:
   \begin{itemize}
   \item intended to give users access to relatively recent changes.
   \item updated somewhat frequently (about once a week).
   \item may be relatively unstable.
   \item three old installation is retained.
   \item installed on:
      \begin{itemize}
      \item CASC Sun workstation cluster
      \end{itemize}
   \end{itemize}

\item The "external" installation:
   \begin{itemize}
   \item intended to give users consistent access across several platforms.
   \item updated less frequently than internal installation.
   \item should be relatively stable.
   \item three old installations are retained.
   \item installed on:
      \begin{itemize}
       \item CASC Sun workstation cluster
       \item Compass DEC cluster
       \item Blue-Pacific
       \item Forest DEC cluster  (classified)
       \item Blue(Sky)-Pacific   (classified)
      \end{itemize}
   \end{itemize}

\end{enumerate}

Rob Falgout is maintaining two lists of users who want to receive
notification of installation updates: one for "internal" updates and
one for "external" updates.  Users should send email to Rob at
rfalgout@llnl.gov to get on either list.

%==========================================================================
\section{Installation Procedures}
\label{Installation Procedures}

All installations are done with the \file{update} Bourne-shell script
located in the top-level directory of the \hypre{} repository.  Typing
\kbd{update -help} will give general usage info.
\begin{itemize}

\item Internal installation: (on CASC cluster only)
\begin{enumerate}

   \item \kbd{update internal} - This checks out the current version
   of the repository, tags it with a date (file \file{VERSION_DATE}),
   compiles the documentation, compiles the source (pure MPI and
   possibly threaded also), temporarily installs it, and sets the
   permissions of the temporary installation to appropriate values.
   This also creates the tar file \file{UPDATE.tar}, currently only
   used in the external installation.
  
   \item \kbd{update internal install} - This saves the previously
   installed version, moves the temporary installation into its
   correct location, and possibly deletes some older installations.
  
   \item Note that step 1 is not actually needed, but is recommended
   in case something doesn't work as planned (e.g., failed checkout,
   failed compile, etc.).  Also, step 1 provides the opportunity to
   test the library in non-CASC codes such as ARES or ALE3D via the
   temporary installation in the \file{UPDATE} subdirectory.
  
   \item Send email to appropriate users indicating the update.
   Browse the \file{updatelog.linear_solvers} file to assemble a list
   of changes since the last update.

\end{enumerate}

\item External installation: (on CASC cluster)
\begin{enumerate}

   \item \kbd{update external}

   \item \kbd{update external install}

   \item Note that this automatically copies the \file{UPDATE.tar}
   file from the internal installation to use here.  That is, the
   external installation is always the same as some version of the
   internal installation.

\end{enumerate}

\item External installation: (on all other supported machines,
e.g. the Compass cluster)
\begin{enumerate}

   \item FTP the file \file{UPDATE.tar} to west (for example), and put it
   in the install directory (typing \kbd{update -dirs} will give a list).

   \item \kbd{tar xf UPDATE.tar}

   \item \kbd{update external -dec}

   \item \kbd{update external -dec install}

\end{enumerate}

\item Doublecheck the permissions on some of the files to make sure that
they are set correctly.  The installation should be world-readable,
and should give write access to people in the \file{hypre} group.  If
you find something is not getting set correctly, fix manually, and
report to the person maintaining the \file{update} script.

\end{itemize}

