\chapter{Solvers and Preconditioners}
\label{solvers}

The procedure for setup and use of solvers and preconditioners is largely
the same. We will refer to them both as solvers in the sequel except when noted. 
In normal usage, 
the preconditioner is chosen and constructed before the solver,
and then handed to the solver as part of the solver's setup.

{\bf Setup:}

\begin{enumerate}

\item
{\bf Pass the information defining the problem to the solver.} In the typical user cycle, the user
has passed this information into a matrix through one of the conceptual interfaces prior to
setting up the solver. In this situation, the problem definition information is then passed to
the solver by passing the constructed matrix into the solver. As described before, the matrix
and solver must be compatible, in that the matrix must provide the services needed by
the solver. Krylov solvers, for example, need only the matrix-vector multiplication operation,
contained in the LinearOperator interface. 
Most preconditioners, on the other hand, have additional requirements such as access to the
coefficients of the matrix, e.g. the GetRow function in the \code{HYPRE\_DistributedMatrix} interface.

\item
{\bf Choose parameters that control construction of the preconditioner/solver.} 
Parameters are chosen through the "Set" calls provided by the solver.
As is true throughout \hypre, all parameters have reasonable defaults if not chosen.
Note that in \hypre, we convergence criteria can be chosen after the preconditioner/solver
has been setup.

\item
{\bf Pass the preconditioner to the solver.} For solvers that are not preconditioned, this step
is omitted. The preconditioner is passed through the \code{SetPreconditioner} call. The object passed
in as the preconditioner must be a fully constructed Solver.

\item
{\bf Set up the solver.} In the "procedural C" interface for \hypre, this is the \code{Setup} call.
In the ESI-compliant interface for \hypre, this is the \code{GetConstructedObject} from the
solver builder interface.

\end{enumerate}

At this point, the solver/preconditioner is fully constructed and ready for use. 

{\bf Use:}

\begin{enumerate}

\item
{\bf Set convergence criteria.} Convergence can be controlled by the number of iterations,
as well as various tolerances such as relative residual, preconditioned residual, etc.
Like all parameters, reasonable defaults are used. Generally speaking, Krylov solvers
default to a tolerance-based convergence criteria, while solvers used traditionally as
preconditioners default to convergence based on the number of iterations (usually a single
iteration). Users are free to change these, though care must be taken. Some Krylov solvers,
for example, are not guaranteed to converge unless the preconditioner is a "constant operator",
requiring a fixed number of iterations. For these solvers, a preconditioner
with a tolerance-based convergence criteria will not usually work.

\item
{\bf Apply the solver to the right hand side.} For the procedural C interface, this is the 
\code{Solve} function. From the ESI-compliant interface, this is the \code{Apply} function.

\end{enumerate}
