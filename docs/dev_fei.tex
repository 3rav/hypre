\chapter{Finite Element Interface}
\label{chapter-FEI}

Charles Tong deserves credit for writing this.  Thanks, Charles!

\section{Introduction}

The finite element interface (FEI) describes a suite of function interfaces 
that permits finite element application developers to use various linear system
solvers with little change in the user code.  The actual implementation of
the FEI, however, depends on the underlying linear solver package.  For example,
FEI implementations for ISIS++ and Aztec have already been been developed.
To allow application users (e.g. ALE3D) access to the suite of high performance
preconditioners in \hypre{}, an HYPRE-FEI has also been implemented, and its
capabilities and usage are described in this document.

\section{HYPRE-FEI Capabilities}

By providing an FEI link to \hypre{}, FEI users can access the Krylov solvers and
preconditioners available in \hypre{}.  Several direct and iterative solvers are
currently available to users :

\begin{enumerate}
\item conjugate gradient (CG), 
\item generalized minimal residual (GMRES) methods,
\item SuperLU direct solver,
\item SuperLU direct solver with iterative refinement, and
\item Y12M direct solver.
\end{enumerate}

The \hypre{} preconditioners available via the FEI currently are :

\begin{enumerate}
\item diagonal, 
\item parallel incomplete LU with threshold (PILUT),
\item parallel algebraic multigrid (BOOMERAMG),
\item parallel sparse approximate inverse (ParaSails).
\end{enumerate}

To choose between these options as well as the parameters specific to a
particular method, users are only required to call the FEI {\sf parameters}
functions with the corresponding settings.  (Refer to later sections on   
a list of parameter options.)

Capabilities under development are :
\begin{enumerate}
\item to form and solve a reduced linear system (in view of constraints), and
\item a link to the ML's parallel smoothed aggregation method.
\end{enumerate}

\section{HYPRE-FEI Installation}

The ultimate objective is for users to obtain a copy of \hypre{}, configure, and
compile it.  However, at this stage the most convenient way for COMPASS and
Pacific Blue users to obtain a collection of library functions and link them
to the application code.  In this section, detailed description of this linking
will be described.

\subsection{Obtaining the library files}

Upon request, users will obtain a collection of library files for \hypre{}.  The
files are :

\begin{enumerate}
\item {\sf libHYPRE\_FEI\_BASE.a}
\item {\sf libHYPRE\_FEI.a}
\item {\sf libHYPRE\_parcsr\_ls.a}
\item {\sf libHYPRE\_ParaSails.a}
\item {\sf libHYPRE\_FEI\_isismv.a}
\item {\sf libHYPRE\_IJ\_mv.a}
\item {\sf libHYPRE\_utilities.a}
\item {\sf libHYPRE\_parcsr\_mv.a}
\item {\sf libHYPRE\_MatrixMatrix.a}
\item {\sf libHYPRE\_DistributedMatrix.a}
\item {\sf libHYPRE\_DistributedMatrixPilut.a}
\item {\sf libsuperlu\_alpha.a}
\item {\sf libHYPRE\_seq\_mv.a}
\end{enumerate}

\subsection{Linking with the library files}

It is recommended that these library files are linked in this given order.
(For example, it these files are stored in the /usr/tmp directory, then
one way is to link them with

\begin{tabbing}
\hspace{0.5in} \= -L/usr/tmp -lHYPRE\_FEI\_BASE -lHYPRE\_FEI -lHYPRE\_parcsr\_ls \\
\> -lHYPRE\_ParaSails -lHYPRE\_FEI\_isismv -lHYPRE\_IJ\_mv -lHYPRE\_utilities \\
\> -lHYPRE\_parcsr\_mv -lHYPRE\_MatrixMatrix -lHYPRE\_DistributedMatrix \\
\> -lHYPRE\_DistributedMatrixPilut -lsuperlu\_alpha -lHYPRE\_seq\_mv
\end{tabbing}

along with all the other libraries.  

Since some of these library files make calls to LAPACK and BLAS functions, 
the corresponding libraries need to be linked along with (placed after) these 
library files.  For example, on the DEC Compass system, it suffices to link
with the {\sf dxml} library. (So {\sf -ldxml} is placed after the above link
sequence, with {\sf -lm} placed after {\sf -ldxml}).

\subsection{Some more caveats especially for ALE3D developers}

Building an application executable often requires linking with many different
software packages, and many software packages use some LAPACK and/or BLAS
functions.  In order to alleviate the problem of multiply defined functions
at link time, it is recommended that all software libraries are stripped of
all LAPACK and BLAS function definitions.  These LAPACK and BLAS functions 
should then be resolved at link time by linking with the system LAPACK and
BLAS libraries (e.g. dxml on DEC Compass).  Both \hypre{} and SuperLU were
built with this in mind.  However, some other software library files needed
by ALE3D may have the BLAS functions defined in them (notably libchaco.a
in /usr/apps/bdivport/lib).  To avoid the problem of multiply defined functions,
it is recommended that the libchaco.a file be stripped of the BLAS functions.
Stripping can be achieved by calling the following Unix command

\begin{tabbing}
\hspace{0.5in} \= {\sf ar -d libchaco.a symmlqblas.o}
\end{tabbing}

 
\section{HYPRE-FEI Usage}

Again, the HYPRE-FEI users interact directly with \hypre{} only via the {\sf parameters}
function.  In this section we list all parameters recognized by the HYPRE-FEI.

\begin{description}
\item[solver XXX] - where XXX specifies one of {\sf cg}, {\sf gmres},
                    {\sf superlu}, or {\sf superlux}.  The default is {\sf gmres}.
\item[preconditioner XXX] - where XXX is one of {\sf diagonal}, {\sf pilut},
                    {\sf parasails}, or {\sf boomeramg}. The default 
                    is {\sf diagonal}.
\item[gmres-dim XXX] - where XXX is an integer specifying the value of m in
                       restarted GMRES(m).  The default value is 50.
\item[maxIterations XXX] - where XXX is an integer specifying the maximum number
                           of iterations permitted for CG or GMRES.
                           The default value is 1000.
\item[tolerance XXX] - where XXX is a floating point number specifying the 
                       termination criterion for CG or GMRES.  The default value is
                       1.0E-10.
\item[pilut-row-size XXX] - where XXX is an integer specifying the maximum
                       number of nonzeros kept in the formation of imcomplete L
                       and U).  If this is not called, a value will be selected
                       based on the sparsity of the matrix.
\item[pilut-drop-tol XXX] - where XXX is a floating point number specifying the 
                       threshold to drop small entries in L and U.  The default
                       value is 0.0.
\item[superlu-ordering XXX] - where XXX specifies on of {\sf natural} or
                       {\sf mmd} (minimum degree ordering).  This ordering
                       is used to minimize the number of nonzeros generated
                       in the LU decomposition.  The default is natural ordering.
\item[superlu-scale XXX] - where XXX specifies on of {\sf y} (perform row
                       and column scalings before decomposition) or {\sf n}.
                       The default is no scaling.
\item[amg-coarsen-type XXX] - where XXX specifies on of {\sf falgout},
                       {\sf ruge}, or {\sf default} coarsening for BOOMERAMG.
\item[amg-num-sweeps XXX] - where XXX is an integer specifying the number of
                       pre- and post-smoothing at each level of BOOMERAMG.
                       The default is one pre- and one post-smoothings.
\item[amg-relax-type XXX] - where XXX is one of {\sf jacobi} (Damped Jacobi),
                       {\sf gs-slow} (sequential Gauss-Seidel), {\sf gs-fast}
                       (Gauss-Seidel on interior nodes), {\sf hybrid},
                       or {\sf direct}. The default is {\sf hybrid}.
\item[amg-relax-weight XXX] - where XXX is a floating point number between 0 and 1
                       specifying the damping factor for BOOMERAMG's damped
                       Jacobi smoother.  The default value is 0.5.
\item[amg-strong-threshold XXX] - where XXX is a floating point number between 0 
                       and 1 specifying the threshold used to determine
                       strong coupling in BOOMERAMG's coasening.  The default 
                       value is 0.25.
\item[parasails-threshold XXX] - where XXX is a floating point number between 0 
                       and 1 specifying the threshold used to prune small entries
                       in setting up the sparse approximate inverse.  The default
                       value is 0.0.
\item[parasails-nlevels XXX] - where XXX is an integer larger than 0 specifying 
                       the desired sparsity of the approximate inverse.  The
                       default value is 1.
\end{description}


\section{HYPRE-FEI Extensions}

In order to support the reduced system formulation for ALE3D applications
with slide surfaces, two additional functions are provided via HYPRE-FEI
on top of the base FEI specification :

\begin{description}
\item[loadSlaveList(int,int*)] - this function is used to load information
                        about which unknowns are the slave nodes in the
                        slide surfaces.  This information is useful for
                        the construction of reduced linear system.
\item[buildReducedSystem() ] - this function can be called to reduced a linear
                        system with constraints to a better-conditioned reduced
                        form.
\end{description}

\section{HYPRE-FEI Status}

All the solvers and preconditioners have been tested with a few ALE3D test
problem on the DEC Compass clusters with 1 processor.  The reduced system
function has been tested on 1 and 2 processors for a toy test problem.  We
are in the process of testing them on multiple processors and on real test
problems with ALE3D.

\section{HYPRE-FEI Comments and Contacts}

Any comments about this HYPRE-FEI interface can be directed to Charles Tong
(X23411 or chtong@llnl.gov) or Edmond Chow (X31915 or chow8@llnl.gov).

