\chapter{Finite Element Interface}
\label{chapter-FEI}

\section{Introduction}

The finite element interface (FEI) describes a suite of function interfaces 
that permits finite element application developers to use various linear 
system solver packages with little change in the user code.  Its 
specification and its first implementation have been developed at Sandia 
in past years.  Encapsulated in the FEI implementation (FEI 2.x.x developed at 
Sandia) is an abstract linear solver class called {\sf LinearSystemCore}, which 
can be binded at run time to any linear solver object (that inherits from the 
abstract class).  For example, the {\sf LinearSystemCore} implementation for 
\hypre{}, FETI, ISIS++, PETSC, and Prometheus have already been been 
developed.  To allow application users access to the suite of high performance 
preconditioners in \hypre{}, an \hypre{}-based 
{\sf LinearSystemCore} class has been defined and 
implemented, and its capabilities and usage are described in this document.

The \hypre{} {\sf LinearSystemCore} has been designed not only to be used via 
the FEI, but also to be accessed directly.  Thus, it represents yet another 
matrix/solver interface in addition to the existing interfaces (for example, 
the IJ interface).

\section{HYPRE\_LinSysCore Capabilities}

An {\sf HYPRE\_LinSysCore} has been defined that inherits from the abstract 
{\sf LinearSystemCore} class (there is a {\sf NOFEI} option that allows the
{\sf HYRE\_LinSysCore} to be built independent of the {\sf FEI}).  It can be 
instantiated indirectly by applications using the function call 
{\sf HYPRE\_LinSysCore\_Create()}.  The {\sf HYPRE\_LinSysCore} thus created 
should be passed to the {\sf FEI} implementation using the {\sf FEI\_Create} 
function.  By providing an {\sf LinearSystemCore} link to \hypre{}, FEI 
users can access the linear solvers and preconditioners available in 
\hypre{}.  In the following we summarize the \hypre{} capabilities 
accessible through the {\sf HYPRE\_LinSysCore} interface.

To choose between these options as well as the parameters specific to a
particular method, users are only required to call the FEI {\sf parameters}
({\sf HYPRE\_parameters} for standalone use)
functions with the corresponding settings.  (Refer to later sections on   
a list of parameter options.)  Usage of third party software (SuperLU, ML)
will require further tuning.

\subsection{Parallel Matrix and Vector Construction} 

This linear solver interface allows users access to the parallel compressed
sparse row ({\sf ParCSR}) matrix class in \hypre{}.  
The requirements about how the global matrix is partitioned among the
processors are that each processor holds a contiguous block of rows 
and the equation numbers in processors of lower rank are lower than those
in processors of higher rank.  The matrix can be loaded in parallel - 
a row or a block of rows at a time.  The solution and right hand side vectors
are constructed accordingly.

\subsection{Sequential and Parallel Solvers} 

Several direct and iterative solvers are currently available to users :

\begin{enumerate}
\item Krylov solvers (CG, GMRES, FGMRES, TFQMR, BiCGSTAB, symmetric QMR),
\item Boomeramg (a parallel algebraic multigrid solver),
\item SuperLU direct solver (sequential),
\item SuperLU direct solver with iterative refinement (sequential), 
\end{enumerate}

\subsection{Parallel Preconditioners} 

The \hypre{} preconditioners available via the {\sf LinearSystemCore} 
currently are :

\begin{enumerate}
\item none,
\item diagonal, 
\item parallel incomplete LU with threshold (PILUT),
\item another parallel incomplete LU (Euclid),
\item parallel sparse approximate inverse (ParaSails),
\item parallel algebraic multigrid (BOOMERAMG),
\item parallel domain decomposition with inexact local solves (DDIlut), 
\item least-squares polynomial preconditioner,
\item $2 \times 2$ block preconditioner, and
\item $2 \times 2$ Uzawa preconditioner.
\end{enumerate}

\subsection{Matrix Reduction} 

There are two matrix reduction algorithms available in {\sf HYRE\_LinSysCore} 
which are presented in the following subsections.

\subsubsection{Schur Complement Reduction}

The incoming linear system of equations is assumed to be in the form :
\[ \left[ 
\begin{array}{cc} 
   D   & B \\
   B^T & 0
\end{array}
  \right] 
  \left[
\begin{array}{c} 
   x_1 \\
   x_2
\end{array}
  \right] 
  =
  \left[
\begin{array}{c} 
   b_1 \\
   b_2
\end{array}
  \right] 
\]
where $D$ is a diagonal matrix.  After Schur complement reduction is applied, 
the resulting linear system becomes
$$
- B^T D^{-1} B x_2 = b_2 - B^T D^{-1} b_1.
$$

\subsubsection{Slide Surface Reduction}

With the presence of slide surfaces, the matrix is in the same form as in the
case of Schur complement reduction.  Here $A$ represents the relationship 
between the master, slave, and other degrees of freedom.  The matrix block 
$[B^T 0]$ corresponds to the constraint equations.  The goal of reduction
is to eliminate the constraints.  As proposed by Manteuffel, the trick is
to re-order the system into a $3 \times 3$ block matrix.
\[ 
\left[ 
\begin{array}{ccc} 
   A_{11}  & A_{12} & N \\
   A_{21}  & A_{22} & D \\
   N_{T}   & D      & 0 \\
\end{array}
\right] 
=
\left[ 
\begin{array}{ccc} 
   A_{11}       & \hat{A}_{12} \\
   \hat{A}_{21} & \hat{A}_{22}.
\end{array}
\right] 
\]
The reduced system has the form :
$$
(A_{11} - \hat{A}_{21} \hat{A}_{22}^{-1} \hat{A}_{12}) x_1 =
b_1 - \hat{A}_{21} \hat{A}_{22}^{-1} b_2,
$$
which is symmetric positive definite (SPD) if the original matrix is PD.
In addition, $\hat{A}_{22}^{-1}$ is easy to compute. 

There are three slide surface reduction algorithms in \hypre{}.  
The first follows the matrix formulation in this section.  The second is
similar except that it replaces the eliminated slave equations with
identity rows so that the degree of freedom at each node is preserved.
This is essential for certain block algorithms such as the smoothed
aggregation multilevel preconditioners.
The third is similar to the second except that it is more general and 
can be applied to problems with intersecting slide surfaces (sequential only
for intersecting slide surfaces).

\subsection{Other Features} 

To improve the efficiency of the \hypre{} solvers, a few other features 
have been incorporated.  We list a few of these features below :

\begin{enumerate}
\item Preconditioner reuse - For multiple linear solves with matrices that are
      slightly perturbed from each other, oftentimes the use of the same 
      preconditioners can save preconditioner setup times but suffer little
      convergence rate degradation.
\item Projection methods - For multiple solves that use the same matrix,
      previous solution vectors can sometimes be used to give a better initial
      guess for subsequent solves.  Two projection schemes have been implemented
      in \hypre{} - A-conjugate projection (for SPD matrices) and minimal 
      residual projection (for both SPD and non-SPD matrices).
\item The sparsity pattern of the matrix is in general not destroyed after
      it has been loaded to an \hypre{} matrix.  But if the matrix is not to
      be reused, an option is provided to clean up this pattern matrix to
      conserve memory usage.
\end{enumerate}

\section{HYPRE\_LSI Interface}

As described briefly before, users can use {\sf HYPRE\_LinSysCore}
functions without going through the FEI.  To create a FEI-free {\sf HYPRE\_LSI} 
library, a ``make nofei" should be called during installation.
Since the {\sf HYPRE\_LSI} is specified as a C++ class, the corresponding
C wrapper functions have been written to aid the non-C++ users.  The
C functions are given below.

\begin{tabbing}
{\sf HYPRE\_LinSysCore\_create(LinSysCore **lsc, MPI\_Comm comm)} \\[1mm]
{\sf HYPRE\_LinSysCore\_destroy(LinSysCore **lsc)} \\[1mm]
{\sf HYPRE\_parameters(LinSysCore *lsc, int nParams, char **params)} \\[1mm]
{\sf HYPRE\_setGlobalOffsets(LinSysCore* lsc, int leng, int* nodeOffsets,} \\
\hspace{1.0in} {\sf int* eqnOffsets, int* blkEqnOffsets)} \\[1mm]
{\sf HYPRE\_setMatrixStructure(LinSysCore *lsc, int** ptColIndices,} \\
\hspace{1.0in} {\sf int* ptRowLengths, int** blkColIndices, int* blkRowLengths, int* ptRowsPerBlkRow)} \\[1mm]
{\sf HYPRE\_resetMatrixAndVector(LinSysCore *lsc, double val)} \\[1mm]
{\sf HYPRE\_resetMatrix(LinSysCore *lsc, double val)} \\[1mm]
{\sf HYPRE\_resetRHSVector(LinSysCore *lsc, double val)} \\[1mm]
{\sf HYPRE\_sumIntoSystemMatrix(LinSysCore *lsc, int numPtRows, const int* ptRows,}\\
\hspace{1.0in} {\sf int numPtCols, const int* ptCols, int numBlkRows, const int* blkRows,} \\
\hspace{1.0in} {\sf int numBlkCols, const int* blkCols, const double* const* values)} \\[1mm]
{\sf HYPRE\_sumIntoRHSVector(LinSysCore *lsc, int num, const double* values, const int* indices)} \\[1mm]
{\sf HYPRE\_matrixLoadComplete(LinSysCore *lsc)} \\[1mm]
{\sf HYPRE\_enforceEssentialBC(LinSysCore *lsc, int* globalEqn, double* alpha,
                             double* gamma, int leng)} \\[1mm]

{\sf HYPRE\_enforceRemoteEssBCs(LinSysCore *lsc,int numEqns,int* globalEqns, int** colIndices,} \\
\hspace{1.0in} {\sf int* colIndLen, double** coefs)} \\[1mm]

{\sf HYPRE\_enforceOtherBC(LinSysCore *lsc, int* globalEqn, double* alpha, double *beta} \\
\hspace{1.0in} {\sf double* gamma, int leng)} \\[1mm]

{\sf HYPRE\_putInitialGuess(LinSysCore *lsc, const int* eqnNumbers,
                          const double* values, int leng)} \\[1mm]
{\sf HYPRE\_getSolution(LinSysCore *lsc, double *answers, int leng)} \\[1mm]

{\sf HYPRE\_getSolnEntry(LinSysCore *lsc, int eqnNumber, double *answer)} \\[1mm]

{\sf HYPRE\_formResidual(LinSysCore *lsc, double *values, int leng)} \\[1mm]

{\sf HYPRE\_launchSolver(LinSysCore *lsc, int *solveStatus, int *iter)} \\[1mm]
\end{tabbing}

\section{HYPRE LinearSystemCore Parameters}

Again, FEI/HYPRE users interact directly with \hypre{} only via the 
{\sf parameters} function (or {\sf FEI\_parameters} in C).  In this 
section we list all parameters recognized by the {\sf HYPRE\_LinSysCore}.

\subsection{Parameters for Solvers and Preconditioners}

\begin{description}
\item[solver xxx] where xxx specifies one of {\sf cg}, {\sf gmres},
           {\sf fgmres}, {\sf bicgs}, {\sf bicgstab}, {\sf tfqmr}, 
           {\sf symqmr}, {\sf superlu}, or {\sf superlux}.  The 
           default is {\sf gmres}.
           The solver type can be followed by {\sf override} to
           specify its priority when multiple solvers are declared
           at random order.
\item[preconditioner xxx] where xxx is one of {\sf diagonal}, {\sf pilut},
           {\sf euclid}, {\sf parasails}, {\sf boomeramg}, {\sf ddilut}, 
           {\sf poly}, {\sf blockP}, {\sf Uzawa}, or {\sf mli}. The 
           default is {\sf diagonal}.  Another option for 
           xxx is {\sf reuse} which allows the preconditioner to be reused 
           (this is to be set after a preconditioner has been set up already).
           The preconditioner type can be followed by {\sf override} to
           specify its priority when multiple preconditioners are declared
           at random order.
\item[maxIterations xxx] where xxx is an integer specifying the maximum 
           number of iterations permitted for the iterative solvers.
           The default value is 1000.
\item[tolerance xxx] where xxx is a floating point number specifying the 
           termination criterion for the iterative solvers.  The default 
           value is 1.0E-6.
\item[gmresDim xxx] where xxx is an integer specifying the value of m in
           restarted GMRES(m).  The default value is 100.
\item[stopCrit xxx] where xxx is one of {\sf absolute} or {\sf relative}
           stopping criterion.
\item[superluOrdering xxx] - where xxx specifies one of {\sf natural} or
           {\sf mmd} (minimum degree ordering).  This ordering
           is used to minimize the number of nonzeros generated
           in the LU decomposition.  The default is natural ordering.
\item[superluScale xxx] where xxx specifies one of {\sf y} (perform row
           and column scalings before decomposition) or {\sf n}.
           The default is no scaling.
\end{description}

\subsection{Parameters for ILUT, SPAI, and Polynomial Preconditioners}

\begin{description}
\item[pilutFillin xxx] where xxx is an integer specifying the maximum
           number of nonzeros kept in the formation of imcomplete L
           and U.  If this is not called, a value will be selected
           based on the sparsity of the matrix.
\item[pilutDropTol xxx] where xxx is a floating point number specifying the 
           threshold to drop small entries in L and U.  The default
           value is 0.0.
\item[ddilutFillin xxx] where xxx is a floating point number specifying 
           the maximum number of nonzeros kept in the formation of local 
           incomplete L and U (a value of $0.0$ means same sparsity as $A$,
           and a value of $1.0$ means two times the number of nonzeros as
           $A$.).  If this is not called, a value will be selected
           based on the sparsity of the matrix.
\item[ddilutDropTol xxx] where xxx is a floating point number specifying the 
           threshold to drop small entries in L and U.  The default
           value is 0.0.
\item[pilutFillin xxx] where xxx is an integer specifying the 
           maximum number of nonzeros to keep for each row of L and U.
           The default value is the number of nonzeros in the matrix A.
\item[pilutDropTol xxx] where xxx is a floating point number specifying the 
           threshold to drop small entries in L and U.  The default
           value is 0.0.
\item[euclidNlevels xxx] where xxx is an non-negative integer specifying 
           the desired sparsity of the incomplete factors.  The
           default value is 0.
\item[euclidThreshold xxx] where xxx is a floating point number specifying 
           the threshold used to sparsify the incomplete factors.  The default
           value is 0.0.
\item[parasailsThreshold xxx] where xxx is a floating point number between 0 
           and 1 specifying the threshold used to prune small entries
           in setting up the sparse approximate inverse.  The default
           value is 0.0.
\item[parasailsNlevels xxx] where xxx is an integer larger than 0 specifying 
           the desired sparsity of the approximate inverse.  The
           default value is 1.
\item[parasailsFilter xxx] where xxx is a floating point number between 0 
           and 1 specifying the threshold used to prune small entries
           in $A$.  The default value is 0.0.
\item[parasailsLoadbal xxx] where xxx is a floating point number between 0 
           and 1 specifying how load balancing has to be done 
           (Edmond, explain please).  The default value is 0.0.
\item[parasailsSymmetric] sets Parasails to take $A$ as symmetric.
\item[parasailsUnSymmetric] sets Parasails to take $A$ as nonsymmetric
                            (default).
\item[parasailsReuse] sets Parasails to reuse the sparsity pattern of $A$.
\item[polyorder xxx] where xxx is the order of the least-squares polynomial 
           preconditioner.
\end{description}

\subsection{Parameters for Multilevel Preconditioners}

\subsubsection{Parameters for Boomeramg Preconditioner}
\begin{description}
\item[amgCoarsenType xxx] where xxx specifies one of {\sf falgout} or
           {\sf ruge}, or {\sf default (CLJP)} coarsening for BOOMERAMG.
\item[amgMeasureType xxx] where xxx specifies one of {\sf local} or
           or {\sf global}.  This parameter affects how coarsening is performed
           in parallel.
\item[amgNumSweeps xxx] where xxx is an integer specifying the number of
           pre- and post-smoothing at each level of BOOMERAMG.
           The default is two pre- and two post-smoothings.
\item[amgRelaxType xxx] where xxx is one of {\sf jacobi} (Damped Jacobi),
           {\sf gs-slow} (sequential Gauss-Seidel), {\sf gs-fast}
           (Gauss-Seidel on interior nodes), or {\sf hybrid}.
           The default is {\sf hybrid}.
\item[amgRelaxWeight xxx] where xxx is a floating point number between 
           0 and 1 specifying the damping factor for BOOMERAMG's damped
           Jacobi and GS smoothers.  The default value is 1.0.
\item[amgRelaxOmega xxx] where xxx is a floating point number between 
           0 and 1 specifying the damping factor for BOOMERAMG's hybrid
           smoother for multiple processors.  The default value is 1.0.
\item[amgStrongThreshold xxx] where xxx is a floating point number between 0 
           and 1 specifying the threshold used to determine
           strong coupling in BOOMERAMG's coasening.  The default 
           value is 0.25.
\item[amgSystemSize xxx] where xxx is the degree of freedom per node.
\item[amgUseGSMG] - tells BOOMERAMG to use a different coarsening called GSMG.
\item[amgGSMGNumSamples] where xxx is the number of samples to generate
           to determine how to coarsen for GSMG.
\end{description}

\subsubsection{Parameters for MLI Preconditioner}
\begin{description}
\item[outputLevel xxx] where xxx is the output level for diagnostics.
\item[numLevels xxx] where xxx is the maximum number of levels used.
\item[maxIterations xxx] where xxx is the maximum number of iterations.
\item[cycleType xxx] where xxx is either 'V' or 'W' (V or W cycle).
\item[strengthThreshold xxx] strength threshold for coarsening.
\item[method xxx] where xxx is either {\sf AMGSA} (default), {\sf AMGSAe},
     {\sf AMGSADD}, or {\sf AMGSADDe} to indicate which MLI method is to be
     used.
\item[smoother xxx] where xxx is either {\sf Jacobi}, {\sf BJacobi}, {\sf GS}, 
     {\sf SGS} (default), {\bf BSGS}, {\sf ParaSails}, {\sf MLS}, 
     {\sf CGJacobi}, {\sf CGBJacobi}, or {\sf Chebyshev}.
\item[numSweeps xxx] where xxx is the number of smoother sweeps (default = 2).
\item[coarseSolver xxx] where xxx is one of those in 'smoother' or
     {\sf SuperLU}.
\item[minCoarseSize xxx] where xxx is the minimum coarse grid size to
     control the number of levels used.
\item[Pweight xxx] where xxx is the relaxation parameter for the prolongation
     smoother (default 0.0).
\item[nodeDOF xxx] where xxx is the degree of freedom for each node.
\item[nullSpaceDim xxx] where xxx is the dimension of the null space for
     the coarse grid.
\item[useNodalCoord xxx] where xxx is either 'on' or 'off' to indicate whether
     the nodal coordinates are used to generate the initial null space.
\item[saAMGCalibrationSize xxx] where xxx is the additional null space 
     vectors to be generated via calibration.
\end{description}

\subsection{Parameters for Block and Uzawa Preconditioners}
\subsubsection{Parameters for Block Preconditioners}
The parameters for this preconditioner are preceded by the keyword {sf blockP}.
The available parameters after this keywords are:
\begin{description}
\item[blockD] turns on block diagonal preconditioning.
\item[blockT] turns on block tridiagonal preconditioning.
\item[blockLU] turns on block LU preconditioning.
\item[outputLevel xxx] where xxx is the output level for diagnostics.
\item[block1FieldID xxx] where xxx is field ID for the (1,1) block.
\item[block2FieldID xxx] where xxx is field ID for the (2,2) block (for
    ALE3D's implicit hydrodynamics with slide surfaces, the field ID for
    both blocks are -3.)
\item[printInfo] prints information about internal parameter settings.
\item[lumpedMassScheme xxx] where is either {\sf diag} (take the diagonal
     of the (1,1) block) or {\sf ainv} (take the approximate inverse of
     the (1,1) block.
\item[invA11PSNlevels xxx] where xxx is 0 or 1 to indicate the ParaSails
     nlevels to use to generate the lumped mass matrix (if {\sf ainv} is
     selected.)
\item[invA11PSThresh xxx] where xxx is a floating point number between 0 
     and 1 to indicate the ParaSails threshold to use to generate the lumped 
     mass matrix (if {\sf ainv} is selected.)
\item[A11Solver xxx] where xxx is either {\sf cg} or {\sf gmres} as solver
     for the (1,1) block.
\item[A11Tolerance xxx] where xxx is convergence tolerance for the
     (1,1) block.
\item[A11MaxIterations xxx] where xxx is maximum number of iterations for
     the (1,1) block.
\item[A11Precon xxx] where xxx is either {\sf pilut}, {\sf boomeramg}, 
     {\sf euclid}, {\sf parasails}, {\sf ddilut}, or {\sf mli}.
\item[A11PreconPSNlevels xxx] - ParaSails' nlevels.
\item[A11PreconPSThresh xxx] - ParaSails' threshold. 
\item[A11PreconPSFilter xxx] - ParaSails' filter.
\item[A11PreconAMGThresh xxx] - Boomeramg's threshold.
\item[A11PreconAMGRelaxType xxx] - Boomeramg's smoother.
\item[A11PreconAMGNumSweeps xxx] - Boomeramg's numSweeps.
\item[A11PreconAMGSystemSize xxx] - Boomeramg's systemSize.
\item[A11PreconEuclidNLevels xxx] - Euclid's nlevels.
\item[A11PreconEuclidThresh xxx] - Euclid's threshold.
\item[A11PreconPilutFillin xxx] - Pilut's fillin.
\item[A11PreconPilutDropTol xxx] - Pilut's drop tolerance.
\item[A11PreconDDIlutFillin xxx] - DDILUT's fillin.
\item[A11PreconDDIlutDropTol xxx] - DDILUT's drop tolerance.
\item[A11PreconMLIRelaxType xxx] - MLI's smoother.
\item[A11PreconMLIThresh xxx] - MLI's threshold.
\item[A11PreconMLIPweight xxx] - MLI's Pweight.
\item[A11PreconMLINumSweeps xxx] - MLI's numSweeps.
\item[A11PreconMLINodeDOF xxx] - MLI's nodeDOF.
\item[A11PreconMLINullDim xxx] - MLI's null space dimension.
\item[A22Solver xxx] where xxx is either {\sf cg} or {\sf gmres} as solver
     for the (2,2) block.
\item[A22Tolerance xxx] where xxx is convergence tolerance for the
     (2,2) block.
\item[A22MaxIterations xxx] where xxx is maximum number of iterations for
     the (2,2) block.
\item[A22Precon xxx] where xxx is either {\sf pilut}, {\sf boomeramg}, 
     {\sf euclid}, {\sf parasails}, {\sf ddilut}, or {\sf mli}.
\item[A22PreconPSNlevels xxx] - ParaSails nlevels.
\item[A22PreconPSThresh xxx] - ParaSails' threshold.
\item[A22PreconPSFilter xxx] - ParaSails' filter.
\item[A22PreconAMGThresh xxx] - Boomeramg's threshold.
\item[A22PreconAMGRelaxType xxx] - Boomeramg's smoother.
\item[A22PreconAMGNumSweeps xxx] - Boomeramg's numSweeps.
\item[A22PreconAMGSystemSize xxx] - Boomeramg's systemSize.
\item[A22PreconEuclidNLevels xxx] - Euclid's nlevels.
\item[A22PreconEuclidThresh xxx] - Euclid's threshold.
\item[A22PreconPilutFillin xxx] - Pilut's fillin.
\item[A22PreconPilutDropTol xxx] - Pilut's drop tolerance.
\item[A22PreconDDIlutFillin xxx] - DDILUT's fillin.
\item[A22PreconDDIlutDropTol xxx] - DDILUT's drop tolerance.
\item[A22PreconMLIRelaxType xxx] - MLI's smoother.
\item[A22PreconMLIThresh xxx] - MLI's threshold.
\item[A22PreconMLIPweight xxx] - MLI's Pweight.
\item[A22PreconMLINumSweeps xxx] - MLI's numSweeps.
\item[A22PreconMLINodeDOF xxx] - MLI's nodeDOF.
\item[A22PreconMLINullDim xxx] - MLI's null space dimension.
\end{description}

\subsubsection{Parameters for Uzawa Preconditioner}
The Uzawa preconditioner has a similar parameter set as block preconditioner,
as described in the following (except that DDIlut is not available here).
\begin{description}
\item[outputLevel xxx] -  where xxx is the output level for diagnostics.
\item[A11Solver xxx] where xxx is either {\sf cg} or {\sf gmres} as solver
     for the (1,1) block.
\item[A11Tolerance xxx] where xxx is convergence tolerance for the
     (1,1) block.
\item[A11MaxIterations xxx] where xxx is maximum number of iterations for
     the (1,1) block.
\item[A11Precon xxx] where xxx is either {\sf pilut}, {\sf boomeramg}, 
     {\sf euclid}, {\sf parasails}, {\sf ddilut}, or {\sf mli}.
\item[A11PreconPSNlevels xxx] - ParaSails' nlevels.
\item[A11PreconPSThresh xxx] - ParaSails' threshold.
\item[A11PreconPSFilter xxx] - ParaSails' filter.
\item[A11PreconAMGThresh xxx] - Boomeramg's threshold.
\item[A11PreconAMGRelaxType xxx] - Boomeramg's smoother.
\item[A11PreconAMGNumSweeps xxx] - Boomeramg's numSweeps.
\item[A11PreconAMGSystemSize xxx] - Boomeramg's systemSize.
\item[A11PreconEuclidNLevels xxx] - Euclid's nlevels.
\item[A11PreconEuclidThresh xxx] - Euclid's threshold.
\item[A11PreconPilutFillin xxx] - Pilut's fillin.
\item[A11PreconPilutDropTol xxx] - Pilut's drop tolerance.
\item[A11PreconMLIRelaxType xxx] - MLI's smoother.
\item[A11PreconMLIThresh xxx] - MLI's threshold.
\item[A11PreconMLIPweight xxx] - MLI's Pweight.
\item[A11PreconMLINumSweeps xxx] - MLI's numSweeps.
\item[A11PreconMLINodeDOF xxx] - MLI's nodeDOF.
\item[A11PreconMLINullDim xxx] - MLI's null space dimension.
\item[S22SolverDampingFactor xxx] where xxx is the damping (scaling) factor
     for the Schur complement approximation of the (2,2) block.
\item[S22Solver xxx] where xxx is either {\sf cg} or {\sf gmres} as solver
     for the (2,2) block.
\item[S22Tolerance xxx] where xxx is convergence tolerance for the
     (2,2) block.
\item[S22MaxIterations xxx] where xxx is maximum number of iterations for
     the (2,2) block.
\item[S22Precon xxx] where xxx is either {\sf pilut}, {\sf boomeramg}, 
     {\sf euclid}, {\sf parasails}, {\sf ddilut}, or {\sf mli}.
\item[S22PreconPSNlevels xxx] - ParaSails' nlevels.
\item[S22PreconPSThresh xxx] - ParaSails' threshold.
\item[S22PreconPSFilter xxx] - ParaSails' filter.
\item[S22PreconAMGThresh xxx] - Boomeramg's threshold.
\item[S22PreconAMGRelaxType xxx] - Boomeramg's smoother.
\item[S22PreconAMGNumSweeps xxx] - Boomeramg's numSweeps.
\item[S22PreconAMGSystemSize xxx] - Boomeramg's systemSize.
\item[S22PreconEuclidNLevels xxx] - Euclid's nlevels.
\item[S22PreconEuclidThresh xxx] - Euclid's threshold.
\item[S22PreconPilutFillin xxx] - Pilut's fillin.
\item[S22PreconPilutDropTol xxx] - Pilut's drop tolerance.
\item[S22PreconMLIRelaxType xxx] - MLI's smoother.
\item[S22PreconMLIThresh xxx] - MLI's threshold.
\item[S22PreconMLIPweight xxx] - MLI's Pweight.
\item[S22PreconMLINumSweeps xxx] - MLI's numSweeps.
\item[S22PreconMLINodeDOF xxx] - MLI's nodeDOF.
\item[S22PreconMLINullDim xxx] - MLI's null space dimension.
\end{description}

\subsection{Parameters for Matrix Reduction}
\begin{description}
\item[schurReduction] turns on the Schur reduction mode.
\item[slideReduction] turns on the slide reduction mode.
\item[slideReduction2] turns on the slide reduction mode version 2 
(see section 2).
\item[slideReduction3] turns on the slide reduction mode version 3 
(see section 2).
\end{description}

\subsection{Parameters for Diagnostics and Performance Tuning}
\begin{description}
\item[outputLevel xxx] where xxx is an integer specifying the output
           level.  An output level of $1$ prints only the solver 
           information such as number of iterations and timings.
           An output level of $2$ prints debug information such as
           the functions visited and preconditioner information.
           An output level of $3$ or higher prints more debug information 
           such as the matrix and right hand side loaded via the 
           LinearSystemCore functions to the standard output.
\item[setDebug xxx] where xxx is one of {\sf slideReduction1}, 
           {\sf slideReduction2},
           {\sf slideReduction3} (level 1,2,3 diagnostics in the slide surface
           reduction code), {\sf printMat} (print the original matrix
           into a file), {\sf printReducedMat} (print the reduced matrix
           into a file),  {\sf printSol} (print the solution into a file), 
           {\sf ddilut} (output diagnostic information for DDIlut
           preconditioner setup), and {\sf amgDebug} (output diagnostic 
           information for AMG).
\item[optimizeMemory] cleans up the matrix sparsity pattern after the matrix
           has been loaded. (It has been kept to allow matrix reuse.)
\end{description}

\subsection{Miscellaneous Parameters}
\begin{description}
\item[AConjugateProjection xxx] where xxx specifies the number of previous
           solution vectors to keep for the A-conjugate projection. 
           The default is 0 (the projection is off).
\item[minResProjection xxx] where xxx specifies the number of previous
           solution vectors to keep for projection. 
           The default is 0 (the projection is off).
\item[haveFEData] indicates that additional finite element information are 
           available to assist in building more efficient solvers. 
\end{description}

\section{An Example}

In this section we show how to use {\sf HYPRE\_LinSysCore} as a standalone
solver (that is, not via the FEI).  Suppose the matrix partitioning information 
(which rows belong to which processor) has been determined and is available
in the {\sf eqnOffsets} array.  We further suppose that the local submatrix is
available as a compressed sparse row (CSR) matrix in the {\sf ia, ja, val} arrays. 

\begin{tabbing}
\hspace{0.5in} \= {\sf Program Segment} \\[1mm]
\> {\sf startRow = eqnOffsets[mypid];} \\
\> {\sf endRow = eqnOffsets[mypid+1] - 1;} \\
\> {\sf nrows = endRow - startRow + 1} \\
\> {\sf for ( i = startRow; i <= endRow; i++ ) \{ } \\
\> \hspace{0.3in} \= {\sf ncnt = ia[i+1] - ia[i];} \\
\> \> {\sf rowLengths[i-startRow] = ncnt;} \\
\> \> {\sf colIndices[i-startRow] = new int[ncnt];} \\
\> \> {\sf k = 0;} \\
\> \> {\sf for (j = ia[i]; j < ia[i+1]; j++) colIndices[i-startRow][k++] = ja[j];}\\
\> \} \\
\> {\sf HYPRE\_LinSysCore\_create(\&lsc, MPI\_COMM\_WORLD);} \\
\> {\sf HYPRE\_setGlobalOffsets(lsc, nrows, NULL, eqnOffsets, NULL);} \\
\> {\sf HYPRE\_setMatrixStructure(lsc, colIndices, rowLengths, NULL, NULL, NULL);} \\
\> {\sf for ( i = startRow; i <= endRow; i++ ) \{ } \\
\> \> {\sf ncnt = ia[i+1] - ia[i];} \\
\> \> {\sf HYPRE\_sumIntoSystemMatrix(lsc, i, ncnt, \&val[ia[i]], \&ja[ia[i]]);}\\
\> \> {\sf HYPRE\_sumIntoRHSVector(1, \&rhs[i], \&i);} \\
\> \} \\
\> {\sf HYPRE\_matrixLoadComplete();}\\
\> {\sf strcpy(paramString, "solver gmres");} \\
\> {\sf HYPRE\_parameters(1, \&paramString);} \\
\> {\sf strcpy(paramString, "preconditioner boomeramg");} \\
\> {\sf HYPRE\_parameters(1, \&paramString);} \\
\> {\sf HYPRE\_launchSolver(\&status, \&iterations);}
\end{tabbing}

\section{HYPRE LinearSystemCore Installation}

The ultimate objective is for application users to have immediate access
to the latest FEI/HYPRE library files on different computing platforms
via public {\sf lib} directories.  While this feature is forthcoming, careful 
version control is needed for users to keep track of capabilities and bug fixes 
for different installations.  Users who would like to set up the FEI/HYPRE
on their own should do the following :

\begin{enumerate}

\item obtain \hypre{} and the Sandia FEI source codes,
\item compile {\sf FEI} (fei-2.5.0) to create the {\sf libfei\_base.a} file.
\item compile \hypre{} 
\begin{enumerate}
\item go into the {\sf linear\_solvers} directory
\item do a 'configure' with the {\sf --with-FEI\_BASE\_DIR} option set to
      the {\sf FEI} include directory plus other compile options
\item go into {\sf linear\_solvers/FEI\_mv/fei-hypre} and comment out one
      {\sf FEI\_BASE\_DIR} define and uncomment the other that points to
      your desired {\sf FEI} include directory (This step will be taken
      out once the configure script is modified).
\item compile with {\sf make fei} which will create the {\sf libHYPRE\_LSI.a}
      file in the {\sf linear\_solvers/hypre/lib} directory.
\end{enumerate}
\item use the {\sf FEI} in your application code
\begin{enumerate}
\item include {\sf cfei\-hypre.h} in your file 
\item include {\sf fei.h} in your file 
\item make sure your application has an {\sf include} and an {\sf lib} path 
      to the {\sf include} and {\sf lib} directories created above. 
\end{enumerate}

\end{enumerate}

\subsection{Linking with the library files}

To link FEI and \hypre{} into the executable, the following has to be
attached to the linking command :

\begin{tabbing}
\hspace{0.5in} \= {\sf -L\$\{LIBPATHS\} -lfei\_base -lHYPRE\_LSI} 
\end{tabbing}
along with all the other libraries (Note : the order in which the libraries are
listed may be important), where {\sf LIBPATHS} are where 
the \hypre{} and FEI libraray files can be found.  

Since some of these library files make calls to LAPACK and BLAS functions, 
the corresponding libraries need to be linked along with (placed after) these 
library files.  For example, on the DEC cluster, it suffices to link
with the {\sf dxml} library, (So {\sf -ldxml} is placed after the above link
sequence, with {\sf -lm} placed after {\sf -ldxml}.) while the {\it essl}
library can be used on the blue machine.

\subsection{Some more caveats for application developers}

Building an application executable often requires linking with many different
software packages, and many software packages use some LAPACK and/or BLAS
functions.  In order to alleviate the problem of multiply defined functions
at link time, it is recommended that all software libraries are stripped of
all LAPACK and BLAS function definitions.  These LAPACK and BLAS functions 
should then be resolved at link time by linking with the system LAPACK and
BLAS libraries (e.g. dxml on DEC cluster).  Both \hypre{} and SuperLU were
built with this in mind.  However, some other software library files needed
may have the BLAS functions defined in them (notably libchaco.a
in /usr/apps/bdivport/lib).  To avoid the problem of multiply defined 
functions, it is recommended that the libchaco.a file be stripped of the 
BLAS functions.
Stripping can be achieved by calling the following Unix command

\begin{tabbing}
\hspace{0.5in} \= {\sf ar -d libchaco.a symmlqblas.o}
\end{tabbing}

\section{Comments about the FEI/HYPRE Interface and Contacts}

Comments about the FEI/HYPRE interface can be directed to Charles Tong
(X23411, chtong@llnl.gov) or Edmond Chow (X31915, chow8@llnl.gov).

