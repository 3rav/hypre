%==========================================================================
\def\pilut{{\sl PILUT}}
\section{\pilut: Parallel Incomplete Factorization}
\label{PILUT}

\pilut{} is a parallel preconditioner based on Saad's dual-threshold incomplete
factorization algorithm. The original version of \pilut{} was done by
Karypis and Kumar \ref{} in terms of the Cray SHMEM library. The code
was subsequently modified by the \hypre{} team: SHMEM was replaced by
MPI; some algorithmic changes were made; and it was software
engineered to be interoperable with several matrix implementations,
including \hypre{}'s ParCSR format, PETSc's matrices, and ISIS++
RowMatrix. The algorithm produces an approximate factorization $ L U$,
with the preconditioner $M$ defined by $ M = L U $.

{\bf Note:} \pilut{} produces a nonsymmetric preconditioner even when the
original matrix is symmetric. Thus, it is generally inappropriate for
preconditioning symmetric methods such as Conjugate Gradient.

\subsection*{Parameters:}

\begin{itemize}

\item
\code{SetMaxNonzerosPerRow( int LFIL ); (Default: 20)}
Set the maximum number of nonzeros to be retained in each row of $L$ and $U$.
This parameter can be used to control the amount of memory that $L$ and $U$
occupy. Generally, the larger the value of \code{LFIL}, the longer it takes to
calculate the preconditioner and to apply the preconditioner and the larger
the storage requirements, but this trades
off versus a higher quality preconditioner that reduces the number of
iterations.

\item
\code{SetDropTolerance( double tol ); (Default: 0.0001)}
Set the tolerance (relative to the 2-norm of the row) below which entries in L
and U are automatically dropped. \pilut{} first drops entries based on the drop
tolerance, and then retains the largest LFIL elements in each row that remain.
Smaller values of \code{tol} lead to more accurate preconditioners, but can
also lead to increases in the time to calculate the preconditioner.

\end{itemize}
