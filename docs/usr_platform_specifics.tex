\section{Platform Specific Information}
\begin{description}

\item[IBM Blue Pacific]
obtaining code:
\newline
Hypre is installed in /usr/apps/hypre.  This copy is a pure MPI code with optimization turned on.  To obtain source code please contact Rob Falgout or Andy Cleary.
\newline

configure defaults:
\newline
the default C compiler is mpcc.  Without MPI it is xlc, with pthreads it is mpxlc\_r, and with openmp it is mpguidec.  Default flags are -O3 -qstrict -qmaxmem=8192.  By default, MPI derived data types are turned off.      
\newline

configure options:
\newline
Please see configure --help for details.  Most options are available on any platform.  For those then are not the following are available on Blue:  --with-pthreads --with-openmp --without-MPI --with-HYPRE\_blas --without-COMM\_SIMPLE 
\newline


\item[Suns--CASC Cluster]
obtaining code:
\newline
Two versions of Hypre are installed on the CASC cluster.  They are both in /home/casc/software/hypre.  The internal version is updated frequently and is used to beta test the code before full release.  The external version os the same version that is available on other platforms-- it will never be a newer copy than the internal version.  Both copies are pure MPI with optimization turned on. To obtain source code please contact Rob Falgout or Andy Cleary.
\newline

configure defaults:
\newline
The default C compiler is mpicc.  Without MPI it is cc.  Optimization is turned on by default.  The -silent flag is used with Fortran compilers.  
\newline

configure options:
\newline
Please see configure --help for details.  Most options are available on any platform.  For those then are not the following are available on Suns:  --without-MPI --with-HYPRE\_blas --with-purify --with-cegdb --with-gdb
\newline

\item[DECS]
obtaining code:
\newline
Hypre is installed in /usr/apps/hypre.  This copy is a pure MPI code with optimization turned on. To obtain source code please contact Rob Falgout or Andy Cleary.
\newline

configure defaults:
\newline
The default C compiler is mpicc.  Without MPI it is cc, with openmp it is guidec.  If mpicc is NOT used, then the mpi library to link to is found in /usr/opt/MPI170/lib; the rt library in /usr/lib is also linked to.  If the kcc compiler is used, the -o5 optimization flag is used by default, otherwise -0 is used by default.   
\newline

configure options:
\newline
Please see configure --help for details.  Most options are available on any platform.  For those then are not the following are available on DECS:  --with-openmp --without-MPI 
\newline

\end{description}


