%==========================================================================
\chapter{Building the Code}
\label{Building the Code}

To automatically generate machine specific makefiles, type
\kbd{configure} in the \file{linear_solvers} directory.  The configure
script is a portable script generated by GNU Autoconf.  It runs a
series of tests to determine characteristics of the machine on which
it is running, and it uses the results of the these tests to produce
the machine specific makefiles, called `Makefile', from template files
called `Makefile.in' in each directory.  Once the makefiles are
produced you can run make as you would with any other makefile.

This configure script primarily does the following things:
\begin{itemize}
\item selects a compiler
\item provides either optimization or debugging options for the compiler
\item finds the headers and libraries for MPI
\end{itemize}

The configure script has some command-line options that can give you
some control over the choices it will make.  You can type
\begin{verbatim}
   configure --help
\end{verbatim}
to see the list of all of the command-line options to configure. This is
the best resource for information on configure options.  Below is some
additional helpful information.


\begin{description}

\item[Compilers] If you want to choose a compiler then is it recommended
that you choose all (C, C++, Fortran) compilers.

\item[Compiler Flags] To choose optimization or debug, use --enable-opt
or --enable-debug.  (Optimization is the default)  For other compiler 
flags use the --with-CFLAGS option.  

\item

\end{description}


Configure automatically generates a file 
\begin{verbatim} HYPRE_config.h \end{verbatim}that includes all the
header files found to be neceaary by configure.  This file may be used
to see how a compiled version of the library was configured and may also
be included by the user in his own code.
