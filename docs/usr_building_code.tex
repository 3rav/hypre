%==========================================================================
\chapter{Building the Code}
\label{Building the Code}

To automatically generate machine specific makefiles, type
\kbd{configure} in the \file{linear_solvers} directory.  The configure
script is a portable script generated by GNU Autoconf.  It runs a
series of tests to determine characteristics of the machine on which
it is running, and it uses the results of the these tests to produce
the machine specific makefiles, called `Makefile', from template files
called `Makefile.in' in each directory.  Once the makefiles are
produced you can run make as you would with any other makefile.

The configure script produces a file called `config.cache' which
stores some of its results.  If you wish to run configure again in a
way that will get different results, you should remove `config.cache'.

This configure script primarily does the following things:
\begin{itemize}
\item selects a C compiler
\item provides either optimization or debugging options for the C compiler
\item finds the headers and libraries for MPI
\end{itemize}

The configure script has some command-line options that can give you
some control over the choices it will make.  You can type
\begin{verbatim}
   configure --help
\end{verbatim}
to see the list of all of the command-line options to configure, but
the most significant options are at the bottom of the list, after the
line that reads
\begin{verbatim}
   --enable and --with options recognized:
\end{verbatim}

\begin{description}

\item[--with-CC=ARG] This option allows you to choose the C compiler
you wish to use.  The default compiler that configure chooses is gcc,
if it is available.

\item[--enable-opt-debug=ARG] Choose whether you want the C compiler
to have optimization or debugging flags.  For debugging, replace
\kbd{ARG} with \kbd{debug}.  For optimization, replace \kbd{ARG} with
\kbd{opt}.  If you want both sets of flags, replace \kbd{ARG} with
\kbd{both}.  The default is optimization

\item[--without-MPI] This flag suppresses the use of MPI.

\item[--with-mpi-include=DIR]
\item[--with-mpi-libs=LIBS]
\item[--with-mpi-lib-dirs=DIRS] These three flags are to be used if
you want to override the automatic search for MPI.  If you use one of
these flags, you must use all three.  Replace \kbd{DIR} with the path
of the directory that contains \file{mpi.h}, replace \kbd{LIBS} with a
list of the stub names of all the libraries needed for MPI, and
replace \kbd{DIRS} with a list of the directory paths containing the
libraries specified by \kbd{LIBS}.  NOTE: The lists \kbd{LIBS} and
\kbd{DIRS} should be space-separated and contained in quotes, e.g.
\begin{verbatim}
   --with-mpi-libs="nslsocket mpi"
   --with-mpi-lib-dirs="/usr/lib /usr/local/mpi/lib"
\end{verbatim}

\item[--with-mpi-flags=FLAGS] Sometimes other compiler flags are
needed for certain MPI implementations to work.  Replace \kbd{FLAGS}
with a space-separated list of whatever flags are necessary.  This
option does not override the automatic search for MPI.  It can be used
to add to the results of the automatic search, or it can be used along
with the three previous flags.

\item[--with-MPICC=ARG] The automatic search for the MPI stuff is
based on a tool such as \file{mpicc} that configure finds in your
\code{\$PATH}.  If there is an implementation of MPI that you wish to
use, you can replace \kbd{ARG} with the name of that MPI version's C
compiler wrapper, if it has one.  (MPICH has \file{mpicc}, IBM MPI has
\file{mpcc}, other MPIs use other names.)  configure will then
automatically find the necessary libraries and headers.

--with-pthreads

Thos option should be used is POSIX threads are to be used.  The default
is to use MPI only without POSIX threads.

\end{description}

