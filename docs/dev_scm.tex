%==========================================================================
\chapter{Configuration Management}
\label{Configuration Management}

This chapter describes the so-called Software Configuration Management
(SCM) practices used on the \hypre{} project.  Good SCM practices can
enhance the efficiency of software development and maintenance efforts
by providing mechanisms for tracking and controlling the evolution of
a software package.  Whether or not the \hypre{} project uses ``good''
SCM practices is debatable, but they are described here nonetheless.

%==========================================================================
\section{The Repository}
\label{The Repository}

CVS is used as the resource control mechanism for \hypre{}.  This
allows developers to more easily coordinate their efforts, keep
previous revisions of software, and track changes and releases.

Some CVS commands you will use most often are:
\begin{itemize}
\item '\code{cvs checkout linear\_solvers}' - checks out the most
recent version of the repository source into the current directory.
\item '\code{cvs update}' - updates your working version, merging
(or attempting to merge) your file changes with changes made (and
checked in) by other developers.  This will also give information on
current status of your changes, such as which files have been modified
(\code{M}), which files have been added (\code{A}), which files have
been removed (\code{R}), and which files cannot be accounted for
(\code{?}).
\item '\code{cvs commit}' - commits changes to all files in current
directory and all subdirectories.
\item '\code{cvs log <file>}' - prints revision history for a file.
\end{itemize}

It is recommended that developers set up a \file{.cvsrc} file in their
home directory with the following lines in it:
\begin{verbatim}
checkout -P
update -P -d
remove -f
\end{verbatim}
This sets default options for the CVS commands, \code{checkout},
\code{update}, and \code{remove} that make life much easier.
\begin{itemize}
\item The \code{-P} options cause the \code{checkout} and \code{update}
commands to ``prune'' (remove) directories that are empty.

\item The \code{-d} option will create any directories that exist in the
repository if they are missing in your working directory.  So, when
new directories are added to the repository, they will get added to
your working copy automatically when an \code{update} is done.

\item The \code{-f} option causes the remove command to first remove the
file from your working directory, then schedule it for removal from
the repository.  So, if you want to remove a file from the repository,
all you need to do is \code{cvs remove <file>}.
\end{itemize}

The main trunk of the repository must always pass the autotest
procedures outlined in Section \ref{Autotest Procedures}.  So,
developers must take extra care that checked-in code does not break
autotest.  One thing that should almost always be done is to do a
clean checkout into a temporary subdirectory (e.g., \code{~/tmp}),
then do '\code{configure}' and '\code{make}'.  This will at least
verify that the compile is not broken.  Extra measures should be
taken when necessary to insure that the autotest runs also do not
fail.

%==============================
\subsection{Using CVS Branches}
\label{Using CVS Branches}

Sometimes a large change needs to be made to the repository that
involves several developers coordinating their efforts.  Here, it
would be useful to use CVS to share code during the development
process, but this may temporarily break code that previously worked.
To avoid this, developers can use the branch mechanism in CVS.

{\bf WARNING:} The use of branches can easily screw up the repository
if care is not taken.  Developers should follow the guidelines below
to reduce the risk of this.  Do not stray from these guidelines
without first consulting the software technical lead for \hypre{}.

{\bf SECOND WARNING:} Branches should not be created and kept around
for long periods of time.  They should only be used {\em temporarily}.

\begin{itemize}

\item First, coordinate with the other developers involved to decide
on a branch name, and to insure that it is okay to start development
from the latest revision of the repository's main trunk.

\item Create a new branch (we will use the name '\code{foo}' here).
\begin{verbatim}
cvs rtag -b foo linear_solvers
\end{verbatim}

\item Create a working copy of the branch.
\begin{verbatim}
cd ~
mkdir foo
cd foo
cvs checkout -r foo linear_solvers
\end{verbatim}

\item Make modifications to files in '\code{~/foo/linear\_solvers}'.
Code can be checked in/out without doing damage to the main
development trunk.

\item When the code is working, merge changes into the main trunk.
\begin{verbatim}
cd ~
cvs checkout linear_solvers
cvs update -j foo
\end{verbatim}

\item Verify that the merged files are okay, i.e. that the code compiles,
runs, etc.

\item Commit the changes to the main trunk.
\begin{verbatim}
cd ~/linear_solvers
cvs commit .
\end{verbatim}

\item Verify that the commit worked okay by doing a clean checkout, etc.

\item Remove the temporary tag, '\code{foo}', so that this branch is
not accidentally used to make additional changes to the repository.
\begin{verbatim}
cvs rtag -d foo linear_solvers
\end{verbatim}

\end{itemize}

%==========================================================================
\subsection{Making Changes to the Repository}
\label{Making Changes to the Repository}

This section gives guidelines for making changes to the repository.
Because \hypre{} is a fairly large software library with a number of
important customers, formal change control guidelines are necessary in
order to maintain software quality and preserve the sanity of library
developers (those who are not already lost, anyway :).

There are different classes of code changes that can be commited to
the repository, some with little or no prerequisites attached, and
others with heavier prerequisites.
\begin{itemize}

\item {\bf All code} should at least compile before commiting
to the repository.

\item {\bf Code that is not being distributed with a release}
may be changed without consulting with the \hypre{} technical lead.

\item {\bf Code being distributed with a release that is not officially
part of the release} may be changed without consulting with the
\hypre{} technical lead.  This code should compile on all platforms
and pass all autotesting.

\item {\bf Code that is part of a release} may be changed without
consulting the \hypre{} technical lead if the changes will have no
noticeable effect on users.  All user interface changes or
modifications that change the behavior of the library require
coordination with the \hypre{} technical lead.  This code should
compile on all platforms, have sufficient user documentation, and pass
all autotesting.

\end{itemize}


%==============================
\subsection{Tracking Repository Changes}
\label{Tracking Repository Changes}

Changes to the repository are tracked primarily by CVS, which keeps a
chronological log of all changes in the following file.
\begin{verbatim}
/home/casc/repository/updatelog.linear_solvers
\end{verbatim}

CVS tags are used to ``tag'' releases with their corresponding release
number (see Section \ref{Releases}).  For general releases, a new CVS
branch is also created, and bug fixes are made to that branch.  CVS
provides commands for viewing the tags and branches in the repository.

Major changes that need to be reported to users are manually entered
in the \file{CHANGELOG} file in the top-level directory.  This file is
distributed with each \hypre{} release.

Finally, the \hypre{} issue tracking facility, Roundup, provides
the ability to track the status of issues and new features/changes.
Users submit their reports/requests by sending an email to 
\file{hypre-support@llnl.gov}.


%==========================================================================
\section{Releases}
\label{Releases}

The term \hypre{} {\em release} refers to the collection of software
and documentation distributed and supported for our users.  Only a
subset of the repository is in the \hypre{} release at any point in
time.  The files distributed in a release are defined via the Korn
shell script \file{tools/mkdist} used to create the tar file for that
version.  Note that there will be times when pieces of software are 
distributed with the release, but are technically not yet {\em in} 
the release (i.e., they are not yet fully supported).  This is done 
to ease the process of migrating a code from a development stage to 
a production stage.

The \hypre{} technical lead determines when new software is added to
the release.  The following criteria must be met before a code is
added to the release:
\begin{enumerate}

\item the developer has done fairly rigorous testing of code
\item an autotest suite has been set up and the code passes
\item the code conforms to coding guidelines
\item the code uses autoconfig infrastructure and configures on all
supported platforms
\item more than one developer has looked at the code
\item the code has been documented in the User's and Reference Manuals
\item runs through purify cleanly

\end{enumerate}

Two mailing list are available to inform users of new HYPRE general and beta
releases being available.  Announcements about general releases made to the
`hypre-announce' mailing list, and announcements about beta releases are made
to the `hypre-beta-announce' list. Since the alpha releases are only done on
the CASC platforms there is no mailing list for those releases.

These mailing lists are handled through the LLNL Majordomo list server,
Majordomo@lists.llnl.gov. To add yourself to a mailing list, send mail to
<Majordomo@list.llnl.gov> with the following command in the body of the message:

subscribe hypre-announce
        or
subscribe hypre-beta-announce


%==============================
\subsection{Release Numbers}
\label{Release Numbers}

By convention, a release (or version) number is of the form M.mm.rr.
M is the major release number, mm the minor relase number and rr is an
update number.  This number is used when creating a symbolic tag in the 
repository to identify the release. 

The release number will have an 'a' appended for an alpha release or a 'b'
for a beta release.

%==============================
\subsection{Types of Releases}
\label{Types of Releases}

\noindent
We currently have 3 flavors of releases:
\begin{itemize}
\item {\bf Version M.mm.rr} - This is our {\em general release}.  These
releases occur infrequently and are available for download from the
web.  Only bug fixes will be done to these releases (i.e., no new
technology), indicated by an incremented 3rd digit in the version
number.  A new CVS branch is created and tagged to allow bug fixes
without introducing new technology.

\item {\bf Version M.mm.rrb} - These {\em beta releases} will occur somewhat
frequently.  Bug fixes will only be available here from new releases,
which may also include new technology.  Beta releases are not expected
to be as stable as general releases.  The release is tagged, but a CVS
branch is not created.

\item {\bf Version M.mm.rra} - These {\em alpha releases} are essentially
the same as beta releases, and may occur very frequently.  They will
primarily be used to quickly get new code to close collaborators.
Alpha releases are not expected to be as stable as beta releases.
The release is tagged, but a CVS branch is not created.
\end{itemize}

%==============================
\subsection{Release Schedule}
\label{Release Schedule}

Here is the basic schedule used for general releases:
\begin{itemize}
\item {\bf Week 1, Monday} - Tell developers about impending new release.
Based on response, either go forward with the release or delay it.
\item {\bf Week 1, M-F} - Developers prepare for the release by cleaning
up code, documentation, etc.
\item {\bf Week 2, Monday} - Create a new beta release and install.
\item {\bf Week 2, Tuesday} - Initial documentation review meeting.
All changes are to be completed before Week 3, Monday.
\item {\bf Week 3, Tuesday} - Final documentation review meeting.
Additional changes to be completed before Week 3, Wednesday.
\item {\bf Week 3, Thursday} - Create a new general release and test.
\item {\bf Week 4, Monday} - Install general release on LLNL machines.
\end{itemize}

No new big code changes should be checked in during the last two weeks
of the general release schedule (i.e., weeks 2 and 3) without first
consulting with the \hypre{} software technical lead.  The idea is to
use these two weeks to allow the code to ``settle'' a little as we do
testing, etc.  Minor changes are encouraged, of course.
