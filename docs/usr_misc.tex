%==========================================================================
\chapter{Additional Information}

%==========================================================================

\section{Building the Library}

Usually, \hypre{} can be built by simply typing \kbd{./configure}
followed by \kbd{make} in the top-level source directory.
\hypre{} uses GNU Autoconf, Automake, and libtool so the building 
of the \hypre{} library should be familiar to you if you've built GNU
based packages before.

\subsection{Getting the Library}

The \hypre{} distribution available through the Lawrence Livermore 
National Laboratory \hypre{} web page at
\htmladdnormallink{http://www.llnl.gov/CASC/hypre/}
{http://www.llnl.gov/CASC/hypre/} from the Software distribution page. 
The \hypre{} Software distribution page will contain the tarballs for the 2
latest general, beta distributions, as well as documentation of \hypre{}. 
You may also subscribe to the \hypre{} software release announcement mailing 
lists form the \hypre{} web page.

\subsection{Library configuration}

After unpacking the HYPRE tarball, the distribution will be in a 
subdirectory of the form, hypre-VERSION (e.g., hypre-1.8.4, with a
"b" appended if the release is a beta).
\begin{ttfamily}
\begin{mdseries}
\linebreak
\$ \textbf{gzip -cd hypre-1.8.4b.tar.gz | tar -xf -}\linebreak
\$ \textbf{cd hypre-1.8.4b}\linebreak
\$ \textbf{ls}\linebreak
\begin{verbatim}
CHANGELOG                 INSTALL                   docs
COPYRIGHT_and_DISCLAIMER  README                    src
\end{verbatim}
\$ \textbf{cd src}\linebreak
\$ \textbf{ls}\linebreak
\begin{verbatim}
FEI_mv              blas                matrix_matrix       struct_ls
HYPRE.h             config              nopoe               struct_mv
IJ_mv               configure           parcsr_es           tarch
Makefile.am         distributed_ls      parcsr_ls           test
Makefile.in         distributed_matrix  parcsr_mv           utilities
aclocal.m4          krylov              seq_mv
babel               lapack              sstruct_ls
babel-runtime       lib                 sstruct_mv
\end{verbatim}
\end{mdseries}
\end{ttfamily}

The \hypre{} distribution subdirectory will contain files and directories 
similar to the above directory listing.

To automatically generate machine specific makefiles, type
\kbd{./configure} in the top level `src' directory.  The \file{configure}
script is a portable script generated by GNU Autoconf.  It runs a
series of tests to determine characteristics of the machine on which
it is running, and it uses the results of the these tests to produce
the machine specific makefiles, called `Makefile', from template files
called `Makefile.in' in each directory.  Once the makefiles are
produced you can run make as you would with any other makefile.

The configure script primarily does the following things:
\begin{itemize}
\item selects a compiler
\item provides either optimization or debugging options for the compiler
\item finds the headers and libraries for MPI
\end{itemize}

The configure script makes these decisions based on a hierarchical
check.  First, it attempts to identify the machine on which it is
running as a specific supported machine.  Next it will try to identify
the architecture as a supported architecture.  If both of these fail,
generic default decisions are made by the script.  However, the script
does have some command-line options that can give you control over the
choices it will make.  You can type \kbd{configure --help} to see the
list of all of the command-line options to configure. This is the best
resource for information on configure options.  The `INSTALL' file
in the top-level directory contains a little more information on
installing software using configure. 

Configure automatically generates a file \file{HYPRE_config.h} that
includes all the header files found to be necessary by configure.
This file may be used to see how a compiled version of the library was
configured and may also be included by the user in his/her own code.

\hypre{} libraries are built from the top-level source directory 
by simply typing:
\begin{ttfamily}
\begin{mdseries}
\linebreak
\$ \textbf{./configure}\linebreak
\$ \textbf{make install}\linebreak
\linebreak
\end{mdseries}
\end{ttfamily}
Note that by default \hypre{} libraries will be
installed in a subdirectory hypre/lib, and the include files in
hypre/include (i.e., --prefix=`pwd`/hypre). This differs from the standard
GNU install which uses /usr/local/lib, and /usr/local/include. After
testing (below) configure can be rerun with the --prefix=/usr/local
option, followed by make install, to get the GNU behavior.

\subsection{Testing the Library} 

After the make install step above, a few sample
drivers can be used to test the build. cd test make The test drivers ij,
ij\_es, ij\_mv, new\_ij, struct, and sstruct will be built, and should 
be runable from the command line. 
\begin{ttfamily}
\begin{mdseries}
\linebreak
\$ \textbf{./configure}\linebreak
\$ \textbf{make test}\linebreak
\$ \textbf{cd test}\linebreak
\$ \textbf{./ij}\linebreak
\$ \textbf{./ij\_es}\linebreak
\$ \textbf{./ij\_mv}\linebreak
\$ \textbf{./sstruct}\linebreak
\$ \textbf{./struct}\linebreak
\linebreak
\end{mdseries}
\end{ttfamily}
To build the (stand-alone) fei driver the following 
steps are needed, starting from the top-level source directory: 
\begin{ttfamily}
\begin{mdseries}
\linebreak
\$ \textbf{make nofei}\linebreak
\$ \textbf{cd test}\linebreak
\$ \textbf{make fei++}\linebreak
\$ \textbf{./fei}\linebreak
\linebreak 
\end{mdseries}
\end{ttfamily}

\subsection{Linking to the Library}

A program linking with \hypre{} must be compiled with
\kbd{-I\$PREFIX/include} and linked with
\kbd{-L\$PREFIX/lib -l}{\it hypre library name}... 
\kbd{-l}{\it hypre library name}..., where \kbd{\$PREFIX} is the
directory where \hypre{} is installed, specified by the configure
option \kbd{--prefix=PREFIX}.  Additionally, any other
libraries to which \hypre{} is linked must also be linked to by the
users application.  For example, the BLAS library or PETSc library are
often (but not always) linked in by \hypre{} and would also need to be
linked in by the users application.

It may be useful to reference the \code{Makefile} in the \file{test}
subdirectory.  This makefile is designed to build test applications
that link with and call \hypre{}.  All include and linking flags that
are used by \hypre{} and needed by these test applications get
exported to this file by the \file{configure} script.

\subsection{Configure Options}

Configure does it best to try and determine the
compile options needed to build HYPRE. However the defaults that
configure chooses can not meet everyones needs. Configure will display
all of its available options by typing ./configure --help. A few of the
more useful options include: 
\begin{ttfamily}
\begin{mdseries}
\linebreak
\$ \textbf{./configure --enable-shared=yes}, will build both 
shared and static libraries.\linebreak
\$ \textbf{./configure --enable-debug --without-MPI}, will compile 
a serial version and debugging.\linebreak
\$ \textbf{./configure --with-babel}, will build the babel interface 
libraries\linebreak
\$ \textbf{./configure --prefix=\$HOME}, will install \hypre{} in 
to \$HOME\linebreak
\end{mdseries}
\end{ttfamily}
To build 64-bit static libraries on AIX (Note: configure will need to
execute compiled code for its test, so configure needs to be run so
that the default AIX poe scheduling is disabled), use the following: 
\begin{ttfamily}
\begin{mdseries}
\linebreak
\$ \textbf{./nopoe ./configure CC=mpcc CXX=mpCC F77=mpxlf $\backslash$}\linebreak
\textbf{CFLAGS="-q64 -qmaxmem=8192 -DHYPRE\_COMM\_SIMPLE -O3 
-qstrict"$\backslash$}\linebreak
\textbf{CXXFLAGS="-q64 -qmaxmem=8192 -DHYPRE\_COMM\_SIMPLE -O3 
-qstrict"$\backslash$}\linebreak
\textbf{F77FLAGS="-q64 -O3 -qstrict"$\backslash$}\linebreak
\textbf{LDFLAGS="-q64" AR="ar -X64"}\linebreak
\textbf{make -i install}\linebreak
\end{mdseries}
\end{ttfamily}
To support a read-only source tree, and build everything in a 
directory, build (VPATH):
\begin{ttfamily}
\begin{mdseries}
\linebreak
\$ \textbf{mkdir build}\linebreak
\$ \textbf{cd build}\linebreak
\$ \textbf{../configure}\linebreak
\$ \textbf{make test}\linebreak
\end{mdseries}
\end{ttfamily}

\subsection{Make Targets}

\begin{ttfamily}
\begin{mdseries}
\begin{verbatim}
all                     Make all the top-level targets, the default.
clean                   Undo what ever make does.
mostlyclean             Like clean, but leaves libraries.
Maintainer-clean        Delete everything than can be rebuilt.
distclean               Undo what ever configure does.
install                 Copy (install) executables/libraries.
install-strip           Strip the installed files.
uninstall               Undo what every make install does.
TAGS                    Update the tags table.
test                    Build, install libraries, then build test drivers.
dist                    Create a distribution file of the sources files.
check                   Perform self-test (if any).
installcheck            Build and test an install.
distcheck               Build and test a distribution.
\end{verbatim}
\end{mdseries}
\end{ttfamily}

\subsection{Hints}

Use MPI wrappers for the C, C++, and Fortran compilers (e.g.,
mpicc, mpicxx, mpif77).  This will cause the least amount of 
confusion for configure and allow you to conveniently tune the 
compiler flags and libraries to your particular system. If MPI 
is enables mpicc will be the preferred compiler for C, while 
mpiCC or mpicxx will be used for C++, and mpif77 for Fortran.

If when running \kbd{./configure} on AIX you get the error:
\begin{ttfamily}
\begin{mdseries}
\begin{verbatim}
checking for C compiler default output file name... a.out
checking whether the C compiler works... configure: error: cannot run C compiled programs.
If you meant to cross compile, use `--host'.
See `config.log' for more details.
\end{verbatim}
\end{mdseries}
\end{ttfamily}
Rerun configure using \kbd{./nopoe ./configure}.

%==========================================================================

\section{Calling from Fortran}
\label{Calling from Fortran}

Although \hypre{} is written in C, a Fortran interface is provided.
The Fortran interface is very similar to the C interface, and can be
determined from the C interface by a few simple conversion rules.
These conversion rules are described below.

Let us start out with a simple example.  Consider the following
\hypre{} prototype:
\begin{display}
\begin{verbatim}
int HYPRE_IJMatrixSetValues(HYPRE_IJMatrix  matrix,
                            int  nrows, int  *ncols,
                            const int *rows, const int  *cols,
                            const double  *values);
\end{verbatim}
\end{display}
The corresponding Fortran code for calling this routine is as follows:
\begin{display}
\begin{verbatim}
      integer*8         matrix, 
      integer           nrows, ncols(MAX_NCOLS)
      integer           rows(MAX_ROWS), cols(MAX_COLS)
      double precision  values(MAX_COLS)
      integer           ierr

      call HYPRE_IJMatrixSetValues(matrix, nrows, ncols, rows, cols,
     &                             values, ierr)
\end{verbatim}
\end{display}
The Fortran subroutine name is the same, unless the name is longer
than 31 characters.  In these situations, the name is condensed to 31
characters, usually by simple truncation.  For now, users should look
at the Fortran drivers in the \code{test} directory for the correct
condensed names.  In the future, this aspect of the interface conversion
will be made consistent and straightforward.

The Fortran subroutine argument list is always the same as the
corresponding C routine, except that the error return code \code{ierr}
is always last.  Conversion from C parameter types to Fortran argument
type is summarized in Table \ref{table-fortran-interface-types}.
\begin{table}
\center
\begin{tabular}{|l|l|}
\hline
C parameter & Fortran argument \\
\hline\hline
\code{int i} & \code{integer i} \\
\code{double d} & \code{double precision d} \\
\code{int *array} & \code{integer array(*)} \\
\code{double *array} & \code{double precision array(*)} \\
\code{char *string} & \code{character string(*)} \\
\code{HYPRE_Type object} & \code{integer*8 object} \\
\code{HYPRE_Type *object} & \code{integer*8 object} \\
\hline
\end{tabular}
\caption{%
Conversion from C parameters to Fortran arguments
}
\label{table-fortran-interface-types}
\end{table}
Arrays arguments in \hypre{} are always of type \code{(int *)} or
\code{(double *)}, and the corresponding Fortran types are simply
\code{integer} or \code{double precision} arrays.  Note that the
Fortran arrays may be indexed in any manner.  For example, an integer
array of length \code{N} may be declared in fortran as either of the
following:
\begin{display}
\begin{verbatim}
      integer  array(N)
      integer  array(0:N-1)
\end{verbatim}
\end{display}

\hypre{} objects can usually be declared as in the table because
\code{integer*8} usually corresponds to the length of a pointer.
However, there may be some machines where this is not the case
(although we are not aware of any at this time).  On such machines,
the Fortran type for a \hypre{} object should be an \code{integer} of
the appropriate length.

%Fortran 77 subroutine arguments always pass copies of the argument
%addresses upon execution of the subroutine call.  This is referred to
%as call-by-address or call-by-reference.  In the called subroutine,
%the memory space at the argument address can be altered, but the
%calling address cannot be altered.
%
%C function parameters, whether pointers (addresses) or not, are
%directly copied upon entry to the function. This is referred to as
%call-by-value.  Altering the copied C parameter in the called function
%has no effect on the parameter in the calling function.  A pointer
%parameter in C can achieve the same effect as call-by-reference, and
%by this mechanism the languages can interoperate in a straightforward
%manner.
%
%Portability across typical platforms is currently achieved with the
%specific Fortran-calling-C mapping:
%
%\vspace{0.2in}
%
%\begin{tabular}{lcl}
%
%\underline{calling Fortran argument type} & &
%\underline{called C function parameter type} \\
%                              &                   &   \\
%\hspace{0.1in} (addr of) \code{integer*8}        & $\longrightarrow$ &
%\hspace{0.5in} \code{long int*} \\
%\hspace{0.1in} (addr of) \code{integer}          & $\longrightarrow$ &
%\hspace{0.5in} \code{int*} \\
%\hspace{0.1in} (addr of) \code{character}        & $\longrightarrow$ &
%\hspace{0.5in} \code{char*} \\
%\hspace{0.1in} (addr of) \code{double precision} & $\longrightarrow$ &
%\hspace{0.5in} \code{double*} \\
%
%\end{tabular}
%
%\vspace{0.2in}
%
%In particular, C-type
%\code{long int*} points to a space that can hold an address, a space
%which happens to be the size of that allocated by a Fortran
%\code{integer*8} declaration.
%In orther words, C can hold an address in a \code{long int} and
%assign that address to a Fortran \code{integer*8} memory space
%(using \code{long int*} call-by-value).
%
%Addresses in \code{integer*8} variables might not have originated
%in a Fortran call, and if they address inhomogeneously typed collections
%of data (e.g. C structures), Fortran might not have enough information to
%dereference them.  But Fortran can still pass around such addresses, and in
%particular hand them back and forth between various C functions.  The
%\hypre{} Fortran interface makes extensive use of this technique,
%as with the \code{addr} variable in the following general example:
%
%\vspace{0.1in}
%
%\noindent Generalized example:
%
%\vspace{0.1in}
%
%  Fortran calling:
%\begin{verbatim}
%      integer*8        addr
%      integer          intg, ierr
%      character        charact
%      double precision double_precis
%
%      call subroutine_name(addr, intg, charact, double_precis, ierr)
%\end{verbatim}
%
%The interface is designed so that the label \code{subroutine_name} is
%exactly the label of the \hypre{} C-function\footnote{For C function
%names under 32 characters in length, the Fortran name is the same as
%the C name.  For C function names over 31 characters, the Fortran name
%is condensed to less than 32 characters (see reference manual for
%Fortran name in such a case).}.  \hypre{} interlanguage C-wrappers
%currently account for interlanguage linking issues involving appended
%underscores.  The C-wrappers also accomodate Fortran subroutine name
%length limitations.  Fortran variable
%\code{ierr} allows error handling through the interface.
%
%\vspace{0.1in}
%
%\noindent Specific example:
%
%\vspace{0.1in}
%
%  Fortran calling:
%\begin{verbatim}
%      integer*8        IJmatrix, 
%      integer          num_nonzero_coefs, row_index
%      integer          col_indices(MAX_NUM_COLS), ierr
%      double precision coefs(MAX_NUM_COLS)
%             .
%             .
%             .
%
%      call HYPRE_IJMatrixInsertRow(IJmatrix, num_nonzero_coefs, row_index,
%                                   col_indices, coefs, ierr)
%\end{verbatim}
%
%\vspace{0.1in}
%
%  C called:
%\begin{verbatim}
%      int HYPRE_IJMatrixInsertRow( HYPRE_IJMatrix IJmatrix, int n,
%                                   int row, const int *cols,
%                                   const double *values)     
%
%      { int ierr = 0;
%              .
%              .
%              .
%
%        return(ierr); }
%\end{verbatim}
%
%\noindent Since the C function name has less than 32 characters, Fortran
%uses the same name in the subroutine call.

%==========================================================================

\section{Bug Reporting}

\hypre{} has an automated bug reporting mechanism in place that may be used 
as a resource for submitting bugs, desired features, and documentation
problems, as well as querying the status of previous reports.  Access
\htmladdnormallink{http://www-casc.llnl.gov/bugs}{http://www-casc.llnl.gov/bugs}
for full bug tracking details or to submit or query a bug report.
When using the CASC bug reporting site for the first time, click on
``Open a new Bugzilla account'' under the ``User login account
management'' heading.

%==========================================================================
