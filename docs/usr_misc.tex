%=============================================================================
%=============================================================================

\chapter{General Information}

%-----------------------------------------------------------------------------

\section{Getting the Source Code}

The \hypre{} distribution tar file is available from the Software link of the
\hypre{} web page, \url{http://www.llnl.gov/CASC/hypre/}.  The \hypre{}
Software distribution page allows access to the tar files of the latest and
previous general and beta distributions as well as documentation.

%-----------------------------

\section{Configure and Make}

After unpacking the HYPRE tar file, the source code will be in the "src" 
sub-directory of a directory named hypre-VERSION, where VERSION is the current 
version number (e.g., hypre-1.8.4, with a "b" appended for a beta release).

Move to the "src" sub-directory to build \hypre{} for the host platform.  The 
simplest method is to configure, compile and install the libraries in \file{./hypre/lib}
and \file{./hypre/include} directories, which is accomplished by:
\begin{verbatim}
   ./configure
   make
\end{verbatim}

NOTE: when executing on an IBM platform \file{configure} must be executed under 
the nopoe script (\file{./nopoe ./configure <option> ...<option>}) to force a single 
task to be run on the log-in node.

There are many options to \file{configure} and \file{make} to customize such 
things as installation directories, compilers used, compile and load flags, etc.

Executing \file{configure} results in the creation of platform specific files 
that are used when building the library. The information may include such things
as the system type being used for building and executing, compilers being used, 
libraries being searched, option flags being set, etc.  When all of the searching
is done two files are left in the \file{src} directory; \file{config.status} 
contains information to recreate the current configuration and \file{config.log}
contains compiler messages which may help in debugging \file{configure} errors. 

Upon successful completion of \file{configure} the file \file{config/Makefile.config} 
is created from its template \file{config/Makefile.config.in} and \hypre{} is 
ready to be built.

Executing \file{make}, with or without targets being specified, in the \file{src}
directory initiates compiling of all of the source code and building of the \hypre{} 
library.  If any errors occur while compiling, the user can edit the file 
\file{config/Makefile.config} directly then run \file{make} again; without having to 
re-run configure.

When building HYPRE without the install target, the libraries and include files 
will be copied into the default directories, \file{src/hypre/lib} and 
\file{src/hypre/include}, respectively.

When building HYPRE using the install target, the libraries and include files 
will be copied into the directories that the user specified in the options to 
\file{configure}, e.g. \file{--prefix=/usr/apps}.  If none were specified the default 
directories, \file{src/hypre/lib} and \file{src/hypre/include}, are used.

%-----------------------------

\subsection{Configure Options}

There are many options to \file{configure} to allow the user to override and 
refine the defaults for any system. The best way to find out what options are 
available is to display the help package, by executing \file{./configure --help}, which 
also includes the usage information.  The user can mix and match the configure options 
and variable settings to meet their needs.

Some of the commonly used options include:

\begin{verbatim}
   --enable-debug                 Sets compiler flags to generate information 
                                  needed for debugging.
   --enable-shared                Build shared libraries.
                                  NOTE: in order to use the resulting shared 
                                        libraries the user MUST have the path to
                                        the libraries defined in the environment 
                                        variable LD_LIBRARY_PATH. 
   --with-no-global-partitioning  Do NOT use global partitioning options
   --with-print-errors            Print HYPRE errors
\end{verbatim}

The user can mix and match the configure options and variable settings to meet their 
needs.  It should be noted that \hypre{} can be configured with external BLAS and/or
LAPACK libraries, which can be combined with any other option.  This is done as follows:

\begin{verbatim}
   Specifying BLAS only:
      ./configure  --with-blas-lib="blas-lib-name" --with-blas-lib-dirs="path-to-blas-lib"

   Specifying LAPACK only:
      ./configure  --with-lapack-lib="lapack-lib-name" --with-lapack-lib-dirs="path-to-lapack-lib"

   Specifying both BLAS and LAPACK:
      ./configure  --with-blas-lib="blas-lib-name" --with-blas-lib-dirs="path-to-blas-lib" \
                   --with-lapack-lib="lapack-lib-name" --with-lapack-lib-dirs="path-to-lapack-lib"
\end{verbatim}

The output from \file{configure} is several pages long.  It reports the system type 
being used for building and executing, compilers being used, libraries being searched,
option flags being set, etc.  

%-----------------------------

\subsection{Make Targets}

The make step in building \hypre{} is where the compiling, loading and creation 
of libraries occurs.  Make has several options that are called targets.  These include:
\begin{verbatim}
   help         prints the details of each target

   all          default target in all directories
                compile the entire library
                does NOT rebuild documentation

   clean        deletes all files from the current directory that are 
                   created by building the library

   distclean    deletes all files from the current directory that are created
                   by configuring or building the library

   install      compile the source code, build the library and copy executables,
                    libraries, etc to the appropriate directories for user access

   uninstall    deletes all files that the install target created

   tags         runs etags to create a tags table
                file is named TAGS and is saved in the top-level directory

   test         depends on the all target to be completed
                removes existing temporary installation directories
                creates temporary installation directories
                copies all libHYPRE* and *.h files to the temporary locations
                builds the test drivers; linking to the temporary locations to
                   simulate how application codes will link to HYPRE
\end{verbatim}


%==========================================================================

\section{Testing the Library} 

The \kbd{examples} subdirectory contains several codes that can be used to test
the newly created \hypre{} library.  To create the executable versions, move into
the \kbd{examples} subdirectory, enter \kbd{make} then execute the codes as described
in the initial comments section of each source code.

%-----------------------------

\section{Linking to the Library}

An application code linking with \hypre{} must be compiled with \kbd{-I\$PREFIX/include} 
and linked with \kbd{-L\$PREFIX/lib -lHYPRE}, where \kbd{\$PREFIX} is the directory 
where \hypre{} is installed, default is \kbd{hypre}, or as defined by the configure 
option \kbd{--prefix=PREFIX}. As noted above, if \hypre{} was built as a shared library 
the user MUST have its location defined in the environment variable \kbd{LD\_LIBRARY\_PATH}.

We strongly recommend that users link with \kbd{libHYPRE.a} rather than specifying each 
of the separate \hypre{} libraries, i.e.  \kbd{libHYPRE\_krylov.a}, \kbd{libHYPRE\_parcsr\_mv.a},
etc.  By using \kbd{libHYPRE.a} the user is guaranteed to have all capabilities without
having to change their link flags.

As an example of linking with \hypre{}, a user may refer to the \kbd{Makefile} in the 
\file{examples} subdirectory.  It is designed to build codes similar to user applications
that link with and call \hypre{}.  All include and linking flags are defined in the 
\file{Makefile.config} file by \file{configure}.

%-----------------------------------------------------------------------------

\section{Error flags}
Every \hypre{} function returns an integer, which is used to indicate errors
during execution.  Note that the error flag returned by a given function reflects
the errors from {\em all} previous calls to \hypre{} functions.  In particular, 
a value of zero means that all \hypre{} functions up to (and including) the 
current one have completed successfully.  This new error flags system is being
implemented throughout the library, but currently there are still functions that
do not support it.

The error flag is a combination of one or a few error codes, which are
defined in \code{utilities.h} and currently are:
\begin{display} \begin{verbatim}
#define HYPRE_ERROR_GENERIC      1<<0   /* generic error */
#define HYPRE_ERROR_MEMORY       1<<1   /* unable to allocate memory */
#define HYPRE_ERROR_ARG          1<<2   /* argument error */
/* bits 4-8 are reserved for the index of the argument error */
#define HYPRE_ERROR_CONV         1<<8   /* method did not converge as expected */
\end{verbatim} \end{display}

One can use \code{&} to determine exactly which errors have occurred:
\begin{display} \begin{verbatim}
/* call some HYPRE functions */
hypre_ierr = HYPRE_Function();

/* check if the previously called hypre functions returned error(s) */
if (hypre_ierr)
{
   /* check if the error with code HYPRE_ERROR_CODE has occurred */
   if (hypre_ierr & HYPRE_ERROR_CODE)
}
\end{verbatim} \end{display}

The global error flag can also be obtained directly by
calling \code{HYPRE\_GetError()}. If \code{HYPRE\_ERROR\_ARG}
has occurred the argument index can be obtained from \code{HYPRE\_GetErrorArg()}.
To print a description of a given error flag, use
\begin{display} \begin{verbatim}
HYPRE_DescribeError(int ierr, FILE *stream)
\end{verbatim} \end{display}
Finally, if \hypre{} was configured with \code{--with-print-errors}, additional
error information will be printed to the standard error during execution.

%-----------------------------------------------------------------------------

\section{HYPRE LinearSystemCore Installation}
\label{LSI_install}

The ultimate objective is for application users to have immediate access
to the latest FEI/\hypre{} library files on different computing platforms
via public {\sf lib} directories.  While this feature is forthcoming, careful 
version control is needed for users to keep track of capabilities and bug fixes 
for different installations.  Users who would like to set up the FEI/\hypre{}
on their own should do the following :

\begin{enumerate}

\item obtain the \hypre{} and the Sandia FEI source codes (alternatively, use
      the {\sf FEI} implementation in \hypre{}),
\item compile Sandia's {\sf FEI} (fei-2.5.0) to create the
      {\sf libfei\_base.a} file.
\item compile \hypre{} 
\begin{enumerate}
\item download \hypre{} from the web, ungzip and untar it
\item go into the {\sf linear\_solvers} directory
\item do a 'configure' with the {\sf --with-fei-inc-dir} option set to
      the {\sf FEI} include directory plus other compile options
\item compile with {\sf make install} to create the
      {\sf libHYPRE\_LSI*} file in the {\sf linear\_solvers/hypre/lib}
      directory.
\end{enumerate}
\item call the {\sf FEI} functions in your application code (example given
      previously)
\begin{enumerate}
\item include {\sf cfei\-hypre.h} in your file 
\item include {\sf FEI\_Implementation.h} in your file 
\item make sure your application has an {\sf include} and an {\sf lib} path 
      to the {\sf include} and {\sf lib} directories created above. 
\end{enumerate}

\end{enumerate}

%\subsection{Linking with the library files}

To link the {\sf FEI} and \hypre{} into the executable, the following has to be
attached to the linking command :

\begin{tabbing}
\hspace{0.5in} \= {\sf -L\$$\{$LIBPATHS$\}$ -lfei\_base -lHYPRE\_LSI} 
\end{tabbing}
along with all the other libraries (Note : the order in which the libraries are
listed may be important), where {\sf LIBPATHS} are where 
the \hypre{} and {\sf FEI} libraray files can be found.  

Since some of these library files make calls to LAPACK and BLAS functions, 
the corresponding libraries need to be linked along with (placed after) these 
library files.  
%For example, on the DEC cluster, it suffices to link
%with the {\sf dxml} library, (So {\sf -ldxml} is placed after the above link
%sequence, with {\sf -lm} placed after {\sf -ldxml}.) while the {\it essl}
%library can be used on the blue machine. If {\sf SuperLU} is also needed,
%{\sf -lHYPRE\_superlu} should be placed immediately after {\sf HYPRE\_LSI}.

%\subsection{Some more caveats for application developers}

Building an application executable often requires linking with many different
software packages, and many software packages use some LAPACK and/or BLAS
functions.  In order to alleviate the problem of multiply defined functions
at link time, it is recommended that all software libraries are stripped of
all LAPACK and BLAS function definitions.  These LAPACK and BLAS functions 
should then be resolved at link time by linking with the system LAPACK and
BLAS libraries (e.g. dxml on DEC cluster).  Both \hypre{} and SuperLU were
built with this in mind.  However, some other software library files needed
may have the BLAS functions defined in them.  To avoid the problem of
multiply defined functions, it is recommended that the offending library
files be stripped of the BLAS functions.

%\subsection{Comments about the FEI/\hypre{} Interface and Contacts}

%Comments about \hypre{}'s finite element interface can be directed
%to Charles Tong (925-422-3411, chtong@llnl.gov).

\section{Calling HYPRE from Fortran}

A Fortran interface is provided in \hypre{} to enable such applications to call its C
routines.  Typically, the Fortran subroutine name is the same as the C name, unless 
it is longer than 31 characters.  In these situations, the name is condensed to 31
characters, usually by simple truncation.  For now, users should look at the Fortran 
test drivers (*.f codes) in the \code{test} directory for the correct condensed 
names.  In the future, this aspect of the interface conversion will be made 
consistent and straightforward.

The Fortran subroutine argument list is always the same as the corresponding C routine, 
except that the error return code \code{ierr} is always last.  Conversion from C parameter
types to Fortran argument type is summarized in Table \ref{table-fortran-interface-types}.

\begin{table}
\center
\begin{tabular}{|l|l|}
\hline
C parameter & Fortran argument \\
\hline\hline
\code{int i} & \code{integer i} \\
\code{double d} & \code{double precision d} \\
\code{int *array} & \code{integer array(*)} \\
\code{double *array} & \code{double precision array(*)} \\
\code{char *string} & \code{character string(*)} \\
\code{HYPRE\_Type object} & \code{integer*8 object} \\
\code{HYPRE\_Type *object} & \code{integer*8 object} \\
\hline
\end{tabular}
\caption{%
Conversion from C parameters to Fortran arguments
}
\label{table-fortran-interface-types}
\end{table}


Array arguments in \hypre{} are always of type \code{(int *)} or \code{(double *)}, 
and the corresponding Fortran types are simply \code{integer} or \code{double precision} 
arrays.  Note that the Fortran arrays may be indexed in any manner.  For example, an integer
array of length \code{N} may be declared in fortran as either of the following:
\begin{display}
\begin{verbatim}
      integer  array(N)
      integer  array(0:N-1)
\end{verbatim}
\end{display}

\hypre{} objects can usually be declared as in the table because \code{integer*8} 
usually corresponds to the length of a pointer.  However, there may be some machines
where this is not the case (although we are not aware of any at this time).  On such
machines, the Fortran type for a \hypre{} object should be an \code{integer} of
the appropriate length.

This simple example illustrates the above information: 

C prototype:
\begin{display}
\begin{verbatim}
int HYPRE_IJMatrixSetValues(HYPRE_IJMatrix  matrix,
                            int  nrows, int  *ncols,
                            const int *rows, const int  *cols,
                            const double  *values);
\end{verbatim}
\end{display}

The corresponding Fortran code for calling this routine is as follows:
\begin{display}
\begin{verbatim}
      integer*8         matrix, 
      integer           nrows, ncols(MAX_NCOLS)
      integer           rows(MAX_ROWS), cols(MAX_COLS)
      double precision  values(MAX_COLS)
      integer           ierr

      call HYPRE_IJMatrixSetValues(matrix, nrows, ncols, rows, cols,
     &                             values, ierr)
\end{verbatim}
\end{display}

%Fortran 77 subroutine arguments always pass copies of the argument
%addresses upon execution of the subroutine call.  This is referred to
%as call-by-address or call-by-reference.  In the called subroutine,
%the memory space at the argument address can be altered, but the
%calling address cannot be altered.
%
%C function parameters, whether pointers (addresses) or not, are
%directly copied upon entry to the function. This is referred to as
%call-by-value.  Altering the copied C parameter in the called function
%has no effect on the parameter in the calling function.  A pointer
%parameter in C can achieve the same effect as call-by-reference, and
%by this mechanism the languages can interoperate in a straightforward
%manner.
%
%Portability across typical platforms is currently achieved with the
%specific Fortran-calling-C mapping:
%
%\vspace{0.2in}
%
%\begin{tabular}{lcl}
%
%\underline{calling Fortran argument type} & &
%\underline{called C function parameter type} \\
%                              &                   &   \\
%\hspace{0.1in} (addr of) \code{integer*8}        & $\longrightarrow$ &
%\hspace{0.5in} \code{long int*} \\
%\hspace{0.1in} (addr of) \code{integer}          & $\longrightarrow$ &
%\hspace{0.5in} \code{int*} \\
%\hspace{0.1in} (addr of) \code{character}        & $\longrightarrow$ &
%\hspace{0.5in} \code{char*} \\
%\hspace{0.1in} (addr of) \code{double precision} & $\longrightarrow$ &
%\hspace{0.5in} \code{double*} \\
%
%\end{tabular}
%
%\vspace{0.2in}
%
%In particular, C-type
%\code{long int*} points to a space that can hold an address, a space
%which happens to be the size of that allocated by a Fortran
%\code{integer*8} declaration.
%In other words, C can hold an address in a \code{long int} and
%assign that address to a Fortran \code{integer*8} memory space
%(using \code{long int*} call-by-value).
%
%Addresses in \code{integer*8} variables might not have originated
%in a Fortran call, and if they address inhomogeneously typed collections
%of data (e.g. C structures), Fortran might not have enough information to
%dereference them.  But Fortran can still pass around such addresses, and in
%particular hand them back and forth between various C functions.  The
%\hypre{} Fortran interface makes extensive use of this technique,
%as with the \code{addr} variable in the following general example:
%
%\vspace{0.1in}
%
%\noindent Generalized example:
%
%\vspace{0.1in}
%
%  Fortran calling:
%\begin{verbatim}
%      integer*8        addr
%      integer          intg, ierr
%      character        charact
%      double precision double_precis
%
%      call subroutine_name(addr, intg, charact, double_precis, ierr)
%\end{verbatim}
%
%The interface is designed so that the label \code{subroutine_name} is
%exactly the label of the \hypre{} C-function\footnote{For C function
%names under 32 characters in length, the Fortran name is the same as
%the C name.  For C function names over 31 characters, the Fortran name
%is condensed to less than 32 characters (see reference manual for
%Fortran name in such a case).}.  \hypre{} interlanguage C-wrappers
%currently account for interlanguage linking issues involving appended
%underscores.  The C-wrappers also accomodate Fortran subroutine name
%length limitations.  Fortran variable
%\code{ierr} allows error handling through the interface.
%
%\vspace{0.1in}
%
%\noindent Specific example:
%
%\vspace{0.1in}
%
%  Fortran calling:
%\begin{verbatim}
%      integer*8        IJmatrix, 
%      integer          num_nonzero_coefs, row_index
%      integer          col_indices(MAX_NUM_COLS), ierr
%      double precision coefs(MAX_NUM_COLS)
%             .
%             .
%             .
%
%      call HYPRE_IJMatrixInsertRow(IJmatrix, num_nonzero_coefs, row_index,
%                                   col_indices, coefs, ierr)
%\end{verbatim}
%
%\vspace{0.1in}
%
%  C called:
%\begin{verbatim}
%      int HYPRE_IJMatrixInsertRow( HYPRE_IJMatrix IJmatrix, int n,
%                                   int row, const int *cols,
%                                   const double *values)     
%
%      { int ierr = 0;
%              .
%              .
%              .
%
%        return(ierr); }
%\end{verbatim}
%
%\noindent Since the C function name has less than 32 characters, Fortran
%uses the same name in the subroutine call.

%-----------------------------------------------------------------------------

\section{Bug Reporting}

An automated bug reporting mechanism has been set up for \hypre{} to be used for
submitting bugs, desired features and documentation problems, as well as
querying the status of previous reports.  Access
\url{http://www-casc.llnl.gov/bugs} for full bug tracking details or to submit
or query a bug report.  When using the site for the first time, click on ``Open
a new Bugzilla account'' under the ``User login account management'' heading.
