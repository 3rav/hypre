\chapter{Finite Element Interface}
\label{ch-FEI}

\section{Introduction}

User applications access the \hypre{} linear solvers via a pipeline of
two interfaces - user to finite element interface (called {\tt FEI}),
and the finite element to linear solver interface (called 
{\tt LinearSystemCore}). The purpose of {\tt FEI} is to allow users 
to submit the global matrices in the form of element connectivities, 
element stiffness matrices, element loads, and boundary conditions. 
The element information is processed by an implementation of the 
{\tt FEI} (see \cite{FEI-ref}) which loads the global matrix and right
hand side vectors to the linear solver libraries via the 
{\tt LinearSystemCore} interface.
The {\tt LinearSystemCore} interface also facilitates interfacing 
multiple linear system solver packages (such as PetSC or Aztec)
with little change in the user code.

The specification of the {\tt FEI} and its implementation was first
developed at Sandia. A simplified implementation has been implemented
at LLNL in \hypre{}'s finite element module. While most of \hypre{}
is written in C, the {\tt FEI} and {\tt LinearSystemCore}
interfaces are written in C++. In the next section, we 
describe the basic {\tt FEI} functions and a sample program to 
demonstrate how to use them. 
%A brief description of \hypre{}'s 
%internal data structure and solver capabilities is presented in 
%Section 5.3.  Associated with \hypre{}'s finite element interface is 
%an FE-based gray-box multilevel preconditioning module called 
%{\tt MLI} which provides fast multilevel preconditioners.
%A description of the {\tt MLI} is given in Section 5.4.
%In Section 5.5, we describe the available options for using \hypre{}'s
%rich solver capabilities. 
Users who prefer to create their own finite
element packages but would like to use the \hypre{} solvers can link their
packages to \hypre{} via the {\tt LinearSystemCore}. A description
of this interface is given in Section \ref{LSI_overview}. Some installation 
and usage issues are discussed in Section \ref{LSI_install}.

\section{A Brief Description of The Finite Element Interface}

Embedded in application finite element codes are data structures
storing element connectivities, element stiffness matrices, element
loads, boundary conditions, nodal coordinates, etc. An implicit finite
element problem can be solved by assembling the global stiffness matrix
and the corresponding right hand side vector, and then calling linear
solver to calculate the solution. The first step in this process is 
thus to instantiate an {\tt FEI} object by
\begin{tabbing}
\hspace{0.5in} \= {\tt feiPtr = new FEI\_Implementation(solver,mpiComm,rank);}
\end{tabbing}
where {\tt solver} is a container pointing to the underlying linear solver 
package, {\tt mpiComm} is an MPI communicator (e.g. {\tt MPI\_COMM\_WORLD}),
and {\tt rank} is the processor number of the master processor (for 
identifying which processor will produce the screen outputs). 
Next, various finite element information need to be sent into the {\tt FEI}
object.

The first entity to be submitted to the {\tt FEI} is {\it field} information. 
A {\it field} has an identifier called {\tt fieldID} and a rank or
{\tt fieldSize} (number of degrees of freedom). For example, for a simple
3D incompressible Navier Stokes equation, the nodal variable is the velocity
vector which has $3$ degrees of freedom; and the element variable (constant
over the element) is the pressure (scalar). If these are the only variables,
and if we assign {\tt fieldID} $7$ and $8$ to them, respectively, then the
finite element field information can be set up by
\begin{tabbing}
\hspace{0.5in} \= {\tt nFields = 2;} \\
               \> {\tt fieldID = new int[nFields];} \\
               \> {\tt fieldID[0] = $7$; /* velocity vector */} \\
               \> {\tt fieldID[1] = $8$; /* pressure */} \\
               \> {\tt fieldSize = new int[nFields];} \\
               \> {\tt fieldSize[0] = $3$; /* velocity vector */} \\
               \> {\tt fieldSize[1] = $1$; /* pressure */ } \\
               \> {\tt feiPtr$->$initFields(nFields, fieldSize, fieldID);}
\end{tabbing}

Once the field information has been established, we are ready to initialize
an element block. An element block is characterized by the block identifier,
the number of elements, the number of nodes per element, the nodal fields 
and the element fields (fields that have been defined previously). Suppose 
we use $1000$ hexahedral elements in the element block $0$, the setup 
consists of
\begin{tabbing}
\hspace{0.5in} \= {\tt elemBlkID = 0;} \\
               \> {\tt nElems = 1000;} \\
               \> {\tt elemNNodes = 8; /* number of nodes per element */} \\
               \> {\tt nodeNFields = 1; /* nodal field - velocity */} \\
               \> {\tt nodeFieldIDs = new[nodeNFields];} \\
               \> {\tt nodeFieldIDs[0] = fieldID[0]; /* velocity */ } \\
               \> {\tt elemNFields = 1; /* element field - pressure */} \\
               \> {\tt elemFieldIDs = new[elemNFields];} \\
               \> {\tt elemFieldIDs[0] = fieldID[1]; /* pressure */ } \\
 \> {\tt feiPtr$->$initElemBlock(elemBlkID, nElems, elemNNodes, nodeNFields, nodeFieldIDs,}\\
 \> \hspace{1.0in} {\tt elemNFields, elemFieldIDs, 0);} 
\end{tabbing}
The last argument is to specify how the dependent variables are arranged in
the element matrices. A value of $0$ indicates that each variable is to be
arranged in a separate block (as opposed to interleaving).

In a parallel environment, each processor has one or more element blocks.
Unless the element blocks are all disjoint, some of the element blocks
share a common set of nodes on the subdomain boundaries. To facilitate
setting up interprocessor communications, shared nodes between subdomains
on different processors are to be identified and sent to the {\tt FEI}.
Hence, each node in the whole domain is assigned a unique global
identifier. The shared node list on each processor contains a subset
of the global node list
corresponding to the local nodes that are shared with the other processors.
The syntax for setting up the shared nodes is
\begin{tabbing}
\hspace{0.5in} \= {\tt feiPtr$->$initSharedNodes(nShared, sharedIDs, sharedLengs, sharedProcs);}
\end{tabbing}
This completes the initialization phase, and a completion signal is sent to
the {\tt FEI} via
\begin{tabbing}
\hspace{0.5in} \= {\tt feiPtr$->$initComplete();}
\end{tabbing}

Next we begin the {\it load} phase. The first entity for loading is the
nodal boundary conditions. Here we need to specify the number of boundary
equations and the boundary values given by {\tt alpha, beta}, and {\tt gamma}.  Depending whether the boundary conditions are Dirichlet, Neumann, or mixed,
the three values should be passed into the {\tt FEI} accordingly. 

The element stiffness matrices are to be loaded in the next step. We need
to specify the element number $i$, the element block to which element $i$
belongs, the element connectivity information, the element load, and the
element matrix format. The element connectivity specifies a set of $8$ node
global IDs (for hexahedral elements), and the element load is the load or
force for each degree of freedom.  The element format specifies how the
equations are arranged (similar to the interleaving scheme mentioned above).
The calling sequence for loading element stiffness matrices is
\begin{tabbing}
\hspace{0.5in} \= {\tt for (iE = 0; iE < nElems; iE++)} \\
 \> \hspace{0.5in} {\tt feiPtr$->$sumInElem(elemBlkID, elemID, elemConn[iE], elemStiff[iE],} \\
 \> \hspace{1.5in} {\tt elemLoads[iE], elemFormat);}
\end{tabbing}
Again, to complete the loading phase, a completion signal is sent to 
the {\tt FEI} via
\begin{tabbing}
\hspace{0.5in} \= {\tt feiPtr$->$loadComplete();}
\end{tabbing}
 
Now the global stiffness matrix and the corresponding right hand side
have been assembled. Before the linear system is solved, a number of 
solver parameters have to be passed into the {\tt FEI}. A detailed description
of the solver parameters is given in Section 3. An example is given below
\begin{tabbing}
\hspace{0.5in} \= {\tt nParams = 5;} \\
               \> {\tt paramStrings = new char*[nParams];} \\
               \> {\tt for (i = 0; i < nParams; i++) }\\
               \> \hspace{0.5in} {\tt paramStrings[i] = new char[100];} \\
               \> {\tt strcpy(paramStrings[0], "solver cg");} \\
               \> {\tt strcpy(paramStrings[1], "preconditioner diag");} \\
               \> {\tt strcpy(paramStrings[2], "maxiterations 100");} \\
               \> {\tt strcpy(paramStrings[3], "tolerance 1.0e-6");} \\
               \> {\tt strcpy(paramStrings[4], "outputLevel 1");} \\
               \> {\tt feiPtr$->$parameters(nParams, paramStrings);} 
\end{tabbing}

To solve the linear system of equations, we call
\begin{tabbing}
\hspace{0.5in} \= {\tt feiPtr$->$solve(\&status);}
\end{tabbing}
where {\tt status} is returned from the {\tt FEI} indicating whether
the solve is successful. Finally, the solution can be retrieved by
\begin{tabbing}
\hspace{0.5in} \= {\tt feiPtr$->$getBlockNodeSolution(elemBlkID, nNodes, nodeIDList, solnOffsets, solnValues);}
\end{tabbing}
where {\tt solnOffsets[i]} is the index pointing to the first location 
where the variables at node $i$ are returned in {\tt solnValues}.

\section{The LinearSystemCore Interface}
\label{LSI_overview}

As described before, users who prefer to create their own finite element
interface package can also take advantage of the rich solver capabilities
in \hypre{}. In this section we show how to access {\tt HYPRE\_LinSysCore}'s
internal solver directly.  
The matrix class in \hypre{} accessible via the {\tt LinearSystemCore} interface
is the parallel compressed sparse row ({\tt ParCSR}) matrix.  The
requirements about how the global matrix is partitioned among the
processors are that each processor holds a contiguous block of rows and columns
and the equation numbers in processors of lower rank are lower than those
in processors of higher rank.  The {\tt FEI} is responsible for ensuring
that these two requirements are followed. The matrix can be loaded in
parallel - a row or a block of rows at a time.  The solution and right
hand side vectors are constructed accordingly. The matrix rows corresponding
to the shared nodes can be assigned to either processor, and is determined
by the {\tt FEI} itself. Once the incoming matrix and vector data have
been captured in the \hypre{} {\tt ParCSR} format, a whole of matrix and
vector operators are available for use in the \hypre{} solvers.

The following program segment describes the function calls to set up
the internal matrix and solve the linear system.
Users need first to 
construct an array (say, {\tt eqnOffsets} describing the matrix row
partitioning across all processors (so {\tt eqnOffsets[p]} and
{\tt eqnOffset[p+1]} have the starting and ending row indices for processor
p). Furthermore, suppose the local submatrix has been constructed as a
compressed sparse row (CSR) matrix in the {\tt ia, ja, val} arrays. 

\begin{tabbing}
\hspace{0.5in} \= {\tt Program Segment} \\[1mm]
\> {\tt startRow = eqnOffsets[mypid];} \\
\> {\tt endRow = eqnOffsets[mypid+1] - 1;} \\
\> {\tt nrows = endRow - startRow + 1} \\
\> {\tt for ( i = startRow; i $<=$ endRow; i++ ) $\{$ } \\
\> \hspace{0.3in} \= {\tt ncnt = ia[i+1] - ia[i];} \\
\> \> {\tt rowLengths[i-startRow] = ncnt;} \\
\> \> {\tt colIndices[i-startRow] = new int[ncnt];} \\
\> \> {\tt k = 0;} \\
\> \> {\tt for (j = ia[i]; j < ia[i+1]; j++) colIndices[i-startRow][k++] = ja[j];}\\
\> \} \\
\> {\tt HYPRE\_LinSysCore\_create(\&lsc, MPI\_COMM\_WORLD);} \\
\> {\tt HYPRE\_setGlobalOffsets(lsc, nrows, NULL, eqnOffsets, NULL);} \\
\> {\tt HYPRE\_setMatrixStructure(lsc, colIndices, rowLengths, NULL, NULL, NULL);} \\
\> {\tt for ( i = startRow; i <= endRow; i++ ) $\{$ } \\
\> \> {\tt ncnt = ia[i+1] - ia[i];} \\
\> \> {\tt HYPRE\_sumIntoSystemMatrix(lsc, i, ncnt, \&val[ia[i]], \&ja[ia[i]]);}\\
\> \> {\tt HYPRE\_sumIntoRHSVector(1, \&rhs[i], \&i);} \\
\> \} \\
\> {\tt HYPRE\_matrixLoadComplete();}\\
\> {\tt strcpy(paramString, "solver gmres");} \\
\> {\tt HYPRE\_parameters(1, \&paramString);} \\
\> {\tt strcpy(paramString, "preconditioner boomeramg");} \\
\> {\tt HYPRE\_parameters(1, \&paramString);} \\
\> {\tt HYPRE\_launchSolver(\&status, \&iterations);}
\end{tabbing}

A list of available functions is given in the reference manual.
%
%\begin{tabbing}
%{\tt HYPRE\_LinSysCore\_create(LinSysCore **lsc, MPI\_Comm comm)} \\[1mm]
%{\tt HYPRE\_LinSysCore\_destroy(LinSysCore **lsc)} \\[1mm]
%{\tt HYPRE\_parameters(LinSysCore *lsc, int nParams, char **params)} \\[1mm]
%{\tt HYPRE\_setGlobalOffsets(LinSysCore* lsc, int leng, int* nodeOffsets,} \\
%\hspace{1.0in} {\tt int* eqnOffsets, int* blkEqnOffsets)} \\[1mm]
%{\tt HYPRE\_setMatrixStructure(LinSysCore *lsc, int** ptColIndices,} \\
%\hspace{1.0in} {\tt int* ptRowLengths, int** blkColIndices, int* blkRowLengths, int* ptRowsPerBlkRow)} \\[1mm]
%{\tt HYPRE\_resetMatrixAndVector(LinSysCore *lsc, double val)} \\[1mm]
%{\tt HYPRE\_resetMatrix(LinSysCore *lsc, double val)} \\[1mm]
%{\tt HYPRE\_resetRHSVector(LinSysCore *lsc, double val)} \\[1mm]
%{\tt HYPRE\_sumIntoSystemMatrix(LinSysCore *lsc, int numPtRows, const int* ptRows,}\\
%\hspace{1.0in} {\tt int numPtCols, const int* ptCols, int numBlkRows, const int* blkRows,} \\
%\hspace{1.0in} {\tt int numBlkCols, const int* blkCols, const double* const* values)} \\[1mm]
%{\tt HYPRE\_sumIntoRHSVector(LinSysCore *lsc, int num, const double* values, const int* indices)} \\[1mm]
%{\tt HYPRE\_matrixLoadComplete(LinSysCore *lsc)} \\[1mm]
%{\tt HYPRE\_enforceEssentialBC(LinSysCore *lsc, int* globalEqn, double* alpha,
%                             double* gamma, int leng)} \\[1mm]
%
%{\tt HYPRE\_enforceRemoteEssBCs(LinSysCore *lsc,int numEqns,int* globalEqns, int** colIndices,} \\
%\hspace{1.0in} {\tt int* colIndLen, double** coefs)} \\[1mm]

%{\tt HYPRE\_enforceOtherBC(LinSysCore *lsc, int* globalEqn, double* alpha, double *beta} \\
%\hspace{1.0in} {\tt double* gamma, int leng)} \\[1mm]

%{\tt HYPRE\_putInitialGuess(LinSysCore *lsc, const int* eqnNumbers,
%                          const double* values, int leng)} \\[1mm]
%{\tt HYPRE\_getSolution(LinSysCore *lsc, double *answers, int leng)} \\[1mm]

%{\tt HYPRE\_getSolnEntry(LinSysCore *lsc, int eqnNumber, double *answer)} \\[1mm]

%{\tt HYPRE\_formResidual(LinSysCore *lsc, double *values, int leng)} \\[1mm]

%{\tt HYPRE\_launchSolver(LinSysCore *lsc, int *solveStatus, int *iter)} \\[1mm]
%\end{tabbing}

%\section{HYPRE LinearSystemCore Installation}
%
%The ultimate objective is for application users to have immediate access
%to the latest FEI/\hypre{} library files on different computing platforms
%via public {\tt lib} directories.  While this feature is forthcoming, careful 
%version control is needed for users to keep track of capabilities and bug fixes 
%for different installations.  Users who would like to set up the FEI/\hypre{}
%on their own should do the following :
%
%\begin{enumerate}
%
%\item obtain the \hypre{} and the Sandia FEI source codes (alternatively, use
%      the {\tt FEI} implementation in \hypre{}),
%\item compile Sandia's {\tt FEI} (fei-2.5.0) to create the
%      {\tt libfei\_base.a} file.
%\item compile \hypre{} 
%\begin{enumerate}
%\item download \hypre{} from the web, ungzip and untar it
%\item go into the {\tt linear\_solvers} directory
%\item do a 'configure' with the {\tt --with-fei-inc-dir} option set to
%      the {\tt FEI} include directory plus other compile options
%\item compile with {\tt make install} to create the
%      {\tt libHYPRE\_LSI*} file in the {\tt linear\_solvers/hypre/lib}
%      directory.
%\end{enumerate}
%\item call the {\tt FEI} functions in your application code (example given
%      previously)
%\begin{enumerate}
%\item include {\tt cfei\-hypre.h} in your file 
%\item include {\tt FEI\_Implementation.h} in your file 
%\item make sure your application has an {\tt include} and an {\tt lib} path 
%      to the {\tt include} and {\tt lib} directories created above. 
%\end{enumerate}
%
%\end{enumerate}
%
%%\subsection{Linking with the library files}
%
%To link the {\tt FEI} and \hypre{} into the executable, the following has to be
%attached to the linking command :
%
%\begin{tabbing}
%\hspace{0.5in} \= {\tt -L\$$\{$LIBPATHS$\}$ -lfei\_base -lHYPRE\_LSI} 
%\end{tabbing}
%along with all the other libraries (Note : the order in which the libraries are
%listed may be important), where {\tt LIBPATHS} are where 
%the \hypre{} and {\tt FEI} libraray files can be found.  
%
%Since some of these library files make calls to LAPACK and BLAS functions, 
%the corresponding libraries need to be linked along with (placed after) these 
%library files.  
%%For example, on the DEC cluster, it suffices to link
%%with the {\tt dxml} library, (So {\tt -ldxml} is placed after the above link
%%sequence, with {\tt -lm} placed after {\tt -ldxml}.) while the {\it essl}
%%library can be used on the blue machine. If {\tt SuperLU} is also needed,
%%{\tt -lHYPRE\_superlu} should be placed immediately after {\tt HYPRE\_LSI}.
%
%%\subsection{Some more caveats for application developers}
%
%Building an application executable often requires linking with many different
%software packages, and many software packages use some LAPACK and/or BLAS
%functions.  In order to alleviate the problem of multiply defined functions
%at link time, it is recommended that all software libraries are stripped of
%all LAPACK and BLAS function definitions.  These LAPACK and BLAS functions 
%should then be resolved at link time by linking with the system LAPACK and
%BLAS libraries (e.g. dxml on DEC cluster).  Both \hypre{} and SuperLU were
%built with this in mind.  However, some other software library files needed
%may have the BLAS functions defined in them.  To avoid the problem of
%multiply defined functions, it is recommended that the offending library
%files be stripped of the BLAS functions.
%
%%\subsection{Comments about the FEI/\hypre{} Interface and Contacts}
%
%%Comments about \hypre{}'s finite element interface can be directed
%%to Charles Tong (925-422-3411, chtong@llnl.gov).
%
