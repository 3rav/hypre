\chapter{Finite Element Interface (FEI)}
\label{Finite Element Interface}

The finite element interface (FEI) defines a linear solver interface 
for finite element applications.  For information on how to use this
interface, see \cite{FEI-ref}.  
This chapter describes the iterative methods and 
preconditioners in the \hypre{} implementation
of this interface.

Solving a linear system from a finite element problem consists of
four steps in the FEI:
\begin{enumerate}
\item Initialize the structure of the finite-element data,
      including loading the element connectivity data
\item Load the element or super-element stiffness matrices and forcing terms
\item Set solver parameters and solve the linear system
\item Retrieve the solution to the linear system
\end{enumerate}

Parameters to the \hypre{} solvers are specified by calling
\begin{display}
\begin{verbatim}
void FEI_parameters(int sysHandle, int numParams, char **paramStrings);
\end{verbatim}
\end{display}
where {\tt sysHandle} is an identifier for the linear system being solved,
{\tt numParams} is the number of parameter strings, and {\tt paramStrings} is
an array of null-terminated strings with the format: 
``{\em parameter\_name value}''.
For example, setting the preconditioner can be accomplished by:
\begin{display}
\begin{verbatim}
char **paramStrings[1];
paramStrings[0] = (char *) malloc(64*sizeof(char));
strcpy(paramStrings[0], "preconditioner parasails");
FEI_parameters(sysHandle, 1, paramStrings);
\end{verbatim}
\end{display}
All possible parameters are listed in Table \ref{table-fei-param}.


A linear system is then solved by calling 
\begin{display}
\begin{verbatim}
void FEI_iterateToSolve(int sysHandle);
\end{verbatim}
\end{display}

\section{Iterative methods and preconditioners available}

\subsection{Iterative methods}

\begin{enumerate}
\item Krylov solvers (conjugate gradient, GMRES, TFQMR, BiCGSTAB)
\item BoomerAMG (a parallel algebraic multigrid solver)
\item SuperLU direct solver (sequential)
\item SuperLU direct solver with iterative refinement (sequential)
%\item Y12M direct solver
\end{enumerate}

\subsection{Preconditioners}

\begin{enumerate}
\item diagonal
\item parallel incomplete LU with threshold (PILUT)
\item another parallel incomplete LU (Euclid)
\item parallel algebraic multigrid (BoomerAMG)
\item parallel sparse approximate inverse (ParaSails)
\end{enumerate}

%Capabilities under development are :
%\begin{enumerate}
%\item to form and solve a reduced linear system (in view of constraints), and
%\item a link to the ML's parallel smoothed aggregation method.
%\end{enumerate}

\begin{table}[h]
\center
\begin{tabular}{|l|p{4.5in}|}
\hline
Parameter Name & Parameter Values \\
\hline\hline
solver &
\code{cg}, \code{gmres} (default), \code{bicgstab}, \code{tfqmr}, \code{boomeramg}, \code{superlu}, \code{superlux}
\\
preconditioner &
\code{diagonal} (default), \code{pilut}, \code{parasails}, \code{boomeramg}, \code{euclid}
\\
gmresDim &
an integer specifying the value of \code{m} in restarted GMRES(m).
The default value is 50.
\\
maxIterations &
an integer specifying the maximum number of iterations permitted for
CG or GMRES.  The default value is 1000.
\\
tolerance &
a floating point number specifying the termination criterion for CG or
GMRES.  The default value is 1.0E-10.
\\
pilutFillin &
an integer specifying the maximum number of nonzeros kept in the
formation of incomplete L and U.  If this is not called, a value will
be selected based on the sparsity of the matrix.
\\
pilutDropTol &
a floating point number specifying the threshold to drop small entries
in L and U.  The default value is 0.0.
\\
euclidNlevels &
a non-negative integer specifying the desired sparsity of the incomplete
factors. The default value is 0.
\\
euclidThreshold &
a floating point number specifying the threshold used to sparsify the 
incomplete factors. The default value is 0.0.
\\
superluOrdering &
\code{natural} (default) or \code{mmd} (minimum degree ordering).  This
ordering is used to minimize the number of nonzeros generated in the
LU decomposition.  The default is natural ordering.
\\
superluScale &
\code{y} (yes; perform row and column scalings before decomposition) or
\code{n} (no; default).
\\
amgCoarsenType &
\code{falgout}, \code{ruge}, or \code{default} (CLJP) coarsening for BoomerAMG.
\\
amgNumSweeps &
an integer specifying the number of pre- and post-smoothing at each
level of BoomerAMG.  The default is one pre- and one post-smoothings.
\\
amgRelaxType &
\code{jacobi} (Damped Jacobi), \code{gs-slow} (sequential Gauss-Seidel),
\code{gs-fast} (Gauss-Seidel on interior nodes), \code{hybrid}, or
\code{direct}. The default is \code{hybrid}.
\\
amgRelaxWeight &
a floating point number between 0 and 1 specifying the damping factor
for BoomerAMG's damped Jacobi smoother.  The default value is 0.5.
\\
amgStrongThreshold &
a floating point number between 0 and 1 specifying the threshold used
to determine strong coupling in BoomerAMG's coasening.  The default
value is 0.25.
\\
parasailsThreshold &
a floating point number between 0 and 1 specifying the threshold used
to prune small entries in setting up the sparse approximate inverse.
The default value is 0.0.
\\
parasailsNlevels &
an integer larger than 0 specifying the desired sparsity of the
approximate inverse.  The default value is 1.
\\
parasailsFilter &
a floating point number between 0 and 1 defining the threshold used to
prune small entries in A. The default is 0.0.
\\
parasailsLoadbal &
a floating point number between 0 and 1 specifying how load balancing has 
to be done. The default is 0.0.
\\
parasailsSymmetric &
set ParaSails to take A as symmetric.
\\
parasailsUnSymmetric &
set ParaSails to take A as nonsymmetric (default).
\\
parasailsReuse &
set ParaSails to reuse the sparsity pattern of A.
\\
\hline
\end{tabular}
\caption{%
Parameters.
}
\label{table-fei-param}
\end{table}




%\label{sec:fei-param}
%
%\begin{description}
%\item[solver XXX] - where XXX specifies one of {\sf cg}, {\sf gmres},
%                    {\sf superlu}, or {\sf superlux}.  The default is {\sf gmres}.
%\item[preconditioner XXX] - where XXX is one of {\sf diagonal}, {\sf pilut},
%                    {\sf parasails}, or {\sf boomeramg}. The default 
%                    is {\sf diagonal}.
%\item[gmres-dim XXX] - where XXX is an integer specifying the value of m in
%                       restarted GMRES(m).  The default value is 50.
%\item[maxIterations XXX] - where XXX is an integer specifying the maximum number
%                           of iterations permitted for CG or GMRES.
%                           The default value is 1000.
%\item[tolerance XXX] - where XXX is a floating point number specifying the 
%                       termination criterion for CG or GMRES.  The default value is
%                       1.0E-10.
%\item[pilut-row-size XXX] - where XXX is an integer specifying the maximum
%                       number of nonzeros kept in the formation of imcomplete L
%                       and U).  If this is not called, a value will be selected
%                       based on the sparsity of the matrix.
%\item[pilut-drop-tol XXX] - where XXX is a floating point number specifying the 
%                       threshold to drop small entries in L and U.  The default
%                       value is 0.0.
%\item[superlu-ordering XXX] - where XXX specifies on of {\sf natural} or
%                       {\sf mmd} (minimum degree ordering).  This ordering
%                       is used to minimize the number of nonzeros generated
%                       in the LU decomposition.  The default is natural ordering.
%\item[superlu-scale XXX] - where XXX specifies on of {\sf y} (perform row
%                       and column scalings before decomposition) or {\sf n}.
%                       The default is no scaling.
%\item[amg-coarsen-type XXX] - where XXX specifies one of {\sf falgout},
%                       {\sf ruge}, or {\sf default} coarsening for BOOMERAMG.
%\item[amg-num-sweeps XXX] - where XXX is an integer specifying the number of
%                       pre- and post-smoothing at each level of BOOMERAMG.
%                       The default is one pre- and one post-smoothings.
%\item[amg-relax-type XXX] - where XXX is one of {\sf jacobi} (Damped Jacobi),
%                       {\sf gs-slow} (sequential Gauss-Seidel), {\sf gs-fast}
%                       (Gauss-Seidel on interior nodes), {\sf hybrid},
%                       or {\sf direct}. The default is {\sf hybrid}.
%\item[amg-relax-weight XXX] - where XXX is a floating point number between 0 and 1
%                       specifying the damping factor for BOOMERAMG's damped
%                       Jacobi smoother.  The default value is 0.5.
%\item[amg-strong-threshold XXX] - where XXX is a floating point number between 0 
%                       and 1 specifying the threshold used to determine
%                       strong coupling in BOOMERAMG's coasening.  The default 
%                       value is 0.25.
%\item[parasails-threshold XXX] - where XXX is a floating point number between 0 
%                       and 1 specifying the threshold used to prune small entries
%                       in setting up the sparse approximate inverse.  The default
%                       value is 0.0.
%\item[parasails-nlevels XXX] - where XXX is an integer larger than 0 specifying 
%                       the desired sparsity of the approximate inverse.  The
%                       default value is 1.
%\end{description}
%
%
%%\section{HYPRE-FEI Extensions}
%%
%%In order to support the reduced system formulation for applications
%%with slide surface constraints,
%%two additional functions are provided via HYPRE-FEI
%%on top of the base FEI specification :
%%
%%\begin{description}
%%\item[loadSlaveList(int,int*)] - this function is used to load information
%%                        about which unknowns are the slave nodes in the
%%                        slide surfaces.  This information is useful for
%%                        the construction of reduced linear system.
%%\item[buildReducedSystem() ] - this function can be called to reduced a linear
%%                        system with constraints to a better-conditioned reduced
%%                        form.
%%\end{description}
%
%%\section{HYPRE-FEI Comments and Contacts}
%%
%%Any comments about this HYPRE-FEI interface can be directed to Charles Tong
%%(X23411 or chtong@llnl.gov) or Edmond Chow (X31915 or chow8@llnl.gov).

