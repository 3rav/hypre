%==========================================================================
\chapter{Design}
\label{Design}

%==========================================================================
\section{Software Architecture}
\label{Software Architecture}

\subsection{More detail on the programming model}


\hypre{} is more than just a
library of solver routines: it 
is actually a collection of pieces that can be "programmed" in different ways.
Technically speaking, 
\hypre{} is a "component model." It is not necessary to have any formal training
in "components" to be able 
to use \hypre{}, however. In our experience, looking at some examples and
programming a simple example 
or two quickly gives users an intuitive feel for how this model works. In
brief, the major "mental shift" 
necessary is to accept that \hypre{} represents all of the parts of a linear
solver library as "encapsulated 
objects", called "components". This includes the obvious candidates such as
matrices, but also the less 
obvious things such as the algorithms used to implement preconditioning. Each
object is opaque and can be 
used only through the advertised "services" that it supports. All function
calls within \hypre{} are 
considered to be a function "of some object/component", and the functions that
work on a particular 
component are exactly the set of "services" that that component provides. In C
language, that component is 
always the first parameter in the function call, for example. Other parameters
in a function call are normal 
parameters, but the first one is always the component that the function call is
"acting on". The list of 
services provided by each component are easily discovered by looking at the
header files for each 
component, as well as other centralized documentation. 

In many ways, the ``component design'' of HYPRE is simpler than object-oriented
design, and it can be explained in an analogy to traditional subroutine
libraries.
As mentioned above, there are two entities in HYPRE: lists of services, and
pieces of software that implement some subset of those services.
This is a direct analog of old-fashioned procedural libraries, for example,
the Basic Linear Algebra Subroutine (BLAS): the ``lists of
services'' are the functions that a library might implement, and the pieces
of software that implement the services are the specific library implementations.
The BLAS are not an exact fit to this analogy, because typically a library
that implements part of the BLAS implements it all.
Imagine, though, that different BLAS libraries implemented only the portion
of the BLAS that the developers chose to implement, either due to manpower,
area of expertise, interest, need, etc.
As a user, the way you would use such a system would be to figure out which
functions you want to call (i.e. which services you would like to use), make
a call to that routine in your program, and then look at all of the available
BLAS implementations to find the ones that have implemented the call that you
made and choose one.
One way to think about HYPRE is exactly analagous: the BLAS functions are 
the analog of the abstract interfaces within HYPRE, and the analog of a BLAS
library is a particular HYPRE component.

Besides for the services that a component provides, there is only one other use
of a given component within 
\hypre{}, and that is as an input to other components. For example, \hypre{} provides
several components 
that implement preconditioning algorithms. A user may very well set up a
preconditioner but never call any 
services from the preconditioner directly. Instead, the preconditioner is
handed to a solver within \hypre{} 
and the solver uses that preconditioner. Thus, the way to decide if you want to
use a particular component 
is to ask "do I want to use any of its services" and "do I need it as input to
another component that I want to 
use?" If the answer to either question is "yes", then you need that component.

The components are then put together as inputs to each other to collaborate on
performing the desired 
action. The "desired action" is typically the solution of a single linear
system of equations and \hypre{} is 
specialized to make that extremely easy, but one of the benefits of \hypre{}'s
programming model is that 
the parts can be combined together in different ways to perform different
actions. Examples might be: 
reusing preconditioners over several timesteps of a time-dependent calculation,
using a solver as a 
preconditioner, or providing user-defined matrices or preconditioners for use
within \hypre{}. Generally, 
these more complex usages are performed by more expert users, including
developers of other solver 
components.


%==========================================================================
\section{Object Model}
\label{Object Model}


{\bf Abstract versus concrete types:} 

In the current version of HYPRE, we use a capitalization scheme
to differentiate types in 
\hypre{}: capital \code{HYPRE} refers to abstract classes, that is, services,
while lower \code{hypre} refers to concrete classes. 
If you
declare something to be of 
type \code{HYPRE_Solver}, for instance (see declaration of components below), it is
equivalent to stating that you 
will be using a component that supports the services listed in the 
\code{HYPRE_Solver}
class. On the other hand, 
lowercase \code{hypre} indicates a specific component, that is, a "concrete class".
The \code{hypre_StructSMGSolve}r, 
for instance, is a particular component. Just as in object-oriented libraries,
"upcasting" is always allowed. 
This means that in the bulk of your program code, you can refer to components
as if they were the abstract 
types. Only in the declaration of the component is the concrete class
mentioned. This technique is 
invaluable for promoting plug and play. In this example, then, you could
declare 

\begin{display}
\begin{verbatim}

HYPRE_Solver A; 
hypre_StructSMGSolver a; 
A = (HYPRE_Solver) a; 

\end{verbatim}
\end{display}

Everywhere else in the code, you
can refer to \code{a} as if it 
was an \code{HYPRE_Solver}. Then, should you decide you would like to, say, 
use the PFMG solver
instead of SMG (these 
algorithms are described in more detail in the "Solvers Available" section),
the only line of code that needs 
to change is 

\begin{display}
\begin{verbatim}

hypre_StructPFMGSolver a;

\end{verbatim}
\end{display}

In this way, users can isolate any
potential changes to a single 
place in their code.

\subsection{The life cycle of a component}

Every encapsulated object or component in \hypre{} has the same basic "life
cycle", and many user 
problems can be prevented simply by understanding this cycle and making sure
that each component has 
had all of the proper steps taken in the user code.

\begin{display}
\begin{verbatim}

Add an example here to use for illustration of the exact sequence and syntax of
steps.

\end{verbatim}
\end{display}

\begin{enumerate}

\item
{\bf Declaration of the "handle" to the component.}  Each \hypre{} component is
represented by a 
"handle", and users must declare these handles just like any other data. They
can either be declared
statically or through dynamic memory mechanism such as \code{MALLOC} in C language,
and the same 
scope rules apply to \hypre{} components as to other data types (though see the
note below on 
reference counting). Though this information is technically "opaque" to users,
generally \hypre{} 
handles are either pointers or integers, and thus, these handles are safe to be
passed around by users in 
parameter lists without worrying about excessive copying going on underneath. 
The handles are all of the form
\code{HYPRE_Service}, where 
\code{Service} is replaced by the name of the family of services that you want your
\hypre{} object to 
provide. For example, a \code{HYPRE_Solve}r is a component that provides solver
services.

\item
{\bf Binding the handle to a concrete type.} Declaration of the handle by

\begin{display}
\begin{verbatim}

HYPRE_Solver solver; 

\end{verbatim}
\end{display}

says the 
following to \hypre{}: "I will be using a \hypre{} component that provides the Solver
service, and I will 
be calling that component "solver"". It does not, however, tell \hypre{} exactly
which component that 
provides the Solver service "solver" should refer to. The most common way for
this information to be 
declared is to set the handle equal to the handle of a specific component. For
example, 

\begin{display}
\begin{verbatim}

HYPRE_Solver solver = (HYPRE_Solver) hypre_StructSMGSolver;

\end{verbatim}
\end{display}

In the OO world, this is the
same thing as 
"instantiating the concrete type", while the service is the "abstract
type". The benefit of this 
system is that the choice of concrete type appears exactly once in the user's
code, and that everywhere 
else, the component is accessed knowing only the services it provides. This is
the mechanism that 
enables "plug and play": a user can switch solvers (or other components) by
changing a single line of 
their code (in fact, mechanisms can be set up to allow runtime switching; this
will be discussed later) 
because everywhere else in the code the components are used through the "common
interfaces" 
defined by the appropriate \hypre{} service.

\item
{\bf Bring the component "to life" through a "new" call.} The first call that must
be made on every 
\hypre{} component is a \code{new} call (though see the section on "Construction
Components" below), as 
in 

\begin{display}
\begin{verbatim}

return_code = HYPRE_Create( solver );

\end{verbatim}
\end{display}

Essentially, these routines allocate
space for the object and 
set up defaults.

\item
{\bf Set parameters.  Important note:} all parameters have reasonable defaults
that will be used if not 
explicitly set by the user.

\item
{\bf Pass in needed information for construction of the component.} The
information required depends 
on the component. Matrices need the coefficients that define the matrix; these
are passed in through 
repeated calls to the chosen conceptual interface. Preconditioners need a
"matrix" that provides the 
necessary access pattern service. Preconditioned solvers need a preconditioner.

\item
{\bf Construct the object.} After the necessary construction information has been
passed in, \hypre{} must 
be instructed to construct the object, currently through the Setup call as in

\begin{display}
\begin{verbatim}

HYPRE_Setup (solver);

\end{verbatim}
\end{display}

\item
{\bf Use the object.} After construction, the object is ready to be used through
its advertised services, or to 
be handed to other components as parameters. Matrices can be used to do
matrix-vector multiplication, 
or be given to solvers/preconditioners (for their construction); solvers can be
used to solve systems, or 
handed to other solvers as preconditioners; etc.

\item
{\bf "Kill" the object.} This is the opposite of the "bringing to life" phase.
Here, the call is to "Free" the 
object as in 

\begin{display}
\begin{verbatim}

HYPRE_Destroy( solver );

\end{verbatim}
\end{display}

NOTE: \hypre{} uses reference counting to
manage memory, and 
thus \code{HYPRE_Free} does not actually deallocate the object unless this was the
last remaining reference 
to the object. This allows users to safely free an object in one part of the
code without worrying about 
whether it is still being used by some other objects that are still alive.

\end{enumerate}

\subsection{Construction versus Use}

All of the \hypre{} interfaces (or sets of services) can be divided into two
categories: {\bf construction} 
interfaces or {\bf use} interfaces. Every concrete component in \hypre{} 
(designated by
\code{hypre}) must support at 
least one construction interface (note: if a component supports more than one
construction interface, you 
cannot mix and match calls from those interfaces. One and only one interface
must be chosen and called, or 
else errors may occur) and may support any number of use interfaces. In terms
of the lifecycle given above, 
"construction" is steps 2-6, and "use" is step 7. Thus, for a given concrete
component, to figure out how to 
perform steps 2-6 on it, you must look up the \hypre{} construction interface that
it supports, and to see 
which operations you may call on it in step 7, you must look up the use
interfaces that that component 
supports.

\subsubsection{Motivation} The motivation for separate construction and use interfaces in
\hypre{} is flexibility, both for 
users and the algorithm developers that contribute algorithms to both \hypre{} and
libraries that are 
compatible with \hypre{} (i.e. "ESI compliant" libraries). The separation is
motivated by the observation 
that different objects might be constructed through the same construction
process and thus it is not possible 
to mandate, for example, that all objects built through the StructuredGrid
interface should necessarily be of 
a particular type. Indeed, there is a good example: the StructuredGrid's most
natural function is to produce 
a matrix that then can be input to solvers, but there are also solver writers
who would like to produce 
solvers directly from the StructuredGrid interface because this allows them to
control construction of the 
matrix in a way that is optimized for their solver. In the second case, then,
the component implements not only the 
\code{StructuredGrid} construction interface, but also the \code{Solver} use interface.

\subsubsection{Construction Components}

There is a subset of components within \hypre{} whose main function is simply to
build other components. 
We call these "construction" components. These components are easily
recognizable: they are the 
components that provide the \code{Build} service whose major function is
\code{GetConstructedObject}. They are 
also easily explainable in the context of the component lifecycle:
a construction component A 
handles steps 2-6 for component B, and the "use" of component A (i.e. step 7)
is to return component B (in 
a state where it is ready to be used). Thus, if component B is built through a
construction component A, 
steps 2-6 of B are done by A and the user does not have to do them explicitly.
An example speaks a 
thousand words:

\hypre{} has several components that build matrices, such as the
\code{hypre_StructGridStructMatrix} builder 
component. This component implements the \code{HYPRE_StructGridInput} interface, but
its only "use" function 
is the Builder interface with the \code{GetConstructedObject} function. Clients use
this component in the normal 
way in steps 1-6. However, when step 7 is reached, the user calls
\code{GetConstructedObject} and is returned a 
second component B. For the component B, the user still has to perform step 1,
the declaration of the 
handle to B. Steps 2-6 are performed by the \code{hypre_StructGridStructMatrix}
component for B, however. 
When B is returned from the component, it is ready to be used. In this case,
the returned object is of type 
\code{hypre_StructMatrix}, and by looking up its headers we see that it supports the
matrix-vector multiplication 
use. It can also be passed as a parameter into various solvers such as
\code{hypre_StructSMG} and 
\code{hypre_StructPFMG}.


\section{User-defined components (experts only)}

