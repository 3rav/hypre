\section{\hypre{} Solvers Available through the FEI} 

On the other side of the {\sf LinearSystemCore} interface is the linear
solver package. Any linear solver package can be used as long as there
is an implementation of the interface in the solver package itself. The
{\sf LinearSystemCore} implementation in \hypre{} is encapsulated in the 
\hypre{} finite element conceptual interface called {\sf HYPRE\_LinSysCore}.
The incoming element stiffness matrices and other information are 
converted into \hypre{}'s internal data structures. In this section we
briefly describe some of these internal data structures and capabilities.

\subsection{Parallel Matrix and Vector Construction} 

The matrix class in \hypre{} accessible via the {\sf LinearSystemCore} interface
is the parallel compressed sparse row ({\sf ParCSR}) matrix.  The 
requirements about how the global matrix is partitioned among the
processors are that each processor holds a contiguous block of rows and columns
and the equation numbers in processors of lower rank are lower than those
in processors of higher rank.  The {\sf FEI} is responsible for ensuring
that these two requirements are followed. The matrix can be loaded in
parallel - a row or a block of rows at a time.  The solution and right
hand side vectors are constructed accordingly. The matrix rows corresponding
to the shared nodes can be assigned to either processor, and is determined
by the {\sf FEI} itself. Once the incoming matrix and vector data have
been captured in the \hypre{} {\sf ParCSR} format, a whole of matrix and
vector operators are available for use in the \hypre{} solvers.

\subsection{Sequential and Parallel Solvers} 

\hypre{} currently has many iterative solvers. There is also internally a
version of the sequential {\sf SuperLU} direct solver (developed at U.C.
Berkeley) suitable to small problems (may be up to the size of $10000$).
In the following we list some of these internal solvers.

\begin{enumerate}
\item Krylov solvers (CG, GMRES, FGMRES, TFQMR, BiCGSTAB, symmetric QMR),
\item Boomeramg (a parallel algebraic multigrid solver),
\item SuperLU direct solver (sequential),
\item SuperLU direct solver with iterative refinement (sequential), 
\end{enumerate}

\subsection{Parallel Preconditioners} 

The performance of the Krylov solvers can be improved by clever selection
of preconditioners. \hypre{} currently has the following preconditioners 
available via the {\sf LinearSystemCore} interface: 

\begin{enumerate}
\item no preconditioning,
\item diagonal, 
\item another parallel incomplete LU ({\sf Euclid}),
\item parallel sparse approximate inverse ({\sf ParaSails}),
\item parallel algebraic multigrid ({\sf BoomerAMG}),
\item another parallel algebraic multigrid ({\sf MLI} based on
the smoothed aggregation multigrid methods),
\item parallel domain decomposition with inexact local solves ({\sf DDIlut}), 
\item least-squares polynomial preconditioner,
\item $2 \times 2$ block preconditioner, and
\item $2 \times 2$ Uzawa preconditioner.
\end{enumerate}

Some of these preconditioners can be tuned by a number of internal parameters
modifiable by users. A description of these parameters is given in later
sections.

\subsection{Matrix Reduction} 

For some structural mechanics problems with multi-point constraints the 
discretization matrix is indefinite (eigenvalues lie in both sides of
the imaginary axis). Indefinite matrices are much more different to solve
than definite matrices. Methods have been developed to reduce these
indefinite matrices to definite matrices.  Two matrix reduction algorithms
have been implemented in \hypre{}, as presented in the following subsections.

\subsubsection{Schur Complement Reduction}

The incoming linear system of equations is assumed to be in the form :
\[ \left[ 
\begin{array}{cc} 
   D   & B \\
   B^T & 0
\end{array}
  \right] 
  \left[
\begin{array}{c} 
   x_1 \\
   x_2
\end{array}
  \right] 
  =
  \left[
\begin{array}{c} 
   b_1 \\
   b_2
\end{array}
  \right] 
\]
where $D$ is a diagonal matrix.  After Schur complement reduction is applied, 
the resulting linear system becomes
$$
- B^T D^{-1} B x_2 = b_2 - B^T D^{-1} b_1.
$$

\subsubsection{Slide Surface Reduction}

With the presence of slide surfaces, the matrix is in the same form as in the
case of Schur complement reduction.  Here $A$ represents the relationship 
between the master, slave, and other degrees of freedom.  The matrix block 
$[B^T 0]$ corresponds to the constraint equations.  The goal of reduction
is to eliminate the constraints.  As proposed by Manteuffel, the trick is
to re-order the system into a $3 \times 3$ block matrix.
\[ 
\left[ 
\begin{array}{ccc} 
   A_{11}  & A_{12} & N \\
   A_{21}  & A_{22} & D \\
   N_{T}   & D      & 0 \\
\end{array}
\right] 
=
\left[ 
\begin{array}{ccc} 
   A_{11}       & \hat{A}_{12} \\
   \hat{A}_{21} & \hat{A}_{22}.
\end{array}
\right] 
\]
The reduced system has the form :
$$
(A_{11} - \hat{A}_{21} \hat{A}_{22}^{-1} \hat{A}_{12}) x_1 =
b_1 - \hat{A}_{21} \hat{A}_{22}^{-1} b_2,
$$
which is symmetric positive definite (SPD) if the original matrix is PD.
In addition, $\hat{A}_{22}^{-1}$ is easy to compute. 

There are three slide surface reduction algorithms in \hypre{}.  
The first follows the matrix formulation in this section.  The second is
similar except that it replaces the eliminated slave equations with
identity rows so that the degree of freedom at each node is preserved.
This is essential for certain block algorithms such as the smoothed
aggregation multilevel preconditioners.
The third is similar to the second except that it is more general and 
can be applied to problems with intersecting slide surfaces (sequential only
for intersecting slide surfaces).

\subsection{Other Features} 

To improve the efficiency of the \hypre{} solvers, a few other features have been
incorporated.  We list a few of these features below :

\begin{enumerate}
\item Preconditioner reuse - For multiple linear solves with matrices that are
      slightly perturbed from each other, oftentimes the use of the same 
      preconditioners can save preconditioner setup times but suffer little
      convergence rate degradation.
\item Projection methods - For multiple solves that use the same matrix,
      previous solution vectors can sometimes be used to give a better initial
      guess for subsequent solves.  Two projection schemes have been implemented
      in \hypre{} - A-conjugate projection (for SPD matrices) and minimal residual
      projection (for both SPD and non-SPD matrices).
\item The sparsity pattern of the matrix is in general not destroyed after
      it has been loaded to an \hypre{} matrix.  But if the matrix is not to
      be reused, an option is provided to clean up this pattern matrix to
      conserve memory usage.
\end{enumerate}
