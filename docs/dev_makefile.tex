
%==========================================================================
\chapter{The HYPRE Build System}
\label{The HYPRE Build System}


The \hypre{} build system uses GNU Autoconf and a few related scripts to 
create a \file{configure} script. 


configure, make and install the library.  These tools are very inter-
dependent and need to be executed in a specific order. A shell script named 
\file{bootstrap}, is provided in the \file{config} subdirectory to relieve 
developers of needing to know the required versions or constraints.  It is run 
whenever changes are made to the M4 files or the \file{configure.in} script.

%==========================================================================
\section{Makefile Requirements}
\label{Makefile Requirements}


There is one file stored in the \file{config} subdirectory, named \file{Makefile.config.in},



Autotools expect a file named \file{Makefile.am} to be in every direcotry, whether
or not there are source code files.  \file{Makefile.am} is used by automake as a
template which contains the names of the source and include files that need to 
be built as well as include file paths and macro flags.  The source file names 
are added to the \file{FILES} list; include files to be installed are in the 
\file{include_HEADERS} list; and include paths and macro flags (i.e. -Dxxx) are 
in the AM\_CPPFLAGS list for C and C++ compilers or the AM\_FFLAGS list for Fortran
compilers.  These preprocessor flags are global to the build system.  The 
EXTRA\_DIST list is for those files not needed for building \hypre{} but desired 
for its distribution, such as a README file. 

The top level \file{Makefile.am}, i.e. \file{linear_solvers/Makefile.am}, is 
edited to add the name of the new (highest\_level) subdirectory, excluding all of
its child-subdirectories, to the SUBDIRS, HYPRE\_SUBDIR\_LIST and possibly 
DIST\_SUBDIRS lists.  The rest of this file is not likely to need changing.

Whenever changes to a \file{Makefile} are needed, they MUST be made to the \file{Makefile.am}
files---NOT to any other \file{Makefile}.
\file{Makefile.in} and \file{Makefile} are regenerated everytime \file{bootstrap}
or \file{configure} are run.

%==========================================================================
\subsection{Sample Makefile.am with Source Code}
\label{Sample Makefile.am with Source Code}

\begin{verbatim}
## Process this file with automake to create Makefile.in

include_HEADERS = krylov.h
noinst_HEADERS = \
   all_krylov.h \
   pcg.h
dependent_includes = $(include_HEADERS) $(noinst_HEADERS)

AM_CPPFLAGS = \
   -I$(top_builddir)/utilities \
   -I$(top_srcdir)/utilities 

FILES = \
   HYPRE_pcg.c \
   cgnr.c \
   gmres.c \
   pcg.c

$(OBJECTS): $(dependent_includes)

\end{verbatim}

%==========================================================================
\subsection{Sample Makefile.am without Source Code}
\label{Sample Makefile.am without Source Code}

\begin{verbatim}
## Process this file with automake to create Makefile.in

SUBDIRS = . LOBPCG

\end{verbatim}

%==========================================================================
\section{Configure Script Requirements}
\label{Configure Script Requirements}


The GNU tool Autoconf is used to generate the \file{configure} script for automatically
configuring \hypre{} on UNIX-based systems.  It requires GNU M4 to generate the
scripts.  There are several M4 scripts in the \file{linear_solvers/config} directory
that are customized for \hypre{}.

\file{configure.ac} is the template file, located in the \file{config} subdirectory, used 
by Autoconf to create the \file{configure} script.  It is the file that gets changed
when new subdirectories are added.  The names of new subdirectories are added to
the AC\_CONFIG\_FILES list (near the end of the file).

For example, a developer creates subdirectory \file{new_solve} with its own 
subdirectories of \file{solve_setup} and \file{solve_exec}.  The additions to 
AC\_CONFIG\_FILES would be (back slashes indicate continuation lines):

\begin{verbatim}
    new_solve/Makefile \
    new_solve/solve_setup/Makefile \
    new_solve/solve_exec/Makefile \

\end{verbatim}

%==========================================================================
\section{Bootstrap Script}
\label{Bootstrap Script}


The GNU Autotools are very inter-dependent, needing to be run in a specified 
order.  The \file{bootstrap} script is a wrapper for executing them properly, 
thus relieving developers of needing to know the details.

\file{bootstrap} must be run anytime there are changes to any of the M4 routines
or \file{configure.ac} to generate a new \file{configure} script.  The new configure
script must be checked into the \hypre{} repository along with any other new or
modified files.

To run \file{bootstrap} you must be in the top-level directory, which is usually the
\file{linear_solvers} directory.  The execute line is: \file{./config/bootstrap} 

%==========================================================================
\section{Build and Install}
\label{Build and Install}


Now that all of the files are created/edited, \hypre{} is ready to be configured
and made for the host platform.  The simplest method is to configure, compile and
install the libraries in \file{hypre/lib} and \file{hypre/include} directories, which is
accomplished by:
\begin{verbatim}
   ./configure
   make install
\end{verbatim}

Of course there are many options to \file{configure} and \file{make} targets to 
customize such things as installation directories, compilers used, compile and
load flags, etc.  

%==========================================================================
\subsection{Configure}
\label{Configure}


The best way to find out what options are available is to display the help package.
The results of \file{./configure --help} when run on a local Linux platform are:
\begin{verbatim}
'configure' configures hypre 1.8.xx to adapt to many kinds of systems.

Usage: ./configure [OPTION] ... [VAR=VALUE]...

To assign environment variables (e.g. CC, CFLAGS ...), specify them as
VAR=VALUE.  See below for descriptions of some of the useful variables.

Defaults for the options are specified in brackets.

Configuration:
   -h, --help		   display this help and exit
       --help=short	   display options specific to this package
       --help=recursive	   display the short help of all the included packages
   -V, --version	   display version information and exit
   -q, --quiet, --silent   do not print 'checking ...' messages
       --cache-file=FILE   cache test results in FILE [disabled]
   -C, --config-cache      alias for '--cache-file=config.cache'
   -n  --no-create         do not create output files
       --srcdir=DIR        find the sources in DIR [configure dir or '..']

Installation Directories:
   --prefix=PREFIX         install architecture-independent files in PREFIX
                           [/home/hill66/linear_solvers/hypre]
   --exec-prefix=EPREFIX   install architecture-dependent files in EPREFIX 
                           [PREFIX]


By default, 'make install' will install all files in
/home/hill66/linear_solvers/hypre/lib, /home/hill66/linear_solvers/hypre/bin, etc.

You can specify an installation prefix other than /home/hill/linear_solvers/hypre by using 
the --prefix option; for instance --prefix=$HOME.


For fine tuning of the installation directories:
   --bindir=DIR          user executables [EPREFIX/bin]
   --sbindir=DIR         system admin executables [EPREFIX/sbin]
   --libexecdir=DIR      program executables [EPREFIX/libexec]
   --datadir=DIR         read-only architecture independent data [PREFIX/share]
   --sysconfdir=DIR      read-only single machine data [PREFIX/etc]
   --sharedstatedir=DIR  modifiable architecture independent data [PREFIX/com]
   --localstatedir=DIR   modifiable architecture dependent data [PREFIX/var]
   --libdir=DIR          object code libraries [EPREFIX/lib]
   --includedir=DIR      C header files [PREFIX/include]
   --oldincludedir=DIR   C header files for non GCC [/usr/include]
   --infodir=DIR         info documentation [PREFIX/info]
   --mandir=DIR          man documentation [PREFIX/man]


Program Names:
  --program-prefix=PREFIX           prepend PREFIX to installed program names
  --program-suffix=SUFFIX           append SUFFIX to installed program names
  --program-transform-name=PROGRAM  run sed PROGRAM on installed program names


System Types:
   --build=BUILD  configure for building on BUILD [guessed]
   --host=HOST    cross-compile to build programs to run on HOST [BUILD]


Optional Features:
   --disable-FEATURE             do not include FEATURE (same as --enable-FEATURE=no)
   --enable-FEATURE[=ARGS]       include FEATURE [ARG=yes]
   --enable-shared[=PKGS]        build shared libraries [default=no]
   --enable-maintainer-mode      enable make rules and dependencies not useful
                                 (sometimes confusing) to the casual installer
   --enable-debug                compile for debugging
   --enable-static[=PKGS]        build static libraries [default=yes]
   --enable-fast-install[=PKGS]  optimize for fast installation [default=yes]
   --disable-libtool-lock        avoid locking (might break parallel builds)


Optional Packages:
   --with-PACKAGE[=ARG]    use PACKAGE [ARG=yes]
   --without-PACKAGE       do not use PACKAGE (same as --with-PACKAGE=no)
   --with-MPI=DIR          compiles with MPI default; DIR is top-level MPI directory
   --without-MPI           use may affect compiler choice
   --with-timing           use HYPRE timing routines
   --with-openmp           use OpenMP; may affect compiler choice; supported 
                           using guidec on IBM and Compaq
   --with-strict-checking  compiles without MPI and assigns KCC as the C and C++
                           compilers, unless CC and CXX are already set to gcc 
                           and g++.
   --with-insure=FLAGS     FLAGS are optional to pass to insure
   --with-purify=FLAGS     FLAGS are optional to pass to purify
   --with-purify-to-file   direct purify output to file 'purify.log'
   --with-babel            use babel
   --with-libdir           build libHYPRE and libHYPRE_LSI in lib directory
   --with-FEI_BASE_DIR     specify the directory where the fei header files live
   --with-FEI_LIB_DIR      specify an additional directory for fei library installs
   --with-FEI_INC_DIR      specify an additional directory for fei include installs
   --with-gnu-ld           assume the C compiler uses GNU ld [default=no]
   --with-pic              try to use only PIC/non-PIC objects [default=use both]
   --with-tags[=TAGS]      include additional configurations [automatic]
   --with-COMM_SIMPLE      do not use MPI derived data types.  This option is 
                           automatically chosen for IBM, may be selected for other
                           platforms as well.
   --with-blas=<lib>       uses BLAS library <lib>
   --with-lapack=<lib>     uses LAPACK library <lib>


Some influential environment variables:
   CC        C compiler command
   CFLAGS    C compiler flags
   LDFLAGS   linker flags, e.g. -L<lib dir> for libraries in non-standard directory 
             <lib dir>
   CPPFLAGS  C/C++ preprocessor flags, e.g. -I<include dir> for header files in
             non-standard directory <include dir>
   CPP       C preprocessor
   CXX       C++ compiler command
   CXXFLAGS  C++ compiler flags
   F77       Fortran 77 compiler command
   FFLAGS    Fortran 77 compiler flags
   CXXCPP    C++ preprocessor

Use these variables to override the choices made by 'configure' or to help
it find libraries and program swith non-standard names/locations.

Report bugs to <http://www-casc.llnl.gov/bugs/enter_bug.cgi?product=HYPRE>.
\end{verbatim}


As an example, to configure \hypre{} on an AIX platform to build 64-bit libraries
the command line would be:
\begin{verbatim}
./nopoe ./configure \
CC=mpcc  CXX=mpCC  F77=mpxlf \
CFLAGS="-q64 -qmaxmem=8192 -DHYPRE_COMM_SIMPLE -O3 -qstrict" \
CXXFLAGS="-q64 -qmaxmem=8192 -DHYPRE_COMM_SIMPLE -O3 -qstrict" \
FFLAGS="-q64 -O3 -qstrict" \
LDFLAGS="-q64" \
AR="ar -X64"
\end{verbatim}

%==========================================================================
\subsubsection{Sample Output from configure}
\label{Sample Output from configure}

The output of the configure script is several pages long.  It reports on the 
system type being used for building and executing, compilers being used, libraries
being searched and option flags being set.  When all of the searching is done a 
file named \file{config.status} is created which then proceeds to create all of 
the Makefiles.  The following is a very short sample of \file{configure} output.
\begin{verbatim}
./configure
checking build system type. . . i686-pc-linux-gnu
checking host system type. . . i686-pc-linux-gnu
checking for mpcc. . . no
checking for mpikcc. . . no
checking for mpicc. . . mpicc
checking for mpCC. . . no
checking for mpiKCC. . . no
checking for mpiCC. . . mpiCC
checking for mpxlf. . . no
checking for mpikf77. . . no
checking for mpif77. . . mpif77
checking whether the C compiler works. . . yes
checking for egrep. . . grep -E
checking for sys/types.h. . . yes
checking for float.h. . . yes
checking for _Bool. . . no
checking whether closedir returns void. . .  no
checking for sgemm_ in -lessl. . .  no
configure: creating ./config.status
config.status: creating Makefile
config.status: creating parcsr_es/Makefile
config.status: creating parcsr_es/LOBPCG/Makefile
config.status: creating HYPRE_config.h
config.status: HYPRE_config.h is unchanged

\end{verbatim}

%==========================================================================
\subsection{Make}
\label{Make}

The make step in building \hypre{} is where the compiling, loading and creation 
of libraries occurs.  As with \file{configure}, make has options which are called
targets.  These include:
\begin{verbatim}
   all                  default action; makes all top-level targets
   clean                undo whatever make does
   mostlyclean          like clean, but leaves libraries
   maintainer-clean     delete everything that can be rebuilt
   distclean            undo wahtever configure does
   install              copy (install) executables/libraries in appropriate directories
   install-strip        strip the installed files
   uninstall            undo whatever install does
   dist                 create a distribution file(tar file) of the source files
   test                 perform a self-test on what make created
\end{verbatim}

Refer to the \file{Make} documentation or man pages for more detailed information. 
Typically, \file{make} and \file{make install} are the only options needed for building \hypre{}.

%==========================================================================
\subsubsection{Sample Output from Make}
\label{Sample Output from Make}

The execution of \file{make} and its targets creates several pages of output.  
For every file in every (sub)directory, the compile or isntall line is printed.
The following is a brief sample of compile and install.
\begin{verbatim}

Output from make:
Making all in LOBPCG
make[3]: Entering directory '/home/hill66/linear_solvers/parcsr_es/LOBPCG
mpicc -DHAVE_CONFIG_H -I. -I. -I../.. -I../../utilities -I../../utilities 
-I../../IJ_mv -I../../parcsr_mv -I../../parcsr_ls -I../../seq_mv -I../../krylov
-O2 -c lobpcg.c

mpicc -DHAVE_CONFIG_H -I. -I. -I../.. -I../../utilities -I../../utilities 
-I../../IJ_mv -I../../parcsr_mv -I../../parcsr_ls -I../../seq_mv -I../../krylov
-O2 -c lobpcg_matrix.c

mpicc -DHAVE_CONFIG_H -I. -I. -I../.. -I../../utilities -I../../utilities 
-I../../IJ_mv -I../../parcsr_mv -I../../parcsr_ls -I../../seq_mv -I../../krylov
-O2 -c lobpcg_utlities.c
rm -f libHYPRE_lobpcg.a
ar cru libHYPRE_lobpcg.a lobpcg.o lobpcg_matrix.o lobpcg_utilities.o
ranlib libHYPRE_lobpcg.a
make[3]: Leaving directory '/home/hill66/linear_solvers/parcsr_es/LOBPCG
 


Output from make install:
make[1]: Entering directory '/home/hill66/linear_solvers/parcsr_es
Making install in LOBPCG
make[2]: Entering directory '/home/hill66/linear_solvers/parcsr_es/LOBPCG
make[3]: Entering directory '/home/hill66/linear_solvers/parcsr_es/LOBPCG
mkdir -p -- . /home/hill66/linear_solvers/hypre/lib
  /usr/bin/install -c -m 644 libHYPRE_lobpcg.a /home/hill66/linear_solvers/hypre/lib/libHYPRE_lobpcg.a
  ranlib /home/hill66/linear_solvers/hypre/lib/libHYPRE_lobpcg.a
  /usr/bin/install -c -m 644 HYPRE_lobpcg.h /home/hill66/linear_solvers/hypre/include/HYPRE_lobpcg.h
  /usr/bin/install -c -m 644 lobpcg.h /home/hill66/linear_solvers/hypre/include/lobpcg.h
make[3]: Leaving directory '/home/hill66/linear_solvers/parcsr_es/LOBPCG
make[2]: Leaving directory '/home/hill66/linear_solvers/parcsr_es/LOBPCG
make[1]: Leaving directory '/home/hill66/linear_solvers/parcsr_es

\end{verbatim}

%==========================================================================
\section{Testing the Changes}
\label{Testing the Changes}

After having run \file{./configure}, the following sequence of commands will create 
the \hypre{} libraries, install them in the proper directories and build the test codes.
\begin{verbatim}
make test                             ## compiles, builds and installs the libraries
cd test                               ## go to the test subdirectory
./runtest.sh TEST_xx.sh/*.sh          ## execute the runtest script to test all 
                                         scripts in subdirectory TEST_xx; where 
                                         xx can be ij, ij_es, fei, sstruct, struct.
\end{verbatim}

%==========================================================================
\section{Committing Changes to the Repository}
\label{Committing Changes to the Repository}

After verifying all of the changes, the files must be committed to the CVS repository.
This is usually done from the top-level (i.e. \file{linear_solvers}) directory with 
the CVS command:
\begin{verbatim} cvs commit * 
\end{verbatim}

If there are only a few files to be committed the command can be executed from 
any subdirectory with the file names listed on the command line, such as:
\begin{verbatim} cvs commit src1.c src2.c src.h
\end{verbatim}
