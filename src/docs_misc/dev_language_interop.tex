%==========================================================================
\chapter{Language Interoperability}
\label{Language Interoperability}

%==========================================================================
\section{Zero-based vs. One-based Indexing in Matrices}
\label{Zero-based vs. One-based Indexing in Matrices}

Some definitions:
\begin{description}
\item[0-based indexing] Row and column indices are numbered starting at 0.
\item[1-based indexing] Row and column indices are numbered starting at 1.
\item[0-based storage]  Natural for C, possible in Fortran.
\item[1-based storage]  Natural for Fortran, sometimes used in C.
\end{description}

%==========================================================================
\subsection{The issues and a suggested decision}
\label{The issues and a suggested decision}

The issues are as follows:
\begin{enumerate}
  \item We want the code to be natural to write.
  \item We want to interoperate with other codes as much as possible.
  \item We want the code to be as efficient as possible.
\end{enumerate}

In a later section, we describe four different ways to write one
sample code in C.  The sample code does a simple matvec, allocates an
int array that will mark the coarse grid variables in a multigrid
scheme, directly calls a Fortran subroutine to fill in the coarse grid
array (we might typically have a C wrapper around the Fortran
subroutine, but the intention here is to explicitely demonstrate the
parameter translation required to interface with the routine), then
de-allocates the coarse grid array.

The \code{A} and \code{B} examples use 0-based and 1-based indexing
with no pointer tricks.  Example \code{C} uses 1-based indexing with a
pointer trick that is left fully exposed for illustrative purposes.
Example \code{D} supports both 0 and 1-based indexing and uses a
\code{hypre_BaseArray} class (see the following section for details on
this class) to hide the tricks illustrated in example C.  In these
examples, it is assumed that the Fortran subroutine provides the same
indexing support as the calling C-routines.

{\bf Issue 1:} Example \code{A} is the most straightforward and least
confusing of all the examples.  That is, 0-based indexing is the most
natural for C programming.  However, examples \code{C} and \code{D}
are not all that bad either.  Example \code{B} is horribly confusing.
The most natural style for Fortran is 1-based, but a Fortran variant
of examples \code{C} and \code{D} for 0-based indexing would also be
fairly natural.  Arrays would be declared like this:
\begin{verbatim}
     integer  Ai(base:An+base)
\end{verbatim}
and the \code{iiA} loop would look like this:
\begin{verbatim}
     do iiA = base, An-1+base
\end{verbatim}

{\bf Issue 2:} We cannot dictate what other code groups support regarding
indexing.  So, if we want to interoperate with other codes, we need to
support both 0-based and 1-based indexing as in example \code{D} below.

{\bf Issue 3:} To keep efficiency high, we do NOT want to have to modify the
data in the \code{A_i} and \code{A_j} arrays.  That is, we want to use the
pointer tricks below to interoperate with other codes.  However, this
means that the whole code does EITHER 0-based OR 1-based indexing
during any single run, and NOT both.

Suggested decision: Support both 0-based and 1-based indexing.

%==========================================================================
\subsection{Hiding the details}
\label{Hiding the details}

Although code for supporting both 0-based and 1-based indexing is not
too ugly even when the tricks are exposed, it is easier to make
mistakes and/or get confused during code development and debugging.
In an attempt to minimize this, we introduce below a
\code{hypre_BaseArray} class and a set of coding rules and guidelines.

{\noindent\bf The BaseArray class:}
\begin{verbatim}
Types:
  hypre_IntBaseArray, hypre_DoubleBaseArray, ...

  Note: These types are just typedefs for `int' and `double', so
  compilers will not treat them at distinct types (strong typing).
  It would be nice to have compilers do strong typing on them
  to help reduce mistakes, but it is apparantly not possible to
  do this in C *and* retain the ability to use the `[]' operators
  and do casting.

Constructor/destructor macros:
  base_array = hypre_AllocBaseArray(type, size, base)
  hypre_FreeBaseArray(base_array, base)

Macro that returns a pointer to the real data:
  base_array_data = hypre_BaseArrayData(base_array, base)

Macro that converts from 1 base index to another
  new_index = hypre_ConvertBaseIndex(index, base, new_base)

Function that converts from 1 base to another (This function actually
modifies the data in the array, so it should be used only when necessary):
  hypre_ConvertBaseArray(&base_array, base, new_base)
\end{verbatim}

{\noindent\bf Coding Rules and Guidelines:}
\begin{itemize}

\item All arrays with array indices that refer to either rows/columns
of matrices or elements of vectors must be \code{hypre_BaseArray} types.
This reduces the possibility of making coding errors.

\item All \code{hypre_BaseArray} types in the same routine should use the
same base.  It is recommended that only one `base' variable be defined
for each routine in an attempt to enforce this.

\item Care must be taken when calling other codes that depend on the
matrix and vector base to convert (if necessary) to a base that the
called code supports.

\end{itemize}

%==========================================================================
\subsection{Sample code}
\label{Sample code}

\begin{verbatim}

/*--------------------------------------------------------------------------
 * Indexing example code:
 *--------------------------------------------------------------------------*/

/*-----------------------------------------------
 * A) 0-based indexing
 *-----------------------------------------------*/

int
hypre_SampleA( hypre_Matrix *A,
               hypre_Vector *x,
               hypre_Vector *y )
{
   double  *A_data = hypre_MatrixData(A);
   int     *A_i    = hypre_MatrixI(A);
   int     *A_j    = hypre_MatrixJ(A);
   int     *A_n    = hypre_MatrixN(A);

   double  *x_data = hypre_VectorData(x);
   double  *y_data = hypre_VectorData(y);

   int      iiA, jjA;

   int     *coarse_grid;

   /* ... *

   /* matvec *
   for (iiA = 0; iiA < A_n; iiA++ )
   {
      for (jjA = A_i[iiA]; jjA < A_i[iiA+1]; jjA++)
      {
         y_data[iiA] += A_data[jjA] * x_data[A_j[jjA]];
      }
   }

   /* coarse-grid allocation *
   coarse_grid = hypre_TAlloc(int, A_n);

   /* coarse-grid selection function call *
   hypre_coarsen_( ..., coarse_grid);

   /* coarse-grid de-allocation *
   hypre_TFree(coarse_grid);

   return 0;
}

/*-----------------------------------------------
 * B) 1-based indexing (no tricks)
 *-----------------------------------------------*/

int
hypre_SampleB( hypre_Matrix *A,
               hypre_Vector *x,
               hypre_Vector *y )
{
   double  *A_data = hypre_MatrixData(A);
   int     *A_i    = hypre_MatrixI(A);
   int     *A_j    = hypre_MatrixJ(A);
   int     *A_n    = hypre_MatrixN(A);

   double  *x_data = hypre_VectorData(x);
   double  *y_data = hypre_VectorData(y);

   int      iiA, jjA;

   int     *coarse_grid;

   /* ... *

   /* matvec *
   for (iiA = 1; iiA <= A_n; iiA++ )
   {
      for (jjA = A_i[iiA-1]; jjA < A_i[iiA]; jjA++)
      {
         y_data[iiA-1] += A_data[jjA-1] * x_data[A_j[jjA-1]-1];
      }
   }

   /* coarse-grid allocation *
   coarse_grid = hypre_TAlloc(int, A_n);

   /* coarse-grid selection function call *
   hypre_coarsen_( ..., coarse_grid);

   /* coarse-grid de-allocation *
   hypre_TFree(coarse_grid);

   return 0;
}

/*-----------------------------------------------
 * C) 1-based indexing (with tricks):
 *-----------------------------------------------*/

int
hypre_SampleC( hypre_Matrix *A,
               hypre_Vector *x,
               hypre_Vector *y )
{
   double  *A_data = hypre_MatrixData(A) - 1;
   int     *A_i    = hypre_MatrixI(A) - 1;
   int     *A_j    = hypre_MatrixJ(A) - 1;
   int     *A_n    = hypre_MatrixN(A);

   double  *x_data = hypre_VectorData(x) - 1;
   double  *y_data = hypre_VectorData(y) - 1;

   int      iiA, jjA;

   int     *coarse_grid;

   /* ... *

   /* matvec *
   for (iiA = 1; iiA < A_n+1; iiA++ )
   {
      for (jjA = A_i[iiA]; jjA < A_i[iiA+1]; jjA++)
      {
         y_data[iiA] += A_data[jjA] * x_data[A_j[jjA]];
      }
   }

   /* coarse-grid allocation *
   coarse_grid = hypre_TAlloc(int, A_n) - 1;

   /* coarse-grid selection function call *
   hypre_coarsen_( ..., coarse_grid + 1);

   /* coarse-grid de-allocation *
   hypre_TFree(coarse_grid + 1);

   return 0;
}

/*-----------------------------------------------
 * D) 0 & 1-based indexing:
 *
 * We use the same tricks as in example C, except
 * that we "hide" most of them in the following
 * macros and functions:
 *
 * hypre_MatrixData, hypre_MatrixI, hypre_MatrixJ
 * hypre_VectorData
 * hypre_AllocBaseArray
 * hypre_FreeBaseArray
 * hypre_BaseArrayData
 *
 * The integer `base' is either a 0 or a 1 depending
 * on whether the storage is 0 or 1-based.
 *
 * We also use the following types so that the
 * code is more explicit about which variables
 * we are applying "tricks" to.
 *-----------------------------------------------*/

int
hypre_SampleD( hypre_Matrix *A,
               hypre_Vector *x,
               hypre_Vector *y )
{
   hypre_DoubleBaseArray  *A_data = hypre_MatrixData(A);
   hypre_IntBaseArray     *A_i    = hypre_MatrixI(A);
   hypre_IntBaseArray     *A_j    = hypre_MatrixJ(A);
   hypre_IntBaseArray     *A_n    = hypre_MatrixN(A);

   hypre_DoubleBaseArray  *x_data = hypre_VectorData(x);
   hypre_DoubleBaseArray  *y_data = hypre_VectorData(y);

   hypre_IntBaseArray     *coarse_grid;

   int                     iiA, jjA;

   /* ... */

   /* matvec */
   for (iiA = base; iiA < A_n+base; iiA++ )
   {
      for (jjA = A_i[iiA]; jjA < A_i[iiA+1]; jjA++)
      {
         y_data[iiA] += A_data[jjA] * x_data[A_j[jjA]];
      }
   }

   /* coarse-grid allocation ( = hypre_TAlloc(int, A_n) - base ) */
   coarse_grid = hypre_AllocBaseArray(int, A_n, base);

   /* coarse-grid selection function call */
   hypre_coarsen_( ..., &base,
                   hypre_BaseArrayData(coarse_grid, base);

   /* coarse-grid de-allocation ( = hypre_TFree(coarse_grid + base) ) */
   hypre_FreeBaseArray(coarse_grid, base);
   return 0;
}

\end{verbatim}

