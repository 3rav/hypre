\chapter{Finite Element Interface}
\label{chapter-FEI}

\section{Introduction}

User applications access the \hypre{} linear solvers via a pipeline of
two interfaces - user to finite element interface (called {\sf FEI}),
and the finite element to linear solver interface (called 
{\sf LinearSystemCore}). The purpose of {\sf FEI} is to allow users 
to submit the global matrices in the
form of element connectivities, element stiffness matrices, element
loads, and boundary conditions. These element information are processed
by an implementation of the {\sf FEI} which loads the global matrix and right
hand side vectors to the linear solver libraries via the 
{\sf LinearSystemCore} interface.
The {\sf LinearSystemCore} interface also facilitates using 
multiple linear system solver packages (such as PetSC or Aztec)
with little change in the user code.

The specification of the {\sf FEI} and its implementation was first
developed at Sandia. A simplified implementation has been implemented
at LLNL in \hypre{}'s finite element module. In Section 2, we describe the 
basic {\sf FEI} functions and a sample program to demonstrate how to
use them. A brief description of \hypre{}'s internal data structure and solver
capabilities is presented in Section 3.  Associated with \hypre{}'s finite
element interface is an FE-based gray-box multilevel preconditioning
module called {\sf MLI} which provides fast multilevel preconditioners.
A description of the {\sf MLI} is given in Section 4.
In Section 5, we describe the available options for using \hypre{}'s
rich solver capabilities. Users who prefer to create their own finite
element packages but would like to use the \hypre{} solvers can link their
packages to \hypre{} via the {\sf LinearSystemCore}. A description
of this interface is given in Section 6.  Finally, some installation 
and usage issues are discussed in Section 7.

\section{A Brief Description of The Finite Element Interface}

Embedded in application finite element codes are data structures
storing element connectivities, element stiffness matrices, element
loads, boundary conditions, nodal coordinates, etc. An implicit finite
element problem can be solved by assembling the global stiffness matrix
and the corresponding right hand side vector, and then calling linear
solver to calculate the solution. The first step in this process is 
thus to instantiate an {\sf FEI} object by
\begin{tabbing}
\hspace{0.5in} \= {\sf feiPtr = new FEI\_Implementation(mpiComm);}
\end{tabbing}
where {\sf mpiComm} is an MPI communicator (e.g. {\sf MPI\_COMM\_WORLD}).
Next, various finite element information need to be sent into the {\sf FEI}
object.

The first entity to be submitted to the {\sf FEI} is {\it field} information. 
A {\it field} has an identifier called {\it fieldID} and a rank or
{\it fieldSize} (number of degree of freedom). For example, for a simple
3D incompressible Navier Stokes equation, the nodal variable is the velocity
vector which has $3$ degrees of freedom; and the element variable (constant
over the element) is the pressure (scalar). If these are the only variables,
and if we assign {\it fieldID} $7$ and $8$ to them, respectively, then the
finite element field information can be set up by
\begin{tabbing}
\hspace{0.5in} \= {\sf nFields = 2;} \\
               \> {\sf fieldID = new int[nFields];} \\
               \> {\sf fieldID[0] = $7$; /* velocity vector */} \\
               \> {\sf fieldID[1] = $8$; /* pressure */} \\
               \> {\sf fieldSize = new int[nFields];} \\
               \> {\sf fieldSize[0] = $3$; /* velocity vector */} \\
               \> {\sf fieldSize[1] = $1$; /* pressure */ } \\
               \> {\sf feiPtr$->$initFields(nFields, fieldSize, fieldID);}
\end{tabbing}

Once the field information has been established, we are ready to initialize
an element block. An element block is characterized by the block identifier,
the number of elements, the number of nodes per element, the nodal fields 
and the element fields (fields that have been defined previously). Suppose 
we use $1000$ hexahedral elements in the element block $0$, the setup 
consists of
\begin{tabbing}
\hspace{0.5in} \= {\sf elemBlkID = 0;} \\
               \> {\sf nElems = 1000;} \\
               \> {\sf elemNNodes = 8; /* number of nodes per element */} \\
               \> {\sf nodeNFields = 1; /* nodal field - velocity */} \\
               \> {\sf nodeFieldIDs = new[nodeNFields];} \\
               \> {\sf nodeFieldIDs[0] = fieldID[0]; /* velocity */ } \\
               \> {\sf elemNFields = 1; /* element field - pressure */} \\
               \> {\sf elemFieldIDs = new[elemNFields];} \\
               \> {\sf elemFieldIDs[0] = fieldID[1]; /* pressure */ } \\
 \> {\sf feiPtr$->$initElemBlock(elemBlkID, nElems, elemNNodes, nodeNFields, nodeFieldIDs,}\\
 \> \hspace{1.0in} {\sf elemNFields, elemFieldIDs, 0);} 
\end{tabbing}
The last argument is to specify how the dependent variables are arranged in
the element matrices. A value of $0$ indicates that each variable is to be
arranged in a separate block (as opposed to interleaving).

In a parallel environment, each processor has one or more element blocks.
Unless the element blocks are all disjoint, some of the element blocks
share a common set of nodes on the subdomain boundaries. To facilitate
setting up interprocessor communications, shared nodes between subdomains
on different processors are to be identified and sent to the {\sf FEI}.
Hence, each node in the whole domain is assigned a unique global
identifier. The shared node list on each processor contains a subset
of the global node list
corresponding to the local nodes that are shared with the other processors.
The syntax for setting up the shared nodes is
\begin{tabbing}
\hspace{0.5in} \= {\sf feiPtr$->$initSharedNodes(nShared, sharedIDs, sharedLengs, sharedProcs);}
\end{tabbing}
This completes the initialization phase, and a completion signal is sent to
the {\sf FEI} via
\begin{tabbing}
\hspace{0.5in} \= {\sf feiPtr$->$initComplete();}
\end{tabbing}

Next we begin the {\it load} phase. The first entity for loading is the
nodal boundary conditions. Here we need to specify the number of boundary
equations and the boundary values given by {\sf alpha, beta}, and {\sf gamma}.  Depending whether the boundary conditions are Dirichlet, Neumann, or mixed,
the three values should be passed into the {\sf FEI} accordingly. 

The element stiffness matrices are to be loaded in the next step. We need
to specify the element number $i$, the element block to which element $i$
belongs, the element connectivity information, the element load, and the
element matrix format. The element connectivity specifies a set of $8$ node
global IDs (for hexahedral elements), and the element load is the load or
force for each degree of freedom.  The element format specifies how the
equations are arranged (similar to the interleaving scheme mentioned above).
The calling sequence for loading element stiffness matrices is
\begin{tabbing}
\hspace{0.5in} \= {\sf for (iE = 0; iE < nElems; iE++)} \\
 \> \hspace{0.5in} {\sf feiPtr$->$sumInElem(elemBlkID, elemID, elemConn[iE], elemStiff[iE],} \\
 \> \hspace{1.5in} {\sf elemLoads[iE], elemFormat);}
\end{tabbing}
Again, to complete the loading phase, a completion signal is sent to 
the {\sf FEI} via
\begin{tabbing}
\hspace{0.5in} \= {\sf feiPtr$->$loadComplete();}
\end{tabbing}
 
Now the global stiffness matrix and the corresponding right hand side
have been assembled. Before the linear system is solved, a number of 
solver parameters have to be passed into the {\sf FEI}. A detailed description
of the solver parameters is given in Section 3. An example is given below
\begin{tabbing}
\hspace{0.5in} \= {\sf nParams = 5;} \\
               \> {\sf paramStrings = new char*[nParams];} \\
               \> {\sf for (i = 0; i < nParams; i++) }\\
               \> \hspace{0.5in} {\sf paramStrings[i] = new char[100];} \\
               \> {\sf strcpy(paramStrings[0], "solver cg");} \\
               \> {\sf strcpy(paramStrings[1], "preconditioner diag");} \\
               \> {\sf strcpy(paramStrings[2], "maxiterations 100");} \\
               \> {\sf strcpy(paramStrings[3], "tolerance 1.0e-6");} \\
               \> {\sf strcpy(paramStrings[4], "outputLevel 1");} \\
               \> {\sf feiPtr$->$parameters(nParams, paramStrings);} 
\end{tabbing}

To solve the linear system of equations, we call
\begin{tabbing}
\hspace{0.5in} \= {\sf feiPtr$->$solve(\&status);}
\end{tabbing}
where {\sf status} is returned from the {\sf FEI} indicating whether
the solve is successful. Finally, the solution can be retrieved by
\begin{tabbing}
\hspace{0.5in} \= {\sf feiPtr->getBlockNodeSolution(elemBlkID, nNodes, nodeIDList, solnOffsets, solnValues);}
\end{tabbing}
where {\sf solnOffsets[i]} is the index pointing to the first location 
where the variables at node $i$ is returned in {\sf solnValues}.

\section{\hypre{} Solvers} 

On the other side of the {\sf LinearSystemCore} interface is the linear
solver package. Any linear solver package can be used as long as there
is an implementation of the interface in the solver package itself. The
{\sf LinearSystemCore} implementation in \hypre{} is encapsulated in the 
\hypre{} finite element conceptual interface called {\sf HYPRE\_LinSysCore}.
The incoming element stiffness matrices and other information are 
converted into \hypre{}'s internal data structures. In this section we
briefly describe some of these internal data structures and capabilities.

\subsection{Parallel Matrix and Vector Construction} 

The matrix class in \hypre{} accessible via the {\sf LinearSystemCore} interface
is the parallel compressed sparse row ({\sf ParCSR}) matrix.  The 
requirements about how the global matrix is partitioned among the
processors are that each processor holds a contiguous block of rows and columns
and the equation numbers in processors of lower rank are lower than those
in processors of higher rank.  The {\sf FEI} is responsible for ensuring
that these two requirements are followed. The matrix can be loaded in
parallel - a row or a block of rows at a time.  The solution and right
hand side vectors are constructed accordingly. The matrix rows corresponding
to the shared nodes can be assigned to either processor, and is determined
by the {\sf FEI} itself. Once the incoming matrix and vector data have
been captured in the \hypre{} {\sf ParCSR} format, a whole of matrix and
vector operators are available for use in the \hypre{} solvers.

\subsection{Sequential and Parallel Solvers} 

\hypre{} currently has many iterative solvers. There is also internally a
version of the sequential {\sf SuperLU} direct solver (developed at U.C.
Berkeley) suitable to small problems (may be up to the size of $10000$).
In the following we list some of these internal solvers.

\begin{enumerate}
\item Krylov solvers (CG, GMRES, FGMRES, TFQMR, BiCGSTAB, symmetric QMR),
\item Boomeramg (a parallel algebraic multigrid solver),
\item SuperLU direct solver (sequential),
\item SuperLU direct solver with iterative refinement (sequential), 
\end{enumerate}

\subsection{Parallel Preconditioners} 

The performance of the Krylov solvers can be improved by clever selection
of preconditioners. \hypre{} currently has the following preconditioners 
available via the {\sf LinearSystemCore} interface: 

\begin{enumerate}
\item no preconditioning,
\item diagonal, 
\item another parallel incomplete LU ({\sf Euclid}),
\item parallel sparse approximate inverse ({\sf ParaSails}),
\item parallel algebraic multigrid ({\sf BoomerAMG}),
\item another parallel algebraic multigrid ({\sf MLI} based on
the smoothed aggregation multigrid methods),
\item parallel domain decomposition with inexact local solves ({\sf DDIlut}), 
\item least-squares polynomial preconditioner,
\item $2 \times 2$ block preconditioner, and
\item $2 \times 2$ Uzawa preconditioner.
\end{enumerate}

Some of these preconditioners can be tuned by a number of internal parameters
modifiable by users. A description of these parameters is given in later
sections.

\subsection{Matrix Reduction} 

For some structural mechanics problems with multi-point constraints the 
discretization matrix is indefinite (eigenvalues lie in both sides of
the imaginary axis). Indefinite matrices are much more difficult to solve
than definite matrices. Methods have been developed to reduce these
indefinite matrices to definite matrices.  Two matrix reduction algorithms
have been implemented in \hypre{}, as presented in the following subsections.

\subsubsection{Schur Complement Reduction}

The incoming linear system of equations is assumed to be in the form :
\[ \left[ 
\begin{array}{cc} 
   D   & B \\
   B^T & 0
\end{array}
  \right] 
  \left[
\begin{array}{c} 
   x_1 \\
   x_2
\end{array}
  \right] 
  =
  \left[
\begin{array}{c} 
   b_1 \\
   b_2
\end{array}
  \right] 
\]
where $D$ is a diagonal matrix.  After Schur complement reduction is applied, 
the resulting linear system becomes
$$
- B^T D^{-1} B x_2 = b_2 - B^T D^{-1} b_1.
$$

\subsubsection{Slide Surface Reduction}

With the presence of slide surfaces, the matrix is in the same form as in the
case of Schur complement reduction.  Here $A$ represents the relationship 
between the master, slave, and other degrees of freedom.  The matrix block 
$[B^T 0]$ corresponds to the constraint equations.  The goal of reduction
is to eliminate the constraints.  As proposed by Manteuffel, the trick is
to re-order the system into a $3 \times 3$ block matrix.
\[ 
\left[ 
\begin{array}{ccc} 
   A_{11}  & A_{12} & N \\
   A_{21}  & A_{22} & D \\
   N_{T}   & D      & 0 \\
\end{array}
\right] 
=
\left[ 
\begin{array}{ccc} 
   A_{11}       & \hat{A}_{12} \\
   \hat{A}_{21} & \hat{A}_{22}.
\end{array}
\right] 
\]
The reduced system has the form :
$$
(A_{11} - \hat{A}_{21} \hat{A}_{22}^{-1} \hat{A}_{12}) x_1 =
b_1 - \hat{A}_{21} \hat{A}_{22}^{-1} b_2,
$$
which is symmetric positive definite (SPD) if the original matrix is PD.
In addition, $\hat{A}_{22}^{-1}$ is easy to compute. 

There are three slide surface reduction algorithms in \hypre{}.  
The first follows the matrix formulation in this section.  The second is
similar except that it replaces the eliminated slave equations with
identity rows so that the degree of freedom at each node is preserved.
This is essential for certain block algorithms such as the smoothed
aggregation multilevel preconditioners.
The third is similar to the second except that it is more general and 
can be applied to problems with intersecting slide surfaces (sequential only
for intersecting slide surfaces).

\subsection{Other Features} 

To improve the efficiency of the \hypre{} solvers, a few other features have been
incorporated.  We list a few of these features below :

\begin{enumerate}
\item Preconditioner reuse - For multiple linear solves with matrices that are
      slightly perturbed from each other, oftentimes the use of the same 
      preconditioners can save preconditioner setup times but suffer little
      convergence rate degradation.
\item Projection methods - For multiple solves that use the same matrix,
      previous solution vectors can sometimes be used to give a better initial
      guess for subsequent solves.  Two projection schemes have been implemented
      in \hypre{} - A-conjugate projection (for SPD matrices) and minimal residual
      projection (for both SPD and non-SPD matrices).
\item The sparsity pattern of the matrix is in general not destroyed after
      it has been loaded to an \hypre{} matrix.  But if the matrix is not to
      be reused, an option is provided to clean up this pattern matrix to
      conserve memory usage.
\end{enumerate}

\section{The MLI Package}

MLI is an object-oriented module that implements the class of algebraic
multigrid algorithms based on Vanek and Brezina's smoothed aggregation
method.  There are two main algorithms in this module - the original
smoothed aggregation algorithm and the modified version that uses
the finite element substructure matrices to construct the prolongation
operators.  As such, the later algorithm can only be used in the
finite element context via the finite element interface.  In addition,
the nodal coordinates obtained via the finite element interface can be
used to construct a better prolongation operator than the pure 
translation modes.

\section{\hypre{} LinearSystemCore Parameters}

User applications interact directly with \hypre{} via the {\sf parameters}
function. In this section we list the parameters recognized by
{\sf HYPRE\_LinSysCore}.

\subsection{Parameters for Solvers and Preconditioners}

\begin{description}
\item[solver xxx] where xxx specifies one of {\sf cg}, {\sf gmres},
           {\sf fgmres}, {\sf bicgs}, {\sf bicgstab}, {\sf tfqmr}, 
           {\sf symqmr}, {\sf superlu}, or {\sf superlux}.  The 
           default is {\sf gmres}.
           The solver type can be followed by {\sf override} to
           specify its priority when multiple solvers are declared
           at random order.
\item[preconditioner xxx] where xxx is one of {\sf diagonal}, {\sf pilut},
           {\sf euclid}, {\sf parasails}, {\sf boomeramg}, {\sf ddilut}, 
           {\sf poly}, {\sf blockP}, {\sf Uzawa}, or {\sf mli}. The 
           default is {\sf diagonal}.  Another option for 
           xxx is {\sf reuse} which allows the preconditioner to be reused 
           (this is to be set after a preconditioner has been set up already).
           The preconditioner type can be followed by {\sf override} to
           specify its priority when multiple preconditioners are declared
           at random order.
\item[maxIterations xxx] where xxx is an integer specifying the maximum 
           number of iterations permitted for the iterative solvers.
           The default value is 1000.
\item[tolerance xxx] where xxx is a floating point number specifying the 
           termination criterion for the iterative solvers.  The default 
           value is 1.0E-6.
\item[gmresDim xxx] where xxx is an integer specifying the value of m in
           restarted GMRES(m).  The default value is 100.
\item[stopCrit xxx] where xxx is one of {\sf absolute} or {\sf relative}
           stopping criterion.
\item[superluOrdering xxx] - where xxx specifies one of {\sf natural} or
           {\sf mmd} (minimum degree ordering).  This ordering
           is used to minimize the number of nonzeros generated
           in the LU decomposition.  The default is natural ordering.
\item[superluScale xxx] where xxx specifies one of {\sf y} (perform row
           and column scalings before decomposition) or {\sf n}.
           The default is no scaling.
\end{description}

\subsection{Parameters for ILUT, SPAI, and Polynomial Preconditioners}

\begin{description}
\item[ddilutFillin xxx] where xxx is a floating point number specifying 
           the maximum number of nonzeros kept in the formation of local 
           incomplete L and U (a value of $0.0$ means same sparsity as $A$,
           and a value of $1.0$ means two times the number of nonzeros as
           $A$.).  If this is not called, a value will be selected
           based on the sparsity of the matrix.
\item[ddilutDropTol xxx] where xxx is a floating point number specifying the 
           threshold to drop small entries in L and U.  The default
           value is 0.0.
\item[euclidNlevels xxx] where xxx is an non-negative integer specifying 
           the desired sparsity of the incomplete factors.  The
           default value is 0.
\item[euclidThreshold xxx] where xxx is a floating point number specifying 
           the threshold used to sparsify the incomplete factors.  The default
           value is 0.0.
\item[parasailsThreshold xxx] where xxx is a floating point number between 0 
           and 1 specifying the threshold used to prune small entries
           in setting up the sparse approximate inverse.  The default
           value is 0.0.
\item[parasailsNlevels xxx] where xxx is an integer larger than 0 specifying 
           the desired sparsity of the approximate inverse.  The
           default value is 1.
\item[parasailsFilter xxx] where xxx is a floating point number between 0 
           and 1 specifying the threshold used to prune small entries
           in $A$.  The default value is 0.0.
\item[parasailsLoadbal xxx] where xxx is a floating point number between 0 
           and 1 specifying how load balancing has to be done 
           (Edmond, explain please).  The default value is 0.0.
\item[parasailsSymmetric] sets Parasails to take $A$ as symmetric.
\item[parasailsUnSymmetric] sets Parasails to take $A$ as nonsymmetric
                            (default).
\item[parasailsReuse] sets Parasails to reuse the sparsity pattern of $A$.
\item[polyorder xxx] where xxx is the order of the least-squares polynomial 
           preconditioner.
\end{description}

\subsection{Parameters for Multilevel Preconditioners}

\subsubsection{Parameters for BoomerAMG Preconditioner}
\begin{description}
\item[amgCoarsenType xxx] where xxx specifies one of {\sf falgout} or
           {\sf ruge}, or {\sf default (CLJP)} coarsening for BOOMERAMG.
\item[amgMeasureType xxx] where xxx specifies one of {\sf local} or
           or {\sf global}.  This parameter affects how coarsening is performed
           in parallel.
\item[amgNumSweeps xxx] where xxx is an integer specifying the number of
           pre- and post-smoothing at each level of BOOMERAMG.
           The default is two pre- and two post-smoothings.
\item[amgRelaxType xxx] where xxx is one of {\sf jacobi} (Damped Jacobi),
           {\sf gs-slow} (sequential Gauss-Seidel), {\sf gs-fast}
           (Gauss-Seidel on interior nodes), or {\sf hybrid}.
           The default is {\sf hybrid}.
\item[amgRelaxWeight xxx] where xxx is a floating point number between 
           0 and 1 specifying the damping factor for BOOMERAMG's damped
           Jacobi and GS smoothers.  The default value is 1.0.
\item[amgRelaxOmega xxx] where xxx is a floating point number between 
           0 and 1 specifying the damping factor for BOOMERAMG's hybrid
           smoother for multiple processors.  The default value is 1.0.
\item[amgStrongThreshold xxx] where xxx is a floating point number between 0 
           and 1 specifying the threshold used to determine
           strong coupling in BOOMERAMG's coasening.  The default 
           value is 0.25.
\item[amgSystemSize xxx] where xxx is the degree of freedom per node.
\item[amgUseGSMG] - tells BOOMERAMG to use a different coarsening called GSMG.
\item[amgGSMGNumSamples] where xxx is the number of samples to generate
           to determine how to coarsen for GSMG.
\end{description}

\subsubsection{Parameters for MLI Preconditioner}

\begin{description}
\item[outputLevel xxx] where xxx is the output level for diagnostics.
\item[method xxx] where xxx is either {\sf AMGSA} (default), {\sf AMGSAe},
     to indicate which MLI algorithm is to be used.
\item[numLevels xxx] where xxx is the maximum number of levels (default=30)
                     used.
\item[maxIterations xxx] where xxx is the maximum number of iterations
                         (default = 1 as preconditioner).
\item[cycleType xxx] where xxx is either 'V' or 'W' cycle (default = 'V').
\item[strengthThreshold xxx] strength threshold for coarsening (default = 0).
\item[smoother xxx] where xxx is either {\sf Jacobi}, {\sf BJacobi}, {\sf GS}, 
     {\sf SGS}, {\sf HSGS} (SSOR,default), {\sf BSGS}, {\sf ParaSails}, 
     {\sf MLS}, {\sf CGJacobi}, {\sf CGBJacobi}, or {\sf Chebyshev}.
\item[numSweeps xxx] where xxx is the number of smoother sweeps (default = 2).
\item[coarseSolver xxx] where xxx is one of those in 'smoother' or
     {\sf SuperLU} (default).
\item[minCoarseSize xxx] where xxx is the minimum coarse grid size to
     control the number of levels used (default = 3000).
\item[Pweight xxx] where xxx is the relaxation parameter for the prolongation
     smoother (default 0.0).
\item[nodeDOF xxx] where xxx is the degree of freedom for each node
                   (default = 1).
\item[nullSpaceDim xxx] where xxx is the dimension of the null space for
     the coarse grid (default = 1).
\item[useNodalCoord xxx] where xxx is either 'on' or 'off' (default)
     to indicate whether the nodal coordinates are used to generate the 
     initial null space.
\item[saAMGCalibrationSize xxx] where xxx is the additional null space 
     vectors to be generated via calibration (default = 0).
\item[numSmoothVecs xxx] where xxx is the number of near null space 
     vectors used to create the prolongation operator (default = 0).
\item[smoothVecSteps xxx] where xxx is the number of smoothing steps
     used to generate the smooth vectors (default = 0).
\end{description}

In addition, to use 'AMGSAe', the parameter 'haveSFEI' has to be sent into
the FEI using the parameters function (this option is valid only for the
Sandia FEI implementation).

\subsection{Parameters for Block and Uzawa Preconditioners}
\subsubsection{Parameters for Block Preconditioners}
The parameters for this preconditioner are preceded by the keyword {sf blockP}.
The available parameters after this keywords are:
\begin{description}
\item[blockD] turns on block diagonal preconditioning.
\item[blockT] turns on block tridiagonal preconditioning.
\item[blockLU] turns on block LU preconditioning.
\item[outputLevel xxx] where xxx is the output level for diagnostics.
\item[block1FieldID xxx] where xxx is field ID for the (1,1) block.
\item[block2FieldID xxx] where xxx is field ID for the (2,2) block (for
    ALE3D's implicit hydrodynamics with slide surfaces, the field ID for
    both blocks are -3.)
\item[printInfo] prints information about internal parameter settings.
\item[lumpedMassScheme xxx] where is either {\sf diag} (take the diagonal
     of the (1,1) block) or {\sf ainv} (take the approximate inverse of
     the (1,1) block.
\item[invA11PSNlevels xxx] where xxx is 0 or 1 to indicate the ParaSails
     nlevels to use to generate the lumped mass matrix (if {\sf ainv} is
     selected.)
\item[invA11PSThresh xxx] where xxx is a floating point number between 0 
     and 1 to indicate the ParaSails threshold to use to generate the lumped 
     mass matrix (if {\sf ainv} is selected.)
\item[A11Solver xxx] where xxx is either {\sf cg} or {\sf gmres} as solver
     for the (1,1) block.
\item[A11Tolerance xxx] where xxx is convergence tolerance for the
     (1,1) block.
\item[A11MaxIterations xxx] where xxx is maximum number of iterations for
     the (1,1) block.
\item[A11Precon xxx] where xxx is either {\sf pilut}, {\sf boomeramg}, 
     {\sf euclid}, {\sf parasails}, {\sf ddilut}, or {\sf mli}.
\item[A11PreconPSNlevels xxx] - ParaSails' nlevels.
\item[A11PreconPSThresh xxx] - ParaSails' threshold. 
\item[A11PreconPSFilter xxx] - ParaSails' filter.
\item[A11PreconAMGThresh xxx] - Boomeramg's threshold.
\item[A11PreconAMGRelaxType xxx] - Boomeramg's smoother.
\item[A11PreconAMGNumSweeps xxx] - Boomeramg's numSweeps.
\item[A11PreconAMGSystemSize xxx] - Boomeramg's systemSize.
\item[A11PreconEuclidNLevels xxx] - Euclid's nlevels.
\item[A11PreconEuclidThresh xxx] - Euclid's threshold.
\item[A11PreconPilutFillin xxx] - Pilut's fillin.
\item[A11PreconPilutDropTol xxx] - Pilut's drop tolerance.
\item[A11PreconDDIlutFillin xxx] - DDILUT's fillin.
\item[A11PreconDDIlutDropTol xxx] - DDILUT's drop tolerance.
\item[A11PreconMLIRelaxType xxx] - MLI's smoother.
\item[A11PreconMLIThresh xxx] - MLI's threshold.
\item[A11PreconMLIPweight xxx] - MLI's Pweight.
\item[A11PreconMLINumSweeps xxx] - MLI's numSweeps.
\item[A11PreconMLINodeDOF xxx] - MLI's nodeDOF.
\item[A11PreconMLINullDim xxx] - MLI's null space dimension.
\item[A22Solver xxx] where xxx is either {\sf cg} or {\sf gmres} as solver
     for the (2,2) block.
\item[A22Tolerance xxx] where xxx is convergence tolerance for the
     (2,2) block.
\item[A22MaxIterations xxx] where xxx is maximum number of iterations for
     the (2,2) block.
\item[A22Precon xxx] where xxx is either {\sf pilut}, {\sf boomeramg}, 
     {\sf euclid}, {\sf parasails}, {\sf ddilut}, or {\sf mli}.
\item[A22PreconPSNlevels xxx] - ParaSails nlevels.
\item[A22PreconPSThresh xxx] - ParaSails' threshold.
\item[A22PreconPSFilter xxx] - ParaSails' filter.
\item[A22PreconAMGThresh xxx] - Boomeramg's threshold.
\item[A22PreconAMGRelaxType xxx] - Boomeramg's smoother.
\item[A22PreconAMGNumSweeps xxx] - Boomeramg's numSweeps.
\item[A22PreconAMGSystemSize xxx] - Boomeramg's systemSize.
\item[A22PreconEuclidNLevels xxx] - Euclid's nlevels.
\item[A22PreconEuclidThresh xxx] - Euclid's threshold.
\item[A22PreconPilutFillin xxx] - Pilut's fillin.
\item[A22PreconPilutDropTol xxx] - Pilut's drop tolerance.
\item[A22PreconDDIlutFillin xxx] - DDILUT's fillin.
\item[A22PreconDDIlutDropTol xxx] - DDILUT's drop tolerance.
\item[A22PreconMLIRelaxType xxx] - MLI's smoother.
\item[A22PreconMLIThresh xxx] - MLI's threshold.
\item[A22PreconMLIPweight xxx] - MLI's Pweight.
\item[A22PreconMLINumSweeps xxx] - MLI's numSweeps.
\item[A22PreconMLINodeDOF xxx] - MLI's nodeDOF.
\item[A22PreconMLINullDim xxx] - MLI's null space dimension.
\end{description}

\subsubsection{Parameters for Uzawa Preconditioner}
The Uzawa preconditioner has a similar parameter set as block preconditioner,
as described in the following (except that DDIlut is not available here).
\begin{description}
\item[outputLevel xxx] -  where xxx is the output level for diagnostics.
\item[A11Solver xxx] where xxx is either {\sf cg} or {\sf gmres} as solver
     for the (1,1) block.
\item[A11Tolerance xxx] where xxx is convergence tolerance for the
     (1,1) block.
\item[A11MaxIterations xxx] where xxx is maximum number of iterations for
     the (1,1) block.
\item[A11Precon xxx] where xxx is either {\sf pilut}, {\sf boomeramg}, 
     {\sf euclid}, {\sf parasails}, {\sf ddilut}, or {\sf mli}.
\item[A11PreconPSNlevels xxx] - ParaSails' nlevels.
\item[A11PreconPSThresh xxx] - ParaSails' threshold.
\item[A11PreconPSFilter xxx] - ParaSails' filter.
\item[A11PreconAMGThresh xxx] - Boomeramg's threshold.
\item[A11PreconAMGRelaxType xxx] - Boomeramg's smoother.
\item[A11PreconAMGNumSweeps xxx] - Boomeramg's numSweeps.
\item[A11PreconAMGSystemSize xxx] - Boomeramg's systemSize.
\item[A11PreconEuclidNLevels xxx] - Euclid's nlevels.
\item[A11PreconEuclidThresh xxx] - Euclid's threshold.
\item[A11PreconPilutFillin xxx] - Pilut's fillin.
\item[A11PreconPilutDropTol xxx] - Pilut's drop tolerance.
\item[A11PreconMLIRelaxType xxx] - MLI's smoother.
\item[A11PreconMLIThresh xxx] - MLI's threshold.
\item[A11PreconMLIPweight xxx] - MLI's Pweight.
\item[A11PreconMLINumSweeps xxx] - MLI's numSweeps.
\item[A11PreconMLINodeDOF xxx] - MLI's nodeDOF.
\item[A11PreconMLINullDim xxx] - MLI's null space dimension.
\item[S22SolverDampingFactor xxx] where xxx is the damping (scaling) factor
     for the Schur complement approximation of the (2,2) block.
\item[S22Solver xxx] where xxx is either {\sf cg} or {\sf gmres} as solver
     for the (2,2) block.
\item[S22Tolerance xxx] where xxx is convergence tolerance for the
     (2,2) block.
\item[S22MaxIterations xxx] where xxx is maximum number of iterations for
     the (2,2) block.
\item[S22Precon xxx] where xxx is either {\sf pilut}, {\sf boomeramg}, 
     {\sf euclid}, {\sf parasails}, {\sf ddilut}, or {\sf mli}.
\item[S22PreconPSNlevels xxx] - ParaSails' nlevels.
\item[S22PreconPSThresh xxx] - ParaSails' threshold.
\item[S22PreconPSFilter xxx] - ParaSails' filter.
\item[S22PreconAMGThresh xxx] - Boomeramg's threshold.
\item[S22PreconAMGRelaxType xxx] - Boomeramg's smoother.
\item[S22PreconAMGNumSweeps xxx] - Boomeramg's numSweeps.
\item[S22PreconAMGSystemSize xxx] - Boomeramg's systemSize.
\item[S22PreconEuclidNLevels xxx] - Euclid's nlevels.
\item[S22PreconEuclidThresh xxx] - Euclid's threshold.
\item[S22PreconPilutFillin xxx] - Pilut's fillin.
\item[S22PreconPilutDropTol xxx] - Pilut's drop tolerance.
\item[S22PreconMLIRelaxType xxx] - MLI's smoother.
\item[S22PreconMLIThresh xxx] - MLI's threshold.
\item[S22PreconMLIPweight xxx] - MLI's Pweight.
\item[S22PreconMLINumSweeps xxx] - MLI's numSweeps.
\item[S22PreconMLINodeDOF xxx] - MLI's nodeDOF.
\item[S22PreconMLINullDim xxx] - MLI's null space dimension.
\end{description}

\subsection{Parameters for Matrix Reduction}
\begin{description}
\item[schurReduction] turns on the Schur reduction mode.
\item[slideReduction] turns on the slide reduction mode.
\item[slideReduction2] turns on the slide reduction mode version 2 
(see section 2).
\item[slideReduction3] turns on the slide reduction mode version 3 
(see section 2).
\end{description}

\subsection{Parameters for Diagnostics and Performance Tuning}
\begin{description}
\item[outputLevel xxx] where xxx is an integer specifying the output
           level.  An output level of $1$ prints only the solver 
           information such as number of iterations and timings.
           An output level of $2$ prints debug information such as
           the functions visited and preconditioner information.
           An output level of $3$ or higher prints more debug information 
           such as the matrix and right hand side loaded via the 
           LinearSystemCore functions to the standard output.
\item[setDebug xxx] where xxx is one of {\sf slideReduction1}, 
           {\sf slideReduction2},
           {\sf slideReduction3} (level 1,2,3 diagnostics in the slide surface
           reduction code), {\sf printMat} (print the original matrix
           into a file), {\sf printReducedMat} (print the reduced matrix
           into a file),  {\sf printSol} (print the solution into a file), 
           {\sf ddilut} (output diagnostic information for DDIlut
           preconditioner setup), and {\sf amgDebug} (output diagnostic 
           information for AMG).
\item[optimizeMemory] cleans up the matrix sparsity pattern after the matrix
           has been loaded. (It has been kept to allow matrix reuse.)
\item[imposeNoBC] turns off the boundary condition to allow diagnosing the
           matrix (for example, checking the null space.)
\end{description}

\subsection{Miscellaneous Parameters}
\begin{description}
\item[AConjugateProjection xxx] where xxx specifies the number of previous
           solution vectors to keep for the A-conjugate projection. 
           The default is 0 (the projection is off).
\item[minResProjection xxx] where xxx specifies the number of previous
           solution vectors to keep for projection. 
           The default is 0 (the projection is off).
\item[haveFEData] indicates that additional finite element information are 
           available to assist in building more efficient solvers. 
\item[haveSFEI] indicates that the simplified finite element information are 
           available to assist in building more efficient solvers. 
\end{description}

\section{The LinearSystemCore Interface}

As described before, users who prefer to create their own finite element
interface package can also take advantage of the rich solver capabilities
in \hypre{}. In this section we show how to access {\sf HYPRE\_LinSysCore}'s
internal solver directly.  Users who choose this path need first to 
construct an array (say, {\sf eqnOffsets} describing the matrix row
partitioning across all processors (so {\sf eqnOffsets[p]} and
{\sf eqnOffset[p+1]} have the starting and ending row indices for processor
p). Furthermore, suppose the local submatrix has been constructed as a
compressed sparse row (CSR) matrix in the {\sf ia, ja, val} arrays. 
The following program segment describes the function calls to set up
the internal matrix and solve the linear system.

\begin{tabbing}
\hspace{0.5in} \= {\sf Program Segment} \\[1mm]
\> {\sf startRow = eqnOffsets[mypid];} \\
\> {\sf endRow = eqnOffsets[mypid+1] - 1;} \\
\> {\sf nrows = endRow - startRow + 1} \\
\> {\sf for ( i = startRow; i <= endRow; i++ ) \{ } \\
\> \hspace{0.3in} \= {\sf ncnt = ia[i+1] - ia[i];} \\
\> \> {\sf rowLengths[i-startRow] = ncnt;} \\
\> \> {\sf colIndices[i-startRow] = new int[ncnt];} \\
\> \> {\sf k = 0;} \\
\> \> {\sf for (j = ia[i]; j < ia[i+1]; j++) colIndices[i-startRow][k++] = ja[j];}\\
\> \} \\
\> {\sf HYPRE\_LinSysCore\_create(\&lsc, MPI\_COMM\_WORLD);} \\
\> {\sf HYPRE\_setGlobalOffsets(lsc, nrows, NULL, eqnOffsets, NULL);} \\
\> {\sf HYPRE\_setMatrixStructure(lsc, colIndices, rowLengths, NULL, NULL, NULL);} \\
\> {\sf for ( i = startRow; i <= endRow; i++ ) \{ } \\
\> \> {\sf ncnt = ia[i+1] - ia[i];} \\
\> \> {\sf HYPRE\_sumIntoSystemMatrix(lsc, i, ncnt, \&val[ia[i]], \&ja[ia[i]]);}\\
\> \> {\sf HYPRE\_sumIntoRHSVector(1, \&rhs[i], \&i);} \\
\> \} \\
\> {\sf HYPRE\_matrixLoadComplete();}\\
\> {\sf strcpy(paramString, "solver gmres");} \\
\> {\sf HYPRE\_parameters(1, \&paramString);} \\
\> {\sf strcpy(paramString, "preconditioner boomeramg");} \\
\> {\sf HYPRE\_parameters(1, \&paramString);} \\
\> {\sf HYPRE\_launchSolver(\&status, \&iterations);}
\end{tabbing}

A list of available functions is given in the following.

\begin{tabbing}
{\sf HYPRE\_LinSysCore\_create(LinSysCore **lsc, MPI\_Comm comm)} \\[1mm]
{\sf HYPRE\_LinSysCore\_destroy(LinSysCore **lsc)} \\[1mm]
{\sf HYPRE\_parameters(LinSysCore *lsc, int nParams, char **params)} \\[1mm]
{\sf HYPRE\_setGlobalOffsets(LinSysCore* lsc, int leng, int* nodeOffsets,} \\
\hspace{1.0in} {\sf int* eqnOffsets, int* blkEqnOffsets)} \\[1mm]
{\sf HYPRE\_setMatrixStructure(LinSysCore *lsc, int** ptColIndices,} \\
\hspace{1.0in} {\sf int* ptRowLengths, int** blkColIndices, int* blkRowLengths, int* ptRowsPerBlkRow)} \\[1mm]
{\sf HYPRE\_resetMatrixAndVector(LinSysCore *lsc, double val)} \\[1mm]
{\sf HYPRE\_resetMatrix(LinSysCore *lsc, double val)} \\[1mm]
{\sf HYPRE\_resetRHSVector(LinSysCore *lsc, double val)} \\[1mm]
{\sf HYPRE\_sumIntoSystemMatrix(LinSysCore *lsc, int numPtRows, const int* ptRows,}\\
\hspace{1.0in} {\sf int numPtCols, const int* ptCols, int numBlkRows, const int* blkRows,} \\
\hspace{1.0in} {\sf int numBlkCols, const int* blkCols, const double* const* values)} \\[1mm]
{\sf HYPRE\_sumIntoRHSVector(LinSysCore *lsc, int num, const double* values, const int* indices)} \\[1mm]
{\sf HYPRE\_matrixLoadComplete(LinSysCore *lsc)} \\[1mm]
{\sf HYPRE\_enforceEssentialBC(LinSysCore *lsc, int* globalEqn, double* alpha,
                             double* gamma, int leng)} \\[1mm]

{\sf HYPRE\_enforceRemoteEssBCs(LinSysCore *lsc,int numEqns,int* globalEqns, int** colIndices,} \\
\hspace{1.0in} {\sf int* colIndLen, double** coefs)} \\[1mm]

{\sf HYPRE\_enforceOtherBC(LinSysCore *lsc, int* globalEqn, double* alpha, double *beta} \\
\hspace{1.0in} {\sf double* gamma, int leng)} \\[1mm]

{\sf HYPRE\_putInitialGuess(LinSysCore *lsc, const int* eqnNumbers,
                          const double* values, int leng)} \\[1mm]
{\sf HYPRE\_getSolution(LinSysCore *lsc, double *answers, int leng)} \\[1mm]

{\sf HYPRE\_getSolnEntry(LinSysCore *lsc, int eqnNumber, double *answer)} \\[1mm]

{\sf HYPRE\_formResidual(LinSysCore *lsc, double *values, int leng)} \\[1mm]

{\sf HYPRE\_launchSolver(LinSysCore *lsc, int *solveStatus, int *iter)} \\[1mm]
\end{tabbing}

\section{HYPRE LinearSystemCore Installation}

The ultimate objective is for application users to have immediate access
to the latest FEI/\hypre{} library files on different computing platforms
via public {\sf lib} directories.  While this feature is forthcoming, careful 
version control is needed for users to keep track of capabilities and bug fixes 
for different installations.  Users who would like to set up the FEI/\hypre{}
on their own should do the following :

\begin{enumerate}

\item obtain the \hypre{} and the Sandia FEI source codes (alternatively, use
      the {\sf FEI} implementation in \hypre{}),
\item compile Sandia's {\sf FEI} (fei-2.5.0) to create the
      {\sf libfei\_base.a} file.
\item compile \hypre{} 
\begin{enumerate}
\item download \hypre{} from the web, ungzip and untar it
\item go into the {\sf linear\_solvers} directory
\item do a 'configure' with the {\sf --with-FEI\_BASE\_DIR} option set to
      the {\sf FEI} include directory plus other compile options
\item compile with {\sf make install} which will create the
      {\sf libHYPRE\_LSI*} file in the {\sf linear\_solvers/hypre/lib}
      directory.
\end{enumerate}
\item call the {\sf FEI} functions in your application code (example given
      previously)
\begin{enumerate}
\item include {\sf cfei\-hypre.h} in your file 
\item include {\sf FEI\_Implementation.h} in your file 
\item make sure your application has an {\sf include} and an {\sf lib} path 
      to the {\sf include} and {\sf lib} directories created above. 
\end{enumerate}

\end{enumerate}

\subsection{Linking with the library files}

To link the {\sf FEI} and \hypre{} into the executable, the following has to be
attached to the linking command :

\begin{tabbing}
\hspace{0.5in} \= {\sf -L\$\{LIBPATHS\} -lfei\_base -lHYPRE\_LSI} 
\end{tabbing}
along with all the other libraries (Note : the order in which the libraries are
listed may be important), where {\sf LIBPATHS} are where 
the \hypre{} and {\sf FEI} libraray files can be found.  

Since some of these library files make calls to LAPACK and BLAS functions, 
the corresponding libraries need to be linked along with (placed after) these 
library files.  For example, on the DEC cluster, it suffices to link
with the {\sf dxml} library, (So {\sf -ldxml} is placed after the above link
sequence, with {\sf -lm} placed after {\sf -ldxml}.) while the {\it essl}
library can be used on the blue machine. If {\sf SuperLU} is also needed,
{\sf -lHYPRE\_superlu} should be placed immediately after {\sf HYPRE\_LSI}.

\subsection{Some more caveats for application developers}

Building an application executable often requires linking with many different
software packages, and many software packages use some LAPACK and/or BLAS
functions.  In order to alleviate the problem of multiply defined functions
at link time, it is recommended that all software libraries are stripped of
all LAPACK and BLAS function definitions.  These LAPACK and BLAS functions 
should then be resolved at link time by linking with the system LAPACK and
BLAS libraries (e.g. dxml on DEC cluster).  Both \hypre{} and SuperLU were
built with this in mind.  However, some other software library files needed
may have the BLAS functions defined in them.  To avoid the problem of
multiply defined functions, it is recommended that the offending library
files be stripped of the BLAS functions.

\subsection{Comments about the FEI/\hypre{} Interface and Contacts}

Comments about \hypre{}'s finite element interface can be directed
to Charles Tong (925-422-3411, chtong@llnl.gov).

\end{document}

