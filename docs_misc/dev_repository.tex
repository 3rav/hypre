%==========================================================================
\chapter{The Repository}
\label{The Repository}

CVS is used as the resource control mechanism for \hypre{}.  This
allows developers to more easily coordinate their efforts, keep
previous revisions of software, and track changes and releases.

Some CVS commands you will use most often are:
\begin{itemize}
\item '\code{cvs checkout linear_solvers}' - checks out the most
recent version of the repository source into the current directory.
\item '\code{cvs update}' - updates your working version, merging
(or attempting to merge) your file changes with changes made (and
checked in) by other developers.  This will also give information on
current status of your changes, such as which files have been modified
(\code{M}), which files have been added (\code{A}), which files have
been removed (\code{R}), and which files cannot be accounted for
(\code{?}).
\item '\code{cvs commit}' - commits changes to all files in current
directory and all subdirectories.
\item '\code{cvs log <file>}' - prints revision history for a file.
\end{itemize}

It is recommended that developers set up a \file{.cvsrc} file in their
home directory with the following lines in it:
\begin{verbatim}
checkout -P
update -P -d
remove -f
\end{verbatim}
This sets default options for the CVS commands, \code{checkout},
\code{update}, and \code{remove} that make life much easier.
\begin{itemize}
\item The \code{-P} options cause the \code{checkout} and \code{update}
commands to ``prune'' (remove) directories that are empty.

\item The \code{-d} option will create any directories that exist in the
repository if they are missing in your working directory.  So, when
new directories are added to the repository, they will get added to
your working copy automatically when an \code{update} is done.

\item The \code{-f} option causes the remove command to first remove the
file from your working directory, then schedule it for removal from
the repository.  So, if you want to remove a file from the repository,
all you need to do is \code{cvs remove <file>}.
\end{itemize}

The main trunk of the repository must always pass the autotest
procedures outlined in \ref{Autotest Procedures}.  So, developers must
take extra care that checked-in code does not break autotest.  One
thing that should almost always be done is to do a clean checkout into
a temporary subdirectory (e.g., \code{~/tmp}), then do a
'\code{configure}' and a '\code{make beta}'.  This will at least
verify that the compile is not broken.  Extra measures should be taken
when necessary to insure that the autotest runs also do not fail.

%==========================================================================
\section{Using CVS Branches}
\label{Using CVS Branches}

Sometimes a large change needs to be made to the repository that
involves several developers coordinating their efforts.  Here, it
would be useful to use CVS to share code during the development
process, but this may temporarily break code that previously worked.
To avoid this, developers can use the branch mechanism in CVS.

WARNING: The use of branches can easily screw up the repository if
care is not taken.  Developers should follow the guidelines below to
reduce the risk of this.  Do not stray from these guidelines without
first consulting the software technical lead for \hypre{}.

SECOND WARNING: Branches should not be created and kept around for
long periods of time.  They should only be used *temporarily*.

\begin{itemize}

\item First, coordinate with the other developers involved to decide
on a branch name, and to insure that it is okay to start development
from the latest revision of the repository's main trunk.

\item Create a new branch (we will use the name '\code{foo}' here).
\begin{verbatim}
cvs rtag -b foo linear_solvers
\end{verbatim}

\item Create a working copy of the branch.
\begin{verbatim}
cd ~
mkdir foo
cd foo
cvs checkout -r foo linear_solvers
\end{verbatim}

\item Make modifications to files in '\code{~/foo/linear_solvers}'.
Code can be checked in/out without doing damage to the main
development trunk.

\item When the code is working, merge changes into the main trunk.
\begin{verbatim}
cd ~
cvs checkout linear_solvers
cvs update -j foo
\end{verbatim}

\item Verify that the merged files are okay, i.e. that the code compiles,
runs, etc.

\item Commit the changes to the main trunk.
\begin{verbatim}
cd ~/linear_solvers
cvs commit .
\end{verbatim}

\item Verify that the commit worked okay by doing a clean checkout, etc.

\item Remove the temporary tag, '\code{foo}', so that this branch is
not accidentally used to make additional changes to the repository.
\begin{verbatim}
cvs rtag -d foo linear_solvers
\end{verbatim}

\end{itemize}
